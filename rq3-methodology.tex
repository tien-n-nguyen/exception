\subsubsection{RQ3. Effectiveness on Exception Type Recommendation\\}

{\em Baselines.} We compared {\xstate} against the pre-trained
CodeBERT model without any fine-tuning step as in RQ1.

%We compared {\tool} against the state-of-the-art exception type recommendation approach XRank~\cite{nguyen2020code}.

{\em Procedure.} We used the same procedure as in RQ1.

%In this RQ, \tool is evaluated on the DeepEx dataset. Because only the data points that \tool can successfully predict all statements in the try-catch block successfully, the \tool will run the exception type prediction on them. Similarly, as RQ2, to fully use the dataset and evaluate \tool on more data, here we estimate the Accuracy of RQ2 prediction is 100\% which means that we use all positive data in DeepEx dataset as the data we use in this RQ. We do the same data split as RQ1 and RQ2 to train, validate and test the model performance.

{\em Tuning.} We used the same tuning as in RQ1.

{\em Metrics.} We used the same metrics as in RQ1. However, we
computed Recall, Precision, and F-score in two ways. First, we evaluate {\xtype}
in connection with {\xblock} and {\xstate}. That is, we consider a correct case if {xblock} 
gives a correct prediction on whether \code{try-catch} blocks are needed, {\xstate} produces 
correct tags for all the statements, and {\xtype} 
predicted correctly on the exceptions that need to be caught for all the \code{try-catch} blocks. 
Second, we evaluate {\xtype} as individual. That is, we assumed that both {\xblock} and {\xstate}
have given 100\% correct prediction results for their own tasks. Thus, we put the code snippets that
need \code{try-catch} blocks with statements correctly labelled in the oracle and used
{\xtype} to predict the exceptions to be caught for all such blocks.

%Let us use $E$ and $P$ to denote the set of the
%exception types in the oracle and the set of predicted
%ones for one code snippet in the testing set. We aim to measure 1) how
%precise {\tool} predicts the exception types ({\bf Precision}), i.e.,
%how accurate the predicted types in $P$, and 2) how much {\tool} can
%cover in its prediction with respect to the oracle ({\bf Recall}),
%i.e., how much of $E$ that the predicted set $P$ can cover. Toward
%measuring Precision and Recall, we define {\bf Hit-$n$} as the number
%of the cases in which the predicted set $P$ contains
%{\em at least} $n$ correct exception types, i.e., $P$ and $E$ overlaps
%{\em at least} $n$ exception types regardless of the sizes of both
%sets.
%
%In Github dataset, more than 98.1\% of the code snippets have 1--3
%exception types in a \code{catch} clause. Thus, we compute Hit-$n$,
%Precision, and Recall for the size of $E$ (|$E$|) from 1--3 and 3+,
%and for the size of $P$ (|$P$|) from 1--3 and 3+.
%%
%When computing Recall, we use the set $E$ as the basis. Recall at a
%size $N_E$ of $E$ is computed as the ratio between Hit-$n$ at that
%size and the total number of cases with that size $N_E$. We compute
%Recall for all $N_E$ = 1--3, and 3+, and $n$=1--$N_E$. We also define
%{\bf \code{Hit-All}$_{Rec}$} as \code{Hit}-$n$ when the number $n$ of
%overlapping exception types is equal to |$E$|, i.e., all exception
%types in the oracle set for a code snippet are covered.
%%
%When computing Precision, we use the set $P$ as the basis. Precision
%at a size $N_P$ of $P$ is computed as the ratio between Hit-$n$ at
%that size and the total number of cases with that size $N_P$. We
%compute Precision for all $N_P$ = 1--3, and $n$=1--$N_P$. We also
%define {\bf \code{Hit-All}$_{Prec}$} as \code{Hit}-$n$ when the number $n$
%of overlapping exception types is equal to |$P$|, i.e., all predicted
%types in the predicted set $P$ for a snippet are
%correct.


%In this RQ, we use \textbf{Hit-n} as the evaluation metrics. Hit-n
%here means within the true labeled statement set, there are at least
%\textbf{n} statements predicted correctly by \tool.
