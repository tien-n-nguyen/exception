\section{Introduction}
\label{sec:intro}

The online question and answering (Q\&A) forums, e.g., StackOverflow
(S/O) provide important resources for developers to learn how to use
software libraries and frameworks. While the code snippets in an S/O
answer are good starting points, they are often incomplete with
several missing details, even with ambiguous references, etc.  Zhang
{\em et al.}~\cite{zhang-icse19} have conducted a large-scale
empirical study on the nature and extent of manual adaptations of the
S/O code snippets by developers into their Github repositories.  They
reported that the adaptations from S/O code examples to their
Github counterpart projects are prevalent and non-trivial. They
qualitatively inspected all the adaptation cases and classified them
into 24 different adaptation types. They highlighted several
adaptation types including {\em type conversion}, {\em handling
  potential exceptions}, and {\em adding if
  checks}~\cite{zhang-icse19}. Among them, adding a \code{try-catch}
block to wrap the code snippet and listing the handled
exceptions in the catch clause are frequently performed,
%(132 cases out of 629 S/O examples),
yet not automated by existing tools.
%code integration techniques.

The adaptation process for exception handling is not trivial as Nguyen
{\em et al.}~\cite{xrank-fse20} have reported that it is challenging
for developers to learn and memorize what API methods could cause
exceptions and what exceptions need to be handled. Kechagia {\em et
  al.}~\cite{kechagia-msr14} found that 19\% of the crashes in Android
applications could have been caused by insufficient documented
exceptions in Android APIs. Thus, it is desirable to have an automated
tool to recommend proper exception handling for the adaptation of
online code snippets.

There exist several approaches to automatic recommendation of 
exception
handling~\cite{barbosa-bsse12,chanchal-scam14,barbosa-tse18,barbosa-tse16,xrank-fse20,throw-ase22}. They
can be classified into four categories. The first category of
approaches relies on a few {\em heuristics} on exception types, API
calls, and variable types to recommend exception handling
code~\cite{barbosa-bsse12}. These heuristic-based approaches do not
always work in all cases. The second category of approaches
utilized {\em exception handling policies}, which are enforced in all
cases~\cite{barbosa-tse16,barbosa-saner18}. However, the policies need
to be pre-defined and encoded within the recommending tools.  This is not
an ideal solution considering the fast evolution of software
libraries. To enable more flexibility than policy enforcement, the
third category leverages {\em mining algorithms} that derive
similar exception handling for two similar code
fragments~\cite{chanchal-scam14}. While avoiding hard-coding of
the rules, these mining approaches suffer the issue of how much similar
for two fragments to be considered as having similar exception
handling. For the mining approaches, deterministically setting a
threshold for frequent occurrences is also challenging.
%In fact, two API method calls in two different contexts might require
%to handle the same exception. For example, two fragments of code
%using JDK containing an opening of the files for different purposes
%in two contexts might need to catch and handle the \code{IOException}
%due to the call to \code{java.io.nio.new\-Buffered\-Reader}.

To provide more flexibility in code matching, the fourth category
follows the {\em information retrieval} (IR)
direction~\cite{xrank-fse20}. XRank~\cite{xrank-fse20} takes as input
a code fragment and recommends a ranked list of exceptions that need
to be handled in a \code{catch} clause. XHand~\cite{xrank-fse20}
recommends the exception handling code in a \code{catch} block for a
given code fragment. Both use a fuzzy set technique to compute the
associations between the API calls (e.g.,
\code{new\-Buffered\-Reader}) and the exceptions (e.g.,
\code{IOException}).

While the IR-based approach achieves higher accuracy than the
others~\cite{xrank-fse20}, it has key limitations. First, it is
not trivial to {\bf pre-define a threshold} for feature matching for a
retrieval of an exception type or an API element. The effectiveness of
those IR techniques depends much on the correct value of such
pre-defined threshold. Second, the IR-based techniques rely on the
lexical values of the code tokens and API elements, whose names can be
{\em ambiguous} in an incomplete code snippet. For example, the
\code{Document} class in \code{org.\-w3c.\-dom} of the W3C library has
the same simple name as the \code{Document} class in
\code{com.\-google.\-gwt.\-dom.\-client.\-Document} of the Google Web
Toolkit (GWT) library. An API method to open/write/read a
\code{Document} in the W3C library might need to catch a different set
of exceptions than the one in the GWT library. Those IR-based
techniques are not sufficiently flexible to handle such {\bf ambiguous
  names}. Third, the IR techniques {\em do not consider the {\bf
    context} of surrounding code}, thus, cannot leverage the {\em {\bf
    dependencies} among API elements} to resolve the ambiguity of the
names of API elements and exceptions.

In this paper, we propose {\tool}, a learning-based exception handling
recommender, to accept a given Java code snippet and recommend 1) {\em
  whether a try-catch block is needed for the snippet} ({\xblock}), 2)
{\em what statements need to be placed in a \code{try-catch} block}
({\xstate}) and 3) {\em what exception types need to be caught in the
  \code{catch} clause} ({\xtype}).  We find a motivation for such a
data-driven, learning-based approach from the previous studies
reporting that exception handling for the API elements is frequently
repeated across different
projects~\cite{chanchal-scam14,zhong-jss18}. The rationale is that the
designers of a software library have the intents for users to use
certain API elements with corresponding exception handling types.
Thus, we design {\tool} to learn from the exception handling
statements and exception types retrieved from complete source code in
a large code corpus, and derive the exception handling for the
(partial) code snippet under study.

%no threshold
%context: ambiguity and connection (complete code)
%span...

{\tool} is designed to capture the basic insights to overcome key
limitations of the state-of-the-art approaches. First, with the
learning-based approach, {\tool} does not rely on a pre-defined
threshold for explicit feature matching for the retrieval of API
elements or exception types. Second, instead of learning only the
associations between API elements and corresponding exception types,
we also consider the code context surrounding them. This gives {\tool}
two advantages. During training, the complete code enables the
identifications (i.e., the fully-qualified names) of the API
elements. During predicting, for the given incomplete code, the
context enables {\tool} to {\em learn the identities of the API
  elements via the dependencies/relations} among them in the context,
thus, avoiding the name ambiguity. Let us call it {\em dependency
  context}. Moreover, the context of surrounding code also helps the
model implicitly learn the important features to connect between the
API elements and the corresponding exception types.

%{\em We design a ML model that considers exception handling recommendation
%as a dual-task learning between recommending the statements for
%\code{try-catch} block ({\xstate}) and recommending exception types to
%be handled ({\xtype}). The learning for {\xstate} and that for
%{\xtype} can be mutually beneficial. If a model learns what statements
%with certain API method calls to be included in a \code{try-catch}
%block (e.g., \code{inputStream.read}), based on the complete code in
%the corpus, it can learn that certain exception types need to be
%handled (e.g., \code{IOException}). In contrast, if a model learns
%what exceptions need to be handled
%(\code{Index\-Out\-Of\-Bounds\-Exception}), it will certainly be
%easier to decide whether to include a statement in a \code{try-catch}
%block (e.g., a statement with
%\code{java.\-lang.\-String.\-sub\-String}).} {\bf MORE...}

We leverage Graph Convolutional Network (GCN) to represent the program
dependence graph (PDG) to capture the control and data dependencies
among the API elements in the surrounding context. Learning the
dependencies via GCN enables us to realize the key idea {\em ``Tell Me
  Your Friends, I'll Tell You Who You Are''} to learn the identities
of the API elements in a given incomplete code, leading to better
learning in {\xblock}, {\xstate} and {\xtype}. Moreover, we treat the
problem of predicting a \code{try-catch} block as a classification
(Y/N) as well as the problem of deriving the handled exception types
as a classification among the exceptions.
%
To predict which statements in a given code snippet that need to be
wrapped in a \code{try-catch} block, we leverage the explainable ML
model, GNNExplainer~\cite{GNNExplainer}, that {\em ``explains'' on why
  the GCN model has arrived at its decision}. Specifically, {\tool}
takes as input the GCN model along with its decision on whether a
\code{try-catch} block is needed, and the input PDG $G_M$ of the given
code $M$. GNNExplainer will learn the output explainable subgraph,
which is defined as a minimal sub-graph $\mathcal{G}$ in the PDG of
$M$ that {\em minimizes the prediction scores between using the entire
  $G_M$ and using $\mathcal{G}$}. The idea is that if an edge is {\em
  removed} from $G_M$, and {\em the decision of the model is affected,
  then the edge is crucial and must be included in the explanation for
  the detection result}. Thus, the minimal sub-graph $\mathcal{G}$ in
PDG contains the nodes and edges, {\em i.e.}  including the {\em
  crucial statements} that are most decisive/relevant to the
classification on exception types. We consider them as the statements
that need to be placed in a \code{try-catch} block.

We have conducted several experiments to evaluate {\tool}. We have
collected a large dataset of XXX projects from Github, with XX,XXX
methods and XX,XXX \code{try-catch} blocks. Our empirical evaluation
shows that {\tool} achieves a high accuracy of XX\% in predicting if a
code snippet needs a \code{try-catch} block. It also achieves a high
accuracy of XX\% in recommending which statements in the snippet to be
placed in a try-catch block. Our result shows that {\xtype} relatively
improves over the state-of-the-art approach in exception type
recommendation from XX\%–YY\% in accuracy.

In brief, this paper makes the following major contributions:

\noindent {\bf 1. [Neural Network-based Automated Exception Handling
    Recommendation]}. {\tool} is the first neural network approach to
  automated exception handling recommendation for both complete as
  well as partial code.

  \noindent {\bf 2. [Graph Neural Network and Explainable AI (XAI)]} We
  leverage Graph Convolutional Network to capture the program
  dependencies among API elements and exceptions and leverage XAI
  to derive the statements to be placed in a \code{try-catch} block.

\noindent {\bf 3. [Empirical Evaluation]}. We conducted an extensive
empirical evaluation showing {\tool}'s high accuracy in exception
handling recommendation for both statements and exception types.

  
