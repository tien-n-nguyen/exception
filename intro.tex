\section{Introduction}
\label{sec:intro}

The online question and answering (Q\&A) forums, e.g., StackOverflow
(S/O) provide important resources for developers to learn how to use
software libraries and frameworks. While the code snippets in an S/O
answer are good starting points, they are often incomplete with
several missing details and contain ambiguous references, etc.  Zhang
{\em et al.}~\cite{zhang-icse19} have conducted a large-scale
empirical study on the nature and extent of manual adaptations of the
S/O code snippets by developers into their GitHub repositories.  They
reported that the adaptations between S/O code examples and their
GitHub counterpart projects are prevalent and non-trivial. The
qualitatively inspected all the adaptation cases and classified them
into 24 different adaptation types. They highlighted several
adaptation types including {\em type conversion}, {\em handling
  potential exceptions}, and {\em adding if
  checks}~\cite{zhang-icse19}. Among them, adding a \code{try-catch}
block to wrap the code snippet and enumerating all the handled
exceptions in the catch clause are frequently performed (132 cases out
of 629 S/O examples) yet not automated by existing code integration
techniques.

The adaptation process for exception handling is not trivial as Nguyen
{\em et al.}~\cite{xrank-fse20} have reported that it is challenging
for developers to learn and memorize what API method could cause
exceptions and what exceptions need to be handled. Kechagia {\em et
  al.}~\cite{kechagia-msr14} found that 19\% of the crashes in Android
applications could have been caused by insufficient documented
exceptions in Android APIs. Thus, it is desirable to have an automated
tool to recommend proper exception handling for such adaptation of
online code snippets.

There exist several approaches to automatically recommend proper
exception
handling~\cite{barbosa-bsse12,chanchal-scam14,barbosa-tse18,barbosa-tse16,xrank-fse20,throw-ase22}. They
can be classified into different categories. The first category of
approaches relies on a few {\em heuristics} on exception types, API
calls, and variable types to recommend exception handling
code~\cite{barbosa-bsse12}. These heuristic-based approaches do not
always work in all the cases. The second category of approaches
utilized {\em exception handling policies}, which are enforced in all
cases~\cite{barbosa-tse16,barbosa-saner18}. However, the policies need
to be pre-defined and encoded within the enforcing tools.  This is not
an ideal solution considering the fast evolution of software
libraries. To enable more flexibility than policy enforcement, the
third category leverages the {\em mining algorithms} that derives
similar exception handling for two similar code
fragments~\cite{chanchal-scam14}. While avoiding the hard-coding of
rules, these mining approaches suffer the issue of how much similar
for two fragments to be considered as having similar exception
handling. For the mining approaches, deterministically setting a
threshold for frequent occurrences is challenging.
%In fact, two API method calls in two different contexts might require
%to handle the same exception. For example, two fragments of code
%using JDK containing an opening of the files for different purposes
%in two contexts might need to catch and handle the \code{IOException}
%due to the call to \code{java.io.nio.new\-Buffered\-Reader}.

To provide more flexibility in matching the code, the fourth category
follows the {\em information retrieval} (IR)
direction~\cite{xrank-fse20}. XRank~\cite{xrank-fse20} takes as input
a code fragment and recommends a ranked list of exceptions that need
to be handled in a \code{catch} clause. XHand~\cite{xrank-fse20}
recommends the exception handling code in a \code{catch} block for a
given code fragment. Both uses a fuzzy set technique to learn the
associations between the method calls (e.g.,
\code{new\-Buffered\-Reader}) and the exceptions (e.g.,
\code{IOException}). While the IR-based approach achieves higher
accuracy than others~\cite{xrank-fse20}, it has key limitations that
reduce its effectiveness. First, it is not trivial to {\bf pre-define
  a threshold} for feature matching for a retrieval of an element
(e.g., an exception or API method). The effectiveness of those IR
techniques depends much on the correct value of such pre-defined
threshold. Second, the IR-based techniques rely on the lexical values
of the code tokens and API elements, whose names can be {\em
  ambiguous} in an incomplete code snippet. For example, the
\code{Document} class in \code{org.\-w3c.\-dom} of the W3C library has
the same simple name as the \code{Document} class in
\code{com.\-google.\-gwt.\-dom.\-client.\-Document} of the Google Web
Toolkit (GWT) library. An API method to open/write/read a
\code{Document} in the W3C library might need to catch a different set
of exceptions than the one in the GWT library. Those IR-based
techniques are not sufficiently flexible to handle such {\bf ambiguous
  names}. Third, the IR techniques {\em do not consider the {\bf
    context} of surrounding code}, thus, cannot leverage the context
of {\em interdependent API elements} to resolve the ambiguity of API
methods and exceptions.

In this paper, we propose {\tool}, a learning-based exception handling
recommender, to accept a given Java code snippet, and 1) to recommend
{\em what statements need to be placed in a \code{try-catch} block}
and 2) to recommend {\em what exception types need to be caught in the
  \code{catch} clause}.  We find a motivation for such a data-driven,
learning-based approach from the previous studies that exception
handling for the API elements is frequently repeated across different
projects~\cite{chanchal-scam14,zhong-jss18}. The rationale is that the
designers of a software library have the intents for users to use
certain API elements with corresponding exception handling types.
Thus, we design {\tool} to learn from the exception handling
statements and types retrieved from complete source code in a large
code corpus, and derive the exception handling for the (partial) code
snippet under study.

%no threshold
%context: ambiguity and connection (complete code)
%span...

{\tool} is designed to capture the basic insights to overcome key
limitations of the state-of-the-art approaches. First, with the
learning-based approach, {\tool} does not rely on a pre-defined
threshold for explicit feature matching for the retrieval of API
elements or exception types. Second, instead of learning only the
associations between API elements and corresponding exception types,
we also consider the code context surrounding them. This gives {\tool}
two advantages. (1) During training, the complete code enables the
identifications (i.e., the fully-qualified names (FQNs)) of the API
elements. During predicting, for the given incomplete code, the
context enables {\tool} {\em learn the identities via the
  dependencies/relations} among those API elements in the context,
thus, avoiding the name ambiguity. Let us call it {\em dependency
  context}. (2) the context of surrounding code also helps the model
implicitly learn the important features to connect between the key API
elements and the corresponding exception types.

We design a learning-based model ...

We have conducted several experiments to evaluate {\tool}. ...

In brief, this paper makes the following major contributions:

\noindent {\bf 1. [Neural Network-based Automated Exception Handling
    Recommendation]}. {\tool} is the first neural network approach to
  automated exception handling recommendation for both complete as
  well as partial code.

\noindent {\bf 2. [Empirical Evaluation]}. We conducted an extensive
empirical evaluation showing {\tool}'s high accuracy in exception
handling recommendation for both statements and exception types.

  
