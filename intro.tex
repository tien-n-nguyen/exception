\section{Introduction}
\label{sec:intro}

The online question and answering (Q\&A) forums, e.g., StackOverflow
(S/O) provide important resources for developers to learn how to use
software libraries and frameworks. While the code snippets in an S/O
answer are good starting points, they are often incomplete with
several missing details and contain ambiguous references, etc.  Zhang
{\em et al.}~\cite{zhang-icse19} have conducted a large-scale
empirical study on the nature and extent of manual adaptions of the
S/O code snippets by developers into their GitHub repositories.  They
reported that the adaptations between S/O code examples and their
GitHub counterpart projects are prevalent and non-trivial. The
qualitatively inspected all the adaptation cases and classified them
into 24 different adaptation types. They highlighted several
adaptation types including {\em type conversion}, {\em handling
  potential exceptions}, and {\em adding if
  checks}~\cite{zhang-icse19}. Among them, adding a \code{try-catch}
block to wrap the code snippet and enumerating all the handled
exceptions in the catch clause are frequently performed (132 cases out
of 629 S/O examples) yet not automated by existing code integration
techniques.

The adaptation process for exception handling is not trivial as Nguyen
{\em et al.}~\cite{xrank-fse20} have reported that it is challenging
for developers to learn and memorize what API method could cause
exceptions and what exceptions need to be handled. Kechagia {\em et
  al.}~\cite{kechagia-msr14} found that 19\% of the crashes in Android
applications could have been caused by insufficient documented
exceptions in Android APIs. Thus, it is desirable to have an automated
tool to recommend proper exception handling for such adaptation of
online code snippets.
