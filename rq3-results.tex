\subsection{Exception Type Recommendation (RQ3)}
\label{sec:rq3}

\begin{table}[t]
  \caption{Detailed Results on Different Numbers of Exception Types in a
    \code{catch} clause in Oracle Set (RQ3) (Recall)}
  \vspace{-12pt}
	{\small
	  \begin{center}
            \tabcolsep 3.5pt
			\renewcommand{\arraystretch}{1}
			\begin{tabular}{p{4.5cm}<{\centering}|p{1.3cm}<{\centering}|p{1.5cm}<{\centering}}
				\hline
				\# Ex. Types in \code{catch} clause in Oracle Set & Metrics &  Accuracy\\
				\hline
				\multirow{1}{*}{1 (40,376 instances)}   & Hit-1  & 25,841 (64\%) \\
				\hline
				\multirow{2}{*}{2 (2,634 instances)}  & Hit-1   & 2,028 (77\%) \\
				& Hit-2         &  1,081 (41\%) \\
				\hline
				\multirow{3}{*}{3 (547 instances)}  & Hit-1    & 334 (61\%) \\
				& Hit-2     & 181 (33\%)\\
				& Hit-3     & 115 (21\%) \\
				\hline
				\multirow{4}{*}{3+ (391 instances)}  & Hit-1   & 233 (59\%) \\
				& Hit-2     & 134 (35\%) \\
				& Hit-3     & 65 (17\%)\\
                                \hline
                                & Hit-All$_{Rec}$ & 27,037 (62\%)\\
				\hline
			\end{tabular}
	                The sum of the percentages in each row section is not 100\% because Hit-1 contains Hit-2, which contains Hit-3, and so on. Hit-$n$: the overlap between the predicted set and the oracle set is at least $n$ exception types.
			\label{tab:recall-3}
		\end{center}
	}
\end{table}

\subsubsection{{\bf Recall on Exception Type Recommendation}}
\label{sec:req3-recall}

Table~\ref{tab:recall-3} displays the {\em Recall} result on how well
{\em {\tool} covers on the actual exception types in the \code{catch}
  clauses in the oracle}. The result is with respect to the instances
(code snippets) in the dataset with different numbers $K$ of exception
types in the \code{catch} clauses: $K$ = the number of Exception Types
in a \code{catch} clause in the oracle = $1,2,3,3+$. For example, in the
oracle, there are 547 instances with 3 exception types in a
\code{catch} clause. {\tool}'s predicted set correctly contains all 3
exception types for 115 instances (21\%) (\code{Hit-All}). The
predicted set contains {\em at least 2 out of 3} for 181 instances
(33\%), and contains {\em at least} one exception type for 334 instances
(61\%).

%\begin{table}[t]
%	\caption{Results on Recall for Exception Types in a Try-Catch Block in Oracle Set (RQ3)}
%	{\small
%		\begin{center}
%			\renewcommand{\arraystretch}{1}
%			\begin{tabular}{p{1cm}<{\centering}|p{1.5cm}<{\centering}}
%				\hline
%				Metrics & Accuracy \\
%				\hline
%                                Hit-1 & 28,436 (64.7\%)\\
%                                Hit-2 & 1,396\\
%                                Hit-3 & 180\\
%                                Hit-All & 27,037\\
%			\end{tabular}		
%			\label{tab:recall-3-sum}
%		\end{center}
%	}
%\end{table}

%Total: 43,948

%As seen in Table~\ref{tab:recall-3},

There are 40,376 instances (92\%) over 43,948 total instances with a
single exception type in the \code{catch} clause in the
oracle. {\tool} covers correctly the exception type in 25,841
(64\%) of the instances. There are 2,634 instances with two exception
types in the oracle and {\tool} correctly suggests both types in
1,081 (41\%) of the~cases.

%In 2,028 instances (77\%), at least one exception type is predicted correctly.

Note that \code{Hit-All}$_{Rec}$ = \code{Hit-$n$} when $n$ (the number
of overlaps between the predicted and oracle sets) = $K$ (the number
of exception types). For $K$=1..3, which is a total of 43,557
instances (99.1\%), {\bf {\tool} covers all the exception types
  (\code{Hit-All}$_{Rec}$) in 27,037 instances (62\%)}. That is,
developers do not have to search for other exception types in 62\% of
the cases. Thus, it achieves high \code{Hit-All}$_{Rec}$ in 99.1\% of
  the entire dataset.

%{\color{red}{1. Our model do well on hit-1 condition which means in most cases, our model can at least predict one exception type correctly for a try-catch block. 2. Results show that the more exception types one try-catch block has, our model is harder to predict all of them correctly. }}

\subsubsection{{\bf Precision on Exception Type Recommendation}}
\label{sec:req3-precision}


\begin{table}[t]
  \caption{Detailed Results on Different Numbers of Exception Types in a \code{catch} clause in Predicted Set (RQ3) (Precision)}
	\vspace{-12pt}
	{\small
	  \begin{center}
            \tabcolsep 2pt
			\renewcommand{\arraystretch}{1}
			\begin{tabular}{p{5cm}<{\centering}|p{1.4cm}<{\centering}|p{1.6cm}<{\centering}}
				\hline
				\# Ex. Types in \code{catch} clause in Predicted Set & Metrics & Accuracy \\
				\hline
				\multirow{1}{*}{1 (36,995 instances)}   & Hit-1  & 13,589 (37\%) \\
				\hline
				\multirow{2}{*}{2 (4,981 instances)}  & Hit-1   & 1,594 (32\%) \\
				& Hit-2       						& 947 (19\%) \\
				\hline
				\multirow{3}{*}{3 (1,972 instances)}  & Hit-1    & 611 (31\%) \\
				& Hit-2         					& 458 (23\%)\\
				& Hit-3         				  	& 185 (9\%) \\
                                \hline
%                                \multirow{4}{*}{3+ (XXX instances)}  & Hit-1   & XXX (XX\%) \\
%				& Hit-2     & XXX (XX\%) \\
%				& Hit-3     & XX (XX\%)\\
%                                \hline
                                & Hit-All$_{Prec}$ & 14,721 (33.5\%)\\
				\hline
			\end{tabular}
                        The sum of the percentages in each row section is not 100\% because Hit-1 contains Hit-2, which contains Hit-3, and so on. Hit-$n$: the overlap between the predicted set and the oracle set is at least $n$ exception types.
			\label{tab:precision-3}
		\end{center}
	}
\end{table}

Table~\ref{tab:precision-3} shows the {\em Precision} result.
%on how well {\tool} correctly predicts the actual exception types in
%\code{catch} clauses in the oracle.
The result is shown with respect to different numbers $K$ of {\em
  predicted exception types} in the \code{catch} clauses. For example,
as seen in Table~\ref{tab:precision-3}, there are a total of 1,972
instances (code snippets) in which {\tool} predicted three exception
types in a \code{try-catch} block. Among them, with \code{Hit-1}=31\%,
there are 611 instances in which at least one predicted exception type
is correct. With \code{Hit-2}=23\%, there are 458 instances in which
at least two of the predicted exception types are correct. With
\code{Hit-3} = 9\%, there are 185 instances that all three predicted
exception types are correct (some types might be missing).  Producing
the exact-matched sets (\code{Hit-All}) for all exception types when
$K \ge$ 3 is still challenging, however, those cases are only 0.7\% in
the entire~dataset. We do not have the row for 3+ because in our
dataset, 99.3\% of the cases have $\leq$3 exception types, and we
set 3 as the limit for {\xtype}.

Importantly, \code{Hit-All}$_{Prec}$ = \code{Hit-$n$} when
$n$ (the number of overlaps between the predicted and oracle sets) =
$k$ (the number of exception types predicted by {\tool}).
For $k$=1..3, which is a total of 43,948 instances (99.3\%), {\bf
  {\tool} predicts correctly all the exception types
  (\code{Hit-All}$_{Prec}$) in 14,721 instances (33.5\%)}. That is, in
14,721 instances (i.e., 33.5\%), all the predicted exception types are
actually the correct ones. Developers do not have to delete any
exception types in a predicted set, thus, saving their efforts.




%For example, when the number of the CC statements in a p redicted set
%is 𝐾 ′=3, there are 23 bugs in which all of those 3 faulty statements
%are correct (there might be other statements missing). There are 27
%bugs in which two of the 3 predicted, faulty statements are
%correct. There are 55 bugs in which only one of the 3 predicted,
%faulty statements are correct. As seen, regardless of 𝑁, FixLocator is
%more precise than the baselines for all 𝐾 ′s.
