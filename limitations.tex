%\vspace{3pt}

\subsection{\bf Limitations and Threats to Validity}

\subsubsection{Limitations}

{\tool} still has several limitations.
%First, {\tool} can not handle the code with multiple \code{try-catch}
%blocks.
First, it cannot generate new exception types that were not in
the training corpus. Second, it does not support the generation of
exception handling code inside the \code{catch} body. Each
project might have a different way to handle exception types in
the \code{catch} body. Thus, we chose not to support that function.
Third, {\tool} needs training data, thus, does not work for a new
library without any API usage~yet. Fourth, it is possible that the
model produces the labels \code{O}, \code{B-Try}, and \code{I-Try}
that do not correspond to the legal way of a \code{try-catch}
block. Finally, {\xblock} might produce a conflicting result with
{\xstate}.

\subsubsection{Threats to Validity}

{\tool} is specifically for Java. Our empirical evaluation was only on
Android and JDK libraries. The results might vary for other
libraries. Our data might not be representative. However, we
used well-established projects and libraries.  We used only a balanced
dataset, which might not reflect well the ratio in practice.
FuzzyCatch~\cite{xrank-fse20} does not suggest \code{try} block
statements and exceptions, thus, we compared only with XRank (part of
FuzzyCatch). For GPT-3.5, we ran only on a sampled dataset.
