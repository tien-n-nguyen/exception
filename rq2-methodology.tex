\subsubsection{RQ2. Effectiveness on Try-Catch Statement Detection}~\\
\indent {\em Baselines.} We compared {\xstate} against GPT-3.5 as in RQ1.

%the pre-trained CodeBERT model without any fine-tuning step as in RQ1.

%{\em Procedure.} We used the same procedure as in RQ1.

%{\em Tuning.} We used the same tuning as in RQ1.

{\em Procedure and Metrics.} We evaluated the models at both the
instance level and the statement level. At the instance level, a
prediction for a code snippet is considered as correct if all the
statements inside or outside of all the predicted (zero or
multiple) \code{try-catch} blocks must match with the grouping of the
corresponding statements in the corresponding (zero or
multiple) \code{try-catch} blocks in the oracle. To do so, we match
the encoded vector [O, B-Try, I-Try] of the prediction against the
vector for a snippet in the oracle. Because we have both
positive/negative instances, we used the same metrics Precision,
Recall, and F1-score as in RQ1. At the statement level, we evaluated
if a model predicts correctly whether a statement needs to be inside
a \code{try-catch} block, regardless of the blocks themselves. Thus,
we use {\em Accuracy} for the statement-level evaluation, which is
defined as the ratio between the number of correct tagging of
statements over the total number of statements.  Importantly, we also
evaluated {\tool} in two ways. First, we evaluated {\xstate} in
connection with {\xblock}. That is, we consider a case as correct if
both {\xblock} and {\xstate} give correct predictions (correct on the
need of a \code{try-catch} block and correct on statement
tagging). Second, we also evaluated {\xstate} as individual. We assume
that {\xblock} predicted correctly on the positive instances. We
evaluated {\xstate} individually at both instance and statement levels
as explained.

%We used the same metrics as in RQ1 at the instance (snippet) level.
%At the statement level, we use {\em Accuracy}, which is the ratio
%between the number of correct tagging of statements over the total
%number of statements. We computed Recall, Precision, and F-score in
%two ways. First, we evaluated {\xstate} in connection with
%{\xblock}. That is, we consider a correct case if {\xblock} gives a
%correct prediction on whether
%\code{try-catch} blocks are needed, {\em and} {\xstate} produces
%correct tags for all the statements when there exists
%a \code{try-catch} block. Second, we evaluated {\xstate} as
%individual. We assumed that {\xblock} predicts correctly whether a
%code snippet needs \code{try-catch} block(s). We used {\xstate} to
%predict the groupings of statements for the blocks in those snippets
%and compared against the statements in one or multiple blocks in the oracle (vian encoded vector [O, B-Try, I-Try]). We also computed the accuracy
%at the statement level on if a statement needs a \code{try-catch}
%block.
%----------------------------------------

%For a snippet that needs \code{try-catch}, but was predicted as
%not, we consider it as incorrect because the resulting statements
%predicted from {\xstate} do not make sense for incorrect {\xblock}'s
%detection. A snippet that does not need \code{try-catch}, but were
%predicted as yes, we also consider them as incorrect for the same
%reason. For a snippet that does not need a \code{try-catch} block,
%and was predicted as not needing, we consider it as correct.
%For a snippet needing such a block, and was predicted as yes,
%we consider if the current statement was predicted correctly or not.

%Thus, for the evaluation of {\xstate}, we used the set of code
%snippets that {\xblock} predicted correctly as needing a
%\code{try-catch} block. We used the same data splitting scheme with
%80\%, 10\%, and 10\% for training, tuning, and testing as in RQ1. Each
%code snippet in the testing set and the trained R-GCN model in
%\xblock} are the input of the GNNExplainer model in {\xstate} in this
%experiment.

%{\em Tuning.} We used the same parameters as in
%GNNExplainer~\cite{GNNExplainer}. It also has a parameter on the limit
%of the number $N$ of the nodes in the explanation sub-graph. We
%varied $N$ from 1 to 10. The average number of statements in a
%\code{try-catch} block in Github dataset is 5.9.

%{\em Metrics.} If {\xstate} predicts for a statement correctly if it
%is in a \code{try-catch} block or not, we count it as a correct case.
%Otherwise, it is an incorrect one. \textbf{Accuracy} is
%defined as the ratio between the number of correct statements over the
%total number of statements.
