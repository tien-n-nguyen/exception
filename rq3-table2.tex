\vspace{-12pt}
\begin{table}[htpb]
  \caption{Exception Type Recommendation Comparison ({\tool} = {\xblock}+{\xstate}+{\xtype}) (RQ3)}
  \vspace{-12pt}
  \tabcolsep 3pt
  \small
	\begin{center}
		\renewcommand{\arraystretch}{1}
		\begin{tabular}{| p{4.5cm}<{\centering} | p{1cm}<{\centering} | p{1cm}<{\centering}| p{1cm}<{\centering}|}
		  \hline
			Small dataset  & Precision  & Recall & F1-score \\
			\hline
                        %			CodeBERT w/o fine-tuning &  0.0 & 0.0  & 0.0\\
                        % GPT-3.5 & 0.539 & 0.544 & 0.541 \\
                        GPT-3.5 & 0.492 & 0.181 & 0.264 \\
			\hline
			{\tool} = \xblock + \xstate  + \xtype  & \textbf{0.724}  &  \textbf{0.682} & \textbf{0.702}\\
			\hline
		\end{tabular}
		\label{tab:xtype-2}
	\end{center}

\end{table}

Finally, in comparison with GPT-3.5, as seen in
Table~\ref{tab:xtype-2}, {\tool} as evaluated as three components,
achieves relatively higher in Precision, Recall, and F1-score with
{\em 47.1\%, 278\%, and 166\%, respectively}. In 190 positive
instances, {\tool} predicted correctly all three tasks in 30
instances. In all 380 instances, it predicted correctly all
three tasks in 219 instances, resulting a total of 57.6\%.

Examining the result from GPT-3.5, we reported that for popular APIs,
it works well, e.g., \code{ClassNotFoundException},
\code{IllegalArgumentExcept\-ion}, \code{IOException},
\code{IndexOutOfBoundException}, etc. For
unpopular API calls, it resorted to using the general exception
\code{Exception} (694 in total, 58.6\%) or \code{APIException}
as an answer. For the cases of multiple \code{try-catch} blocks, the
common errors are the use of \code{Exception} for all the blocks.

