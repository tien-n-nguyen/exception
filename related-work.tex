\section{Related Work}
\label{sec:related}

The automated approaches to recommend exception handling can be
classified into four categories as presented in Section~\ref{sec:intro}.
%The first category of approaches defines the {\em heuristics} on
%exception types, API calls, and variable types to recommend proper
%exception handling~\cite{barbosa-bsse12}.
%%These heuristic-based approaches do not always work in all the cases
%%and need to be updated for the new/updated API elements or a new
%%library. To address the heuristics,
%The second category of approaches relied on the enforced {\em
%  exception handling policies}~\cite{barbosa-tse16,barbosa-saner18}.
%%Exception policies are defined via domain-specific
%%language~\cite{barbosa-tse16}.
%%Exception Policy Expert (EPE)~\cite{barbosa-saner18} is a tool
%%embedded in Eclipse IDE that warns developers about exception policy
%%violations. The drawback of this category is the hard-code of the
%%policies in the EPE tool.
%%To overcome the pre-defined exception handling policies,
%The third category of approaches aim to {\em mine the frequent
%  exception handling} code from a large corpus. These mining
%approaches~\cite{chanchal-scam14} provide more flexibility than policy
%enforcement.
%%They rely on the idea that similar code has similar exception
%%handling. As with mining, these approaches face the challenge of
%%deterministically setting the threshold for similar code and similar
%%exception handling. In many cases, the two pieces of code are quite
%%different, yet have the same set of exceptions to be handled.
The closest work to {\tool} is the state-of-the-art {\em information
  retrieval} (IR) approaches~\cite{xrank-fse20}, which provides more
flexibility than the others. XRank~\cite{xrank-fse20} recommends a
ranked list of API calls that might need exception handling and
XHand~\cite{xrank-fse20} recommends exception handling code. Both
leverages fuzzy set theory to compute the associations between API
method calls and the exception types. This direction has three key
limitations. First, one needs to pre-define a threshold for feature
matching for the retrieval of API elements or exception types. Second,
the IR techniques are not flexible as the ML approaches because they
use the lexical values of API simple names. Thus, they suffer the
ambiguity in the names of API elements in incomplete code
snippets. Lastly, XRank/XHand considers only pairwise associations
between the API method calls and exceptions. It disregards the
surrounding code context and the dependencies/relations. XRank/XHand
simply uses Groum~\cite{fse09}, a dependency graph among API elements,
to collect the API calls, but did not use dependencies in
computing the association scores.
%among the API elements to disambiguate the simple names of the API
%elements.

In addition to exception handling recommendation research,
ThEx~\cite{throw-ase22} predict which exception(s) shall be thrown
under a given programming context. ThEx learns a classification model
from existing thrown exceptions in different contexts.

%The features include code information surrounding the thrown
%exceptions, such as the thrown locations and related variable names.

