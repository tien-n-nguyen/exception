\subsection{Exception-Related Bug Detection (RQ5)}
\label{sec:rq1}

\begin{table}[t]%[htpb]
  \caption {Exception-Related Bug Detection (RQ5)}
  \vspace{-12pt}
  \small
	\begin{center}
		\renewcommand{\arraystretch}{1}
		\begin{tabular}{|p{1.75cm}<{\centering}|p{1.75cm}<{\centering}|p{1.75cm}<{\centering}|}
		  \hline
			\multirow{2}{*}{} & \multicolumn{2}{c|}{FuzzyCatch Dataset} \\
			\cline{2-3}
			  & \tool  & FuzzyCatch~\cite{xrank-fse20} \\
			\hline
			Recall    & 0.75 & \textbf{0.79}\\
			Precision & \textbf{0.62} & 0.54\\
			F-score   & \textbf{0.68} & 0.64\\
			\hline
		\end{tabular}
		\label{tab:bug}
	\end{center}
\end{table}

%{\color{red}{This section waiting for the XRank Results. But from the current estimate, our approach should have higher F-score. But the recall and precision I'm not sure. Once I have the results, I will update this section.}}

Table~\ref{tab:bug} displays the comparison result. As seen, {\tool}
can be used to detect well real-world exception-related bugs in which
the code snippet needs but did not have the \code{try-catch} blocks or
miss some exceptions. In comparison, {\tool} improves over the
state-of-the-art FuzzyCatch, an IR approach, {\bf 14.8\%} in Precision,
{\bf -5.1\%} in Recall, and {\bf 5.8\%} in F-score. The examples of buggy
code and fixes with the addition of \code{try-catch} blocks are
available at FuzzyCatch's code repository: ebrand.ly/ExDataset.
