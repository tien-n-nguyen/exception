\subsection{Comparison on \code{Try-Catch} Block Necessity Checking Effectiveness (RQ1)}
\label{sec:rq1}

\begin{table}[t]%[htpb]
  \caption{Try-Catch Necessity Checking Comparison (RQ1)}
  \vspace{-12pt}
	\begin{center}
		\renewcommand{\arraystretch}{1}
		\begin{tabular}{|p{1.5cm}<{\centering}|p{1.25cm}<{\centering}|p{1.25cm}<{\centering}|p{1.25cm}<{\centering}}
		  \hline
			\multirow{2}{*}{} & \multicolumn{2}{c|}{Github Dataset} \\
			\cline{2-3}
			  & \tool  & XRank \\
			\hline
			Recall    & \textbf{0.81} & \\
			Precision & \textbf{0.66} & \\
			F-score   & \textbf{0.73} & \\
			\hline
		\end{tabular}
		\label{tab:xblock}
	\end{center}
\end{table}

%{\color{red}{This section waiting for the XRank Results. But from the current estimate, our approach should have higher F-score. But the recall and precision I'm not sure. Once I have the results, I will update this section.}}

Table~\ref{tab:xblock} displays the comparison result. As seen, our
approach achieves a high level of performance across two
datasets. With a Precision of 66\%, it can decide correctly 2 out of 3
cases whether a code snippet requires a \code{try-catch} block or
not. With a Recall of 81\%, {\tool} is able to cover 4 out of 5 cases
that needs to be placed in a \code{try-catch} block. Users just need
to find 1 out of 5 cases. As a result, it achieves a high F-score of
0.73.
%In FuzzyCatch dataset, {\tool} also achieves a high level of
%performance with XX\% precision, YY\% recall, and ZZ\% F-score.
In comparison, {\tool} improves relatively over the state-of-the-art
approach, XRank, {\bf XX\%} in Recall, {\bf YY\%} in Precision, and
{\bf ZZ\%} in F-score.

%In FuzzyCatch dataset, the relative improvements are XX\%, YY\%, and
%ZZ\% in precision, recall, and F-score, respectively.

We examined closely the cases that {\tool} performed better than
XRank.  First, XRank relies on the association scores between the
presence of API method calls and the presence of a \code{try-catch}
block. The decisions on the necessity of a \code{try-catch} block or
the exception types depend on the pre-defined thresholds in XRank on
those association scores. Thus, those pre-defined thresholds might not
be suitable across all API methods. Second, for the incomplete
code snippets in which the names of the API methods in different
packages or libraries are the same (e.g., \code{toString} or
\code{getText} in various JDK packages), XRank cannot distinguish them
and use one entry in the dictionary for them due to its IR
approach. In contrast, unlike XRank which considers only the API
method calls in a \code{try-catch} block, {\tool} considers the code in
the block as the context to learn the program dependencies/relations among the
names of those API elements. That is, it leverages the relations among
the names of API elements to learn their identities, thus,
deciding better the need of \code{try-catch} blocks and the
corresponding exception types.

%Tien

Take as an example a code snippet (not shown) in our dataset with the
presence of \code{getText}. This name is popular with a very large number
of API method candidates.
%For example, in a code snippet, \code{getText} has a very large number
%of API method candidates.
However, considering the relation between \code{css} and
\code{getText} in the code \code{`...css()\-.getText()'}, we only have
4 candidates for \code{getText}. Finally, considering the return
value of \code{getText} as an argument of \code{setInnerText(...)} in
the code \code{`setInnerText(...css()\-.getText())'}, only one candidate is
remained:
\code{com\-.google\-.gwt\-.resources\-.client\-.CssResource\-.getText()}.
Thus, those relations actually can help identify the API elements,
leading to better decision in {\tool} on the \code{try-catch} block
and exception types.
%
Because it has not seen any \code{try-catch} block involving
\code{com\-.....getText()} and those related ones, {\tool} decides
that the code snippet does not need a \code{try-catch} block. In
contrast, XRank considers only the {\em pairwise} associate scores
between an {\em individual API method call} and the exception types
in a \code{catch} clause. It disregards those above
relations/dependencies among the API names. Thus, it might
misunderstand that \code{getText} needs a \code{try-catch} due to the
co-occurrences of other API elements that need one. That is, without
the dependencies, XRank might make incorrect identification of the API
elements via their names, leading to incorrect exception
recommendation.

%considering Groum but only to get better API ...


%\begin{table}[htpb]
%  \caption{Try-Catch Necessity Checking Comparison (RQ1)}
%  \vspace{-12pt}
%	\begin{center}
%		\renewcommand{\arraystretch}{1}
%		\begin{tabular}{p{1.5cm}<{\centering}|p{1.25cm}<{\centering}p{1.25cm}<{\centering}|p{1.25cm}<{\centering}p{1.25cm}<{\centering}}
%			\hline
%			\multirow{2}{*}{} & \multicolumn{2}{c|}{{\tool} Dataset} & \multicolumn{2}{c}{FuzzyCatch Dataset}\\
%			\cline{2-5}
%			  & \tool  & XRank & \tool  & XRank\\
%			\hline
%			Recall    & \textbf{0.81} & &&\\
%			Precision & \textbf{0.66} & &&\\
%			F-score   & \textbf{0.73} & &&\\
%			\hline
%		\end{tabular}
%		\label{tab:xblock}
%	\end{center}
%\end{table}
