\subsection{Comparison on Try-Catch Block Necessity Checking Effectiveness (RQ1)}
\label{sec:rq1}

\begin{table}[t]
  \caption{Try-Catch Necessity Checking Comparison (RQ1)}
  \vspace{-12pt}
	\begin{center}
		\renewcommand{\arraystretch}{1}
		\begin{tabular}{p{1.5cm}<{\centering}|p{1.25cm}<{\centering}p{1.25cm}<{\centering}|p{1.25cm}<{\centering}p{1.25cm}<{\centering}}
			\hline
			\multirow{2}{*}{Category} & \multicolumn{2}{c|}{{\tool} Dataset} & \multicolumn{2}{c}{FuzzyCatch Dataset}\\
			\cline{2-5}
			  & \tool  & XRank & \tool  & XRank\\
			\hline
			Recall    & \textbf{0.81} & &&\\
			Precision & \textbf{0.66} & &&\\
			F-score   & \textbf{0.73} & &&\\
			\hline
		\end{tabular}
		\label{tab:xblock}
	\end{center}
\end{table}

%{\color{red}{This section waiting for the XRank Results. But from the current estimate, our approach should have higher F-score. But the recall and precision I'm not sure. Once I have the results, I will update this section.}}

Table~\ref{tab:xblock} displays the comparison result. As seen,
{\tool} achieves a high level of performance across two datasets. In
our dataset, with a precision of 66\%, it can decide correctly 2 out
of 3 cases whether a code snippet requires a \code{try-catch} block or
not. With a recall of 81\%, {\tool} is able to cover 4 out of 5 cases
that needs to be placed in a \code{try-catch} block. Users just need
to find 1 out of 5 cases. As a result, it achieves a high F-score of
0.73. In FuzzyCatch dataset, {\tool} achieves a high level of
performance with XX\% precision, YY\% recall, and ZZ\% F-score.

In comparison, in {\tool} dataset, {\tool} improves relatively over
the state-of-the-art approach, XRank, XX\% in precision, YY\% in
recall, and ZZ\% in F-score. In FuzzyCatch, the relative improvements
are XX\%, YY\%, and ZZ\% in precision, recall, and F-score,
respectively.

We examined the cases that {\tool} performed better than XRank.
First, XRank relies on the association scores between the presence of
API method calls and the presence of a \code{try-catch} block. The
decisions on the necessity of a \code{try-catch} block or the
exception types depend on the pre-defined thresholds in XRank on those
association scores. Thus, those pre-defined thresholds might not be
suitable with either all the API methods or all the datasets. Second,
for the incomplete code snippets in which the names of the API methods
in different packages or different libraries are the same (e.g.,
\code{toString} or \code{getText} in various JDK packages), XRank
cannot distinguish them and use one entry in the dictionary for them
due to its IR approach. In contrast, unlike XRank which considers only
the API method calls in a \code{try-catch} block, {\tool} consider the
code in the block as the context to learn the program dependencies
among the names of those API elements. That is, it leverages the
relations among the names of those API elements to learn their
identities, thus, deciding better the need of \code{try-catch} blocks
and the corresponding exception types.

For example, in a code snippet, \code{getText} has XXX\% API method
candidates. However, considering that the relation between \code{css}
and \code{getText} in \code{...()\-.css()\-.getText()}, we only have 4
candidates for \code{getText}. Finally, considering the return value
of \code{getText} is an argument of \code{setInnerText}, only one
candidate is remained:
\code{com\-.google\-.gwt\-.resources\-.client\-.CssResource\-.getText()}.
Because it has not seen any \code{try-catch} block for that, {\tool}
decides that the code snippet does not need a \code{try-catch} block.
XRank considers only the {\em pairwise} associate scores between an
{\em individual API method call} and the presence of a
\code{try-catch} block, it disregards the relations/dependencies among
the API names, which can help identify their identities, leading to
the decision on the \code{try-catch} block and exception types.
