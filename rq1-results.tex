\subsection{Comparison on Try-Catch Block Necessity Checking Effectiveness (RQ1)}
\label{sec:rq1}

\begin{table}[h]
  \caption{Try-Catch Necessity Checking Comparison (RQ1)}
  \vspace{-12pt}
	\begin{center}
		\renewcommand{\arraystretch}{1}
		\begin{tabular}{p{1.5cm}<{\centering}|p{1.25cm}<{\centering}p{1.25cm}<{\centering}|p{1.25cm}<{\centering}p{1.25cm}<{\centering}}
			\hline
			\multirow{2}{*}{Category} & \multicolumn{2}{c|}{{\tool} Dataset} & \multicolumn{2}{c}{FuzzyCatch Dataset}\\
			\cline{2-5}
			  & \tool  & XRank & \tool  & XRank\\
			\hline
			Recall    & \textbf{0.81} & &&\\
			Precision & \textbf{0.66} & &&\\
			F-score   & \textbf{0.73} & &&\\
			\hline
		\end{tabular}
		\label{tab:xblock}
	\end{center}
\end{table}

%{\color{red}{This section waiting for the XRank Results. But from the current estimate, our approach should have higher F-score. But the recall and precision I'm not sure. Once I have the results, I will update this section.}}

Table~\ref{tab:xblock} displays the comparison result. As seen,
{\tool} achieves a high level of performance across two datasets. In
our dataset, with a precision of 66\%, it can decide correctly 2 out
of 3 cases whether a code snippet requires a \code{try-catch} block or
not. With a recall of 81\%, {\tool} is able to cover 4 out of 5 cases
that needs to be placed in a \code{try-catch} block. Users just need
to find 1 out of 5 cases. As a result, it achieves a high F-score of
0.73. In FuzzyCatch dataset, {\tool} achieves a high level of
performance with XX\% precision, YY\% recall, and ZZ\% F-score.

In comparison, in {\tool} dataset, {\tool} improves relatively over
the state-of-the-art approach, XRank, XX\% in precision, YY\% in
recall, and ZZ\% in F-score. In FuzzyCatch, the relative improvements
are XX\%, YY\%, and ZZ\% in precision, recall, and F-score,
respectively.

We examined the cases that {\tool} performed better than XRank.
First, XRank relies on the association scores between the presence of
API method calls and the presence of a \code{try-catch} block. The
decisions on the necessity of a \code{try-catch} block or the
exception types depend on the pre-defined thresholds in XRank on those
association scores. Thus, those pre-defined thresholds might not be
suitable with either all the API methods or all the datasets. Second,
...

