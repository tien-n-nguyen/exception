\section{Motivation}
\label{motiv:sec}

\subsection{Motivating Examples}
\label{examples:sec}

\begin{figure}[htbp]
	\centering
	\lstset{
		numbers=left,
		numberstyle= \tiny,
		keywordstyle= \color{blue!70},
		commentstyle= \color{red!50!green!50!blue!50},
		frame=shadowbox,
		rulesepcolor= \color{red!20!green!20!blue!20} ,
		xleftmargin=1.5em,xrightmargin=0em, aboveskip=1em,
		framexleftmargin=1.5em,
                numbersep= 5pt,
		language=C,
    basicstyle=\scriptsize\ttfamily,
    numberstyle=\scriptsize\ttfamily,
    emphstyle=\bfseries,
                moredelim=**[is][\color{red}]{@}{@},
		escapeinside= {(*@}{@*)}
	}
\begin{lstlisting}[]
public static void addLibraryPath(String pathToAdd) throws Exception {
  final Field usrPathsField = ClassLoader.class.(*@{\color{orange}{getDeclaredField("usr\_paths");}@*)
  usrPathsField.setAccessible(true);

  //get array of paths
  final String[] paths = (String[])usrPathsField.get(null);

  //check if the path to add is already present
  for(String path : paths) {
      if(path.equals(pathToAdd)) {
          return;
      }
  }

  //add the new path
  final String[] newPaths = Arrays.copyOf(paths, paths.length + 1);
  newPaths[newPaths.length-1] = pathToAdd;
  usrPathsField.set(null, newPaths);
}
\end{lstlisting}
        \vspace{-12pt}
        \caption{StackOverflow Post \#15409223 on Adding New Paths for
          Native Libraries at Runtime in Java}
        \label{fig:example1}
\end{figure}


Let us use a few real-world examples to explain the problem and
motivate our approach. Figure~\ref{fig:example1} displays a code
snippet in an answer to the StackOverflow (SO) question 15409223 on
how to ``add new paths for native libraries at runtime in Java''.  The
code snippet serves the illustrating purpose, thus, does not contain
all the details regarding what exceptions that need to be handled. It
just contains a throw of a generic \code{Exception} in the method
header (\code{addLibraryPath}). From the empirical study from Zhang
{\em et al.}~\cite{zhang-icse19}, this code snippet was adopted by the
developers into their Github project named \code{armint} as seen in
Figure~\ref{fig:example2}. In \code{armint} code, the developers
handle in a \code{try-catch} block the all possible exceptions caused
by \code{java\-.lang\-.Class\-.get\-Declared\-Field(...)} (line 7)
according to JDK's documentation including
\code{No\-Such\-Field\-Exception}, \code{Security\-Exception},
\code{Illegal\-Argument\-Exception}, and
\code{Illegal\-Access\-Exception} (line 24 of
Figure~\ref{fig:example2}).

In their study, Zhang {\em et al.}~\cite{zhang-icse19} have reported
that such adaptation is still largely manual due to the lack of
documentation for exception handling. Kechagia {\em et
  al.}~\cite{kechagia-msr14} found that 69\% of the methods from
Android packages in their stack traces had undocumented exceptions in
the Android API. The manual adaptation on exception handling by
inserting a \code{try-catch} block is quite popular (in 31 out 629
cases). Such manual adaptation could lead to exception-related bugs,
which could cause serious issues including crashes or unstable
states~\cite{xrank-fse20}. Thus, it is desirable to have an automated
tool to recommend proper exception handling in order to adapt the
incomplete code snippets. For example, what lines need to be included
in a \code{try-catch} block, and what exceptions need to be handled.

\begin{Observation} [{\bf Exception Handling Recommendation}]
Automated recommendation to properly handle exceptions is desirable in
assisting developers to adapt incomplete code snippets into their codebases.
\end{Observation}

%spec, heuristic, mining, IR

To address the issue, several automated approaches have been proposed
to recommend proper exception
handling~\cite{xrank-fse20,barbosa-bsse12,chanchal-scam14,barbosa-tse18,barbosa-tse16}. Earlier
approaches utilized {\em heuristic strategies} on exception types,
method calls, and variables' types to recommend exception handling
code~\cite{barbosa-bsse12}. While the heuristics might not work in all
cases, other approaches enforce {\em exception handling
  policies}~\cite{barbosa-tse16,barbosa-saner18}. However, the
policies must be pre-defined and encoded in the tool. To overcome
that, the {\em mining approaches} are based on the idea that similar
code requires similar exception handling~\cite{chanchal-scam14}. Some
approaches also leverage mined patterns to repair exception-handling
bugs~\cite{zhong-jss18}. A key issue with the mining approaches is
that the two code fragments might not be exactly matched, but they
share a few key elements that require the same exception handling.
Thus, Nguyen {\em et al.}~\cite{xrank-fse20} propose an {\em
  information retrieval} (IR) approach using a fuzzy set technique to
learn the associations between the method calls (e.g.,
\code{get\-Declared\-Field}) and the exceptions (e.g.,
\code{No\-Such\-Field\-Exception}, \code{Security\-Exception}, etc.)
that need to be handled. A key drawback of the IR approach is that it
relies on the threshold for a match in an retrieval, which might often
not be easily pre-defined.

\begin{figure}[htbp]
	\centering
	\lstset{
		numbers=left,
		numberstyle= \tiny,
		keywordstyle= \color{blue!70},
		commentstyle= \color{red!50!green!50!blue!50},
		frame=shadowbox,
		rulesepcolor= \color{red!20!green!20!blue!20} ,
		xleftmargin=1.5em,xrightmargin=0em, aboveskip=1em,
		framexleftmargin=1.5em,
                numbersep= 5pt,
		language=C,
    basicstyle=\scriptsize\ttfamily,
    numberstyle=\scriptsize\ttfamily,
    emphstyle=\bfseries,
                moredelim=**[is][\color{red}]{@}{@},
		escapeinside= {(*@}{@*)}
	}
\begin{lstlisting}[]
/** ...
 * taken from http://stackoverflow.com/questions/15409223/
 * adding-new-paths-for-native-libraries-at-runtime-in-java
 */
private static void addLibraryPath(String pathToAdd) {
  (*@{\color{orange}{try}@*) {
    final Field usrPathsField = ClassLoader.class.(*@{\color{orange}{getDeclaredField("usr\_paths");}@*)
    usrPathsField.setAccessible(true);

    // get array of paths
    final String[] paths = (String[]) usrPathsField.get(null);

    // check if the path to add is already present
    for (String path : paths) {
      if (path.equals(pathToAdd)) {
	return;
      }
    }

    // add the new path
    final String[] newPaths = Arrays.copyOf(paths, paths.length + 1);
    newPaths[newPaths.length - 1] = pathToAdd;
    usrPathsField.set(null, newPaths);
  } (*@{\color{orange}{catch (NoSuchFieldException | SecurityException | IllegalArgumentException |    IllegalAccessException e)}@*) {
	throw new RuntimeException(e);
  }
}
\end{lstlisting}
        \vspace{-12pt}
        \caption{GitHub project \code{armint} Adapts SO Post in Figure~\ref{fig:example1}}
        \label{fig:example2}
\end{figure}


%\subsection{Observations}
%\label{sec:obs}

%To facilitate the reuse of code snippets in a forum, an automated tool
%is needed to derive the fully-qualified names of the API elements in
%the snippets so that the proper import statements are added in the
%code.
%To build an automated tool to assist developers in proper exception
%handling, we draw the motivation from the following observations.

\begin{figure}[t] %[htbp]
	\centering
	\lstset{
		numbers=left,
		numberstyle= \tiny,
		keywordstyle= \color{blue!70},
		commentstyle= \color{red!50!green!50!blue!50},
		frame=shadowbox,
		rulesepcolor= \color{red!20!green!20!blue!20} ,
		xleftmargin=1.5em,xrightmargin=0em, aboveskip=1em,
		framexleftmargin=1.5em,
                numbersep= 5pt,
		language=C,
    basicstyle=\scriptsize\ttfamily,
    numberstyle=\scriptsize\ttfamily,
    emphstyle=\bfseries,
                moredelim=**[is][\color{red}]{@}{@},
		escapeinside= {(*@}{@*)}
	}
\begin{lstlisting}[]
public Object readField(Class<?> clazz, String name, Object instance) {
  (*@{\color{orange}{try}@*) {
    Field field = clazz.(*@{\color{orange}{getDeclaredField}@*)(name);
    if (!field.isAccessible()) {
       field.setAccessible(true);
    }
    return field.get(instance);
  } (*@{\color{orange}{catch (NoSuchFieldException | SecurityException | IllegalArgumentException | IllegalAccessException e)}@*) {
   throw new RuntimeException("Cannot read field value: " + clazz.getName() + "#" + name, e);
  }
}
\end{lstlisting}
        \vspace{-16pt}
        \caption{Project \code{quarkus} with same exception handling}
        \label{fig:example3}
\end{figure}


Now, consider the complete code example in Figure~\ref{fig:example3}
from the Github project named \code{quarkus}. While there are
differences between the complete code in Figure~\ref{fig:example3} and
the adapted code in Figure~\ref{fig:example2}, the presence of the API
call to \code{get\-Declared\-Field} has made the list of the handled
exceptions the same (line 8 in Figure~\ref{fig:example3} and line 24
in Figure~\ref{fig:example2}). This is expected because the designers
of the JDK library have the intents for developers to use the API
method \code{get\-Declared\-Field} within a \code{try-catch} block and
to handle the list of exceptions as in line 8 of
Figure~\ref{fig:example3}. Thus, to adapt the incomplete code snippet
in Figure~\ref{fig:example1}, one could learn from the public code
repositories to properly handle the exceptions.

\begin{Observation} [{\bf Regularity of Exception Handling}]
Finding the patterns from complete code in existing code corpora could
be a good strategy to {\bf learn to properly handle the exceptions} in
adapting an (incomplete) code snippet into a codebase.
\end{Observation}

\begin{Observation} [{\bf Connections between API methods and Exceptions}]
The presence of certain API elements helps decide the exceptions that
need to be handled.
\end{Observation}

In our motivating example, the connection between
\code{java.\-lang.\-Class.\-get\-Declared\-Field} and the exceptions
\code{No\-Such\-Field\-Exception}, \code{Security\-Exception},
\code{Illegal\-Argument\-Exception}, and
\code{Illegal\-Access\-Exception} can be learned from the existing
code corpora. Thus, a model can learn to recommend those exceptions
for an incomplete code snippet involving \code{get\-Declared\-Field}.

\begin{figure}[htbp]
	\centering
	\lstset{
		numbers=left,
		numberstyle= \tiny,
		keywordstyle= \color{blue!70},
		commentstyle= \color{red!50!green!50!blue!50},
		frame=shadowbox,
		rulesepcolor= \color{red!20!green!20!blue!20} ,
		xleftmargin=1.5em,xrightmargin=0em, aboveskip=1em,
		framexleftmargin=1.5em,
                numbersep= 5pt,
		language=C,
    basicstyle=\scriptsize\ttfamily,
    numberstyle=\scriptsize\ttfamily,
    emphstyle=\bfseries,
                moredelim=**[is][\color{red}]{@}{@},
		escapeinside= {(*@}{@*)}
	}
\begin{lstlisting}[]
Charset charset = Charset.forName("US-ASCII");
try (BufferedReader reader = Files.newBufferedReader(file, charset)) {
    String line = null;
    while ((line = reader.readLine()) != null) {
        System.out.println(line);
    }
} catch (IOException x) {
    System.err.format("IOException: %s%n", x);
}
\end{lstlisting}
        \vspace{-12pt}
        \caption{Using \code{newBufferedReader} to read from a file}
        \label{fig:example4}
\end{figure}


For the list of statements in an incomplete code, not all of them
needs to be wrapped around in a \code{try-catch} block. For example,
considering the example of using \code{new\-Buffered\-Reader} in
Figure~\ref{fig:example4}. While the API call
\code{java.\-nio.\-file.\-new\-Bugffered\-Reader} needs to be within a
\code{try-catch} block, the statement at line 1 to retrieve the
character set does not. Moreover, the statement at line 5 with the API
call to \code{readLine} needs to be wrapped in a \code{try-catch}
block as well.

\begin{Observation} [{\bf Learn to Decide What Statements to be in a Try-Catch block}]
  A model can learn from the code corpora what statements need to be
  within a \code{try-catch} block or not.
\end{Observation}
