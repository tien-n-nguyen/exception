\section{Motivation}
\label{motiv:sec}

\subsection{Motivating Examples}
\label{examples:sec}

\begin{figure}[htbp]
	\centering
	\lstset{
		numbers=left,
		numberstyle= \tiny,
		keywordstyle= \color{blue!70},
		commentstyle= \color{red!50!green!50!blue!50},
		frame=shadowbox,
		rulesepcolor= \color{red!20!green!20!blue!20} ,
		xleftmargin=1.5em,xrightmargin=0em, aboveskip=1em,
		framexleftmargin=1.5em,
                numbersep= 5pt,
		language=C,
    basicstyle=\scriptsize\ttfamily,
    numberstyle=\scriptsize\ttfamily,
    emphstyle=\bfseries,
                moredelim=**[is][\color{red}]{@}{@},
		escapeinside= {(*@}{@*)}
	}
\begin{lstlisting}[]
public static void addLibraryPath(String pathToAdd) throws Exception {
  final Field usrPathsField = ClassLoader.class.(*@{\color{orange}{getDeclaredField("usr\_paths");}@*)
  usrPathsField.setAccessible(true);

  //get array of paths
  final String[] paths = (String[])usrPathsField.get(null);

  //check if the path to add is already present
  for(String path : paths) {
      if(path.equals(pathToAdd)) {
          return;
      }
  }

  //add the new path
  final String[] newPaths = Arrays.copyOf(paths, paths.length + 1);
  newPaths[newPaths.length-1] = pathToAdd;
  usrPathsField.set(null, newPaths);
}
\end{lstlisting}
        \vspace{-12pt}
        \caption{StackOverflow Post \#15409223 on Adding New Paths for
          Native Libraries at Runtime in Java}
        \label{fig:example1}
\end{figure}


Let us use a few real-world examples to explain the problem and
motivate our approach. Figure~\ref{fig:example1} displays a code
snippet in an answer to the StackOverflow question 15409223 on
how to ``add new paths for native libraries at runtime in Java''.  The
code snippet serves the illustrating purpose, thus, does not contain
all the details regarding what exceptions that need to be handled. It
contains only a throw of a generic \code{Exception} in the method
header~(\code{addLibraryPath}). From the empirical study from Zhang
{\em et al.}~\cite{zhang-icse19}, this code snippet was adopted by the
developers into their Github project named \code{armint} (see
Figure~\ref{fig:example2}). In \code{armint} code, the developers
handle in a \code{try-catch} block all possible exceptions caused
by \code{java\-.lang\-.Class\-.get\-Declared\-Field(...)} (line 7)
according to JDK's documentation including
\code{No\-Such\-Field\-Exception}, \code{Security\-Exception},
\code{Illegal\-Argument\-Excep\-tion}, and
\code{Illegal\-Access\-Exception} (line 24 of
Figure~\ref{fig:example2}).

In their study, Zhang {\em et al.}~\cite{zhang-icse19} have reported
that such adaptation is still largely manual.
%Kechagia {\em et al.}~\cite{kechagia-msr14} found that 69\% of the
%methods from Android packages in their stack traces had undocumented
%exceptions in the Android APIs.
The manual adaptation on exception handling by inserting a
\code{try-catch} block is quite popular, yet not automated by any
tools. Such manual adaptation for a code snippet could lead to
exception-related bugs, which could cause serious issues including
crashes or unstable states. It is not trivial for
developers to memorize what API methods could cause exceptions and
what exceptions need to be handled~\cite{xrank-fse20}.
%
Thus, it is desirable to have an automated tool to recommend proper
exception handling in order to adapt the incomplete code snippets.
For example, such a tool could recommend what lines need to be
included in a \code{try-catch} block, and what exceptions need to be
handled.

\begin{Observation} [{\bf Exception Handling Recommendation}]
\label{ob1}
Automated recommendation to handle exceptions is desirable to
assist developers in adapting incomplete code snippets into their
codebases.
\end{Observation}

%spec, heuristic, mining, IR

As explained in Section~\ref{sec:intro}, for such recommendation, four
categories of automated approaches have been proposed to recommend
exception
handling~\cite{xrank-fse20,barbosa-bsse12,chanchal-scam14,barbosa-tse18,barbosa-tse16}. While
early approaches were not effective in all cases due to their {\em
  heuristics}~\cite{barbosa-bsse12}, the {\em exception policies} are
too strict in enforcing them, yet requires the rules to be encoded in
the tools~\cite{barbosa-tse16,barbosa-saner18}. The state-of-the-art
{\em information retrieval}-based approaches (e.g.,
XRank/Xhand~\cite{xrank-fse20}) have been shown to outperform the
existing approaches including the {\em mining
  approaches}~\cite{chanchal-scam14} (which suffers the issue of setting a
threshold for frequent occurrences).

%XRank~\cite{xrank-fse20} and XHand~\cite{xrank-fse20} have shown that
%information retrieval (IR) techniques such as fuzzy sets could perform
%better than the mining approaches~\cite{chanchal-scam14}, which
%suffers the issue of setting a threshold for frequent occurrences.

However, the state-of-the-art, IR-based approaches~\cite{xrank-fse20}
have the following key limitations. First, it is not trivial to
pre-define a threshold for feature matching for a retrieval, e.g., the
threshold to determine the associations between an API~call (e.g.,
\code{get\-Declared\-Field}) and an exception type (e.g.,
\code{No\-Such\-Field\-Exception}). Second, relying on the lexical
values of the elements' names, XRank/XHand~\cite{xrank-fse20} cannot
handle the ambiguous names of the APIs or exceptions in an incomplete
code snippet (e.g., the API method \code{get} at line 6 of
Figure~\ref{fig:example1} occurs in multiple libraries), which might
be unparseable to resolve their fully-qualified names. Thus, this
reduces its effectiveness. Finally, XRank/XHand~\cite{xrank-fse20}
consider only the associations between an API method with an exception
type, and discards the surrounding code context in a \code{try-catch}
block.  For example, it learns the association between the names of
the API call (e.g., \code{get\-Declared\-Field}) and the exceptions
(e.g., \code{No\-Such\-Field\-Exception}, \code{Security\-Exception},
etc.) that need to be handled. Without the context, it is even more
challenging to decide the identities of the APIs via only simple
names.

%This makes their tool further suffer the ambiguity issue.


%Earlier approaches utilized {\em heuristic strategies} on exception
%types, method calls, and variables' types to recommend exception
%handling code~\cite{barbosa-bsse12}. While the heuristics might not
%work in all the cases, other approaches enforce {\em exception
%  handling policies}~\cite{barbosa-tse16,barbosa-saner18}. However,
%the policies must be pre-defined and encoded in the tool. To overcome
%that, the {\em mining approaches} are based on the idea that similar
%code requires similar exception handling~\cite{chanchal-scam14}.
%Some approaches also leverage mined patterns to repair
%exception-handling bugs~\cite{zhong-jss18}.
%A key issue with the mining approaches is that the two code fragments
%might not be exactly matched, but they share a few key elements that
%require the same exception handling.  Thus, Nguyen {\em et
%  al.}~\cite{xrank-fse20} propose an {\em information retrieval} (IR)
%approach using a fuzzy set technique to learn the associations between
%the method calls (e.g., \code{get\-Declared\-Field}) and the
%exceptions (e.g., \code{No\-Such\-Field\-Exception},
%\code{Security\-Exception}, etc.)  that need to be handled. A key
%drawback of the IR approach is that it relies on the threshold for a
%match in an retrieval, which might often not be easily pre-defined.


%To address the issue, several automated approaches have been proposed
%to recommend proper exception
%handling~\cite{xrank-fse20,barbosa-bsse12,chanchal-scam14,barbosa-tse18,barbosa-tse16}. Earlier approaches utilized {\em heuristic strategies} on exception types, method calls, and variables' types to recommend exception handling
%code~\cite{barbosa-bsse12}. While the heuristics might not work in all
%the cases, other approaches enforce {\em exception handling
%  policies}~\cite{barbosa-tse16,barbosa-saner18}. However, the
%policies must be pre-defined and encoded in the tool. To overcome
%that, the {\em mining approaches} are based on the idea that similar
%code requires similar exception handling~\cite{chanchal-scam14}.
%A key issue with the mining approaches is that the two code fragments
%might not be exactly matched, but they share a few key elements that
%require the same exception handling.  Thus, Nguyen {\em et
%  al.}~\cite{xrank-fse20} propose an {\em information retrieval} (IR)
%approach using a fuzzy set technique to learn the associations between
%the method calls (e.g., \code{get\-Declared\-Field}) and the
%exceptions (e.g., \code{No\-Such\-Field\-Exception},
%\code{Security\-Exception}, etc.)  that need to be handled. A key
%drawback of the IR approach is that it relies on the threshold for a
%match in an retrieval, which might often not be easily pre-defined.

\begin{figure}[htbp]
	\centering
	\lstset{
		numbers=left,
		numberstyle= \tiny,
		keywordstyle= \color{blue!70},
		commentstyle= \color{red!50!green!50!blue!50},
		frame=shadowbox,
		rulesepcolor= \color{red!20!green!20!blue!20} ,
		xleftmargin=1.5em,xrightmargin=0em, aboveskip=1em,
		framexleftmargin=1.5em,
                numbersep= 5pt,
		language=C,
    basicstyle=\scriptsize\ttfamily,
    numberstyle=\scriptsize\ttfamily,
    emphstyle=\bfseries,
                moredelim=**[is][\color{red}]{@}{@},
		escapeinside= {(*@}{@*)}
	}
\begin{lstlisting}[]
/** ...
 * taken from http://stackoverflow.com/questions/15409223/
 * adding-new-paths-for-native-libraries-at-runtime-in-java
 */
private static void addLibraryPath(String pathToAdd) {
  (*@{\color{orange}{try}@*) {
    final Field usrPathsField = ClassLoader.class.(*@{\color{orange}{getDeclaredField("usr\_paths");}@*)
    usrPathsField.setAccessible(true);

    // get array of paths
    final String[] paths = (String[]) usrPathsField.get(null);

    // check if the path to add is already present
    for (String path : paths) {
      if (path.equals(pathToAdd)) {
	return;
      }
    }

    // add the new path
    final String[] newPaths = Arrays.copyOf(paths, paths.length + 1);
    newPaths[newPaths.length - 1] = pathToAdd;
    usrPathsField.set(null, newPaths);
  } (*@{\color{orange}{catch (NoSuchFieldException | SecurityException | IllegalArgumentException |    IllegalAccessException e)}@*) {
	throw new RuntimeException(e);
  }
}
\end{lstlisting}
        \vspace{-12pt}
        \caption{GitHub project \code{armint} Adapts SO Post in Figure~\ref{fig:example1}}
        \label{fig:example2}
\end{figure}


%\subsection{Observations}
%\label{sec:obs}

%To facilitate the reuse of code snippets in a forum, an automated tool
%is needed to derive the fully-qualified names of the API elements in
%the snippets so that the proper import statements are added in the
%code.
%To build an automated tool to assist developers in proper exception
%handling, we draw the motivation from the following observations.

\begin{figure}[t] %[htbp]
	\centering
	\lstset{
		numbers=left,
		numberstyle= \tiny,
		keywordstyle= \color{blue!70},
		commentstyle= \color{red!50!green!50!blue!50},
		frame=shadowbox,
		rulesepcolor= \color{red!20!green!20!blue!20} ,
		xleftmargin=1.5em,xrightmargin=0em, aboveskip=1em,
		framexleftmargin=1.5em,
                numbersep= 5pt,
		language=C,
    basicstyle=\scriptsize\ttfamily,
    numberstyle=\scriptsize\ttfamily,
    emphstyle=\bfseries,
                moredelim=**[is][\color{red}]{@}{@},
		escapeinside= {(*@}{@*)}
	}
\begin{lstlisting}[]
public Object readField(Class<?> clazz, String name, Object instance) {
  (*@{\color{orange}{try}@*) {
    Field field = clazz.(*@{\color{orange}{getDeclaredField}@*)(name);
    if (!field.isAccessible()) {
       field.setAccessible(true);
    }
    return field.get(instance);
  } (*@{\color{orange}{catch (NoSuchFieldException | SecurityException | IllegalArgumentException | IllegalAccessException e)}@*) {
   throw new RuntimeException("Cannot read field value: " + clazz.getName() + "#" + name, e);
  }
}
\end{lstlisting}
        \vspace{-16pt}
        \caption{Project \code{quarkus} with same exception handling}
        \label{fig:example3}
\end{figure}


Now, consider the complete code example in Figure~\ref{fig:example3}
from the Github project named \code{quarkus}. While there are
differences between the complete code in Figure~\ref{fig:example3} and
the adapted code in Figure~\ref{fig:example2}, due to the presence of
the API call to \code{get\-Declared\-Field} in both code, the list of
the handled exceptions is the same (line 8 in
Figure~\ref{fig:example3} and line 24 in
Figure~\ref{fig:example2}). This is expected because the designers of
the JDK library have the intent for developers to use the API method
\code{get\-Declared\-Field} within a \code{try-catch} block and to
handle the list of exceptions as in line 8 of
Figure~\ref{fig:example3}. Thus, to adapt the incomplete code snippet
in Figure~\ref{fig:example1}, one could learn from the public code
repositories to properly handle the exceptions.

\begin{Observation} [{\bf Regularity of Exception Handling}]
\label{ob2}
Finding the patterns from complete code in existing code corpora could
be a good strategy to {\bf learn to properly handle the exceptions} in
adapting an (incomplete) code snippet into a codebase.
\end{Observation}

%Tien
%Explain on using context to determine the APIs and exception types
%And make connections between them.

%=======================================================
\begin{Observation} [{\bf Connections between API methods and Exceptions}]
\label{ob3}
The presence of certain API elements helps decide the exceptions that
need to be handled.
\end{Observation}

In our motivating example, the connection between
\code{java.\-lang.\-Class.\-get\-Declared\-Field} and the exceptions
\code{No\-Such\-Field\-Exception}, \code{Secur\-ity\-Exception},
\code{Illegal\-Argument\-Exception}, and
\code{Illegal\-Access\-Exception} can be learned from the existing
code corpora. Thus, a model can learn to recommend those exceptions
for an incomplete code snippet involving \code{get\-Declared\-Field}.
%======================================================

\begin{Observation} [{\bf Dependencies and Surrounding Context Help Resolve Name Ambiguity}]
\label{ob4}
The program dependencies among the API elements and the surrounding
code context can help resolve the ambiguity of the names of those
elements in incomplete code snippets.
\end{Observation}

For an incomplete code snippet, as explained earlier, the simple names
of the API elements (methods, fields, classes) could be ambiguous.
However, if a model can learn from the complete code the
fully-qualified names (FQNs) of the API elements, the context
consisting of those API elements and their program dependencies
%as well as the other program elements in the surrounding context
could help a model determine the correct identities of the API
elements, leading to correct prediction of the handled exceptions.
%
In Figure~\ref{fig:example1}, to determine the identities of
\code{Field} (line 2), \code{get\-Declared\-Field} (line 2),
\code{set\-Accessible} (line 3), \code{get} (line 6), etc., a model
could rely on the dependencies among them in the surrounding context.
For example, the return type of \code{get\-Declared\-Field} is
\code{Field} (thanks to line 2), which has an API method named
\code{set\-Accessible} (thanks to line 3) and another API method named
\code{get} (thanks to line 6). Considering all such dependencies among
the API elements in the context and with the knowledge learned from
the complete code, a model could decide that the identity of
\code{Field} is \code{java.\-lang.\-Class.\-Field}, that of
\code{set\-Accessible} is
\code{java.\-lang.\-Class.\-Field.\-set\-Accessible}, and that of the
\code{get} function at line 6 is
\code{java.\-lang.\-Class.\-Field.\-get}. The rationale is that a
model could see such dependencies among those API elements before in
a complete code in the corpus.

%that of \code{Field} is
%\code{java.\-lang.\-Class.\-Field}, and that of \code{set\-Accessible}
%is \code{java.\-lang.\-Class.\-Field.\-set\-Accessible}.


%======================================================
For the list of statements in an incomplete code, not all of them
needs to be wrapped around in a \code{try-catch} block. For example,
considering the example of using \code{new\-Buffered\-Reader} in
Figure~\ref{fig:example4}. While the API call
\code{java.\-nio.\-file.\-new\-Bugffered\-Reader} needs to be within a
\code{try-catch} block, the statement at line 1 to retrieve the
character set does not. Moreover, the statement at line 5 with the API
call to \code{readLine} needs to be wrapped in a \code{try-catch}
block as well.

\begin{figure}[htbp]
	\centering
	\lstset{
		numbers=left,
		numberstyle= \tiny,
		keywordstyle= \color{blue!70},
		commentstyle= \color{red!50!green!50!blue!50},
		frame=shadowbox,
		rulesepcolor= \color{red!20!green!20!blue!20} ,
		xleftmargin=1.5em,xrightmargin=0em, aboveskip=1em,
		framexleftmargin=1.5em,
                numbersep= 5pt,
		language=C,
    basicstyle=\scriptsize\ttfamily,
    numberstyle=\scriptsize\ttfamily,
    emphstyle=\bfseries,
                moredelim=**[is][\color{red}]{@}{@},
		escapeinside= {(*@}{@*)}
	}
\begin{lstlisting}[]
Charset charset = Charset.forName("US-ASCII");
try (BufferedReader reader = Files.newBufferedReader(file, charset)) {
    String line = null;
    while ((line = reader.readLine()) != null) {
        System.out.println(line);
    }
} catch (IOException x) {
    System.err.format("IOException: %s%n", x);
}
\end{lstlisting}
        \vspace{-12pt}
        \caption{Using \code{newBufferedReader} to read from a file}
        \label{fig:example4}
\end{figure}


\begin{Observation} [{\bf Learn to Decide What Statements to be in a Try-Catch block}]
\label{ob5}
  A model can learn from the code corpora what statements need to be
  within a \code{try-catch} block or not.
\end{Observation}
