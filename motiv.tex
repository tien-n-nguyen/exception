\section{Motivation}
\label{motiv:sec}

\subsection{Motivating Examples}
\label{examples:sec}

\begin{figure}[htbp]
	\centering
	\lstset{
		numbers=left,
		numberstyle= \tiny,
		keywordstyle= \color{blue!70},
		commentstyle= \color{red!50!green!50!blue!50},
		frame=shadowbox,
		rulesepcolor= \color{red!20!green!20!blue!20} ,
		xleftmargin=1.5em,xrightmargin=0em, aboveskip=1em,
		framexleftmargin=1.5em,
                numbersep= 5pt,
		language=C,
    basicstyle=\scriptsize\ttfamily,
    numberstyle=\scriptsize\ttfamily,
    emphstyle=\bfseries,
                moredelim=**[is][\color{red}]{@}{@},
		escapeinside= {(*@}{@*)}
	}
\begin{lstlisting}[]
public static void addLibraryPath(String pathToAdd) throws Exception {
  final Field usrPathsField = ClassLoader.class.(*@{\color{orange}{getDeclaredField("usr\_paths");}@*)
  usrPathsField.setAccessible(true);

  //get array of paths
  final String[] paths = (String[])usrPathsField.get(null);

  //check if the path to add is already present
  for(String path : paths) {
      if(path.equals(pathToAdd)) {
          return;
      }
  }

  //add the new path
  final String[] newPaths = Arrays.copyOf(paths, paths.length + 1);
  newPaths[newPaths.length-1] = pathToAdd;
  usrPathsField.set(null, newPaths);
}
\end{lstlisting}
        \vspace{-12pt}
        \caption{StackOverflow Post \#15409223 on Adding New Paths for
          Native Libraries at Runtime in Java}
        \label{fig:example1}
\end{figure}


Let us use a few real-world examples to explain the problem and
motivate our approach. Figure~\ref{fig:example1} displays a code
snippet in~an~answer to the StackOverflow (S/O) question 15409223 on
how~to~ {\em ``add new paths for native libraries at runtime in
  Java''}.  The code snippet serves as an illustration in the S/O
post, thus, does not contain all the details on what exceptions that
need to be handled. It contains only a throw of a generic
\code{Exception} in the method header~(\code{addLibraryPath}). From
Zhang {\em et al.}'s study~\cite{zhang-icse19}, this code snippet was
adopted by developers into a GitHub project named \code{armint}
(Figure~\ref{fig:example2}).~\code{armint}'s developers handle in a
\code{try-catch} block several exceptions caused~by
\code{java\-.lang\-.Class\-.get\-Declared\-Field(...)} (line 7)
according to JDK's documentation, e.g.,
\code{No\-Such\-Field\-Exception}, \code{Security\-Exception},
\code{Illegal\-Argu\-ment\-Excep\-tion}, and
\code{Illegal\-Access\-Exception} (line 24,
Figure~\ref{fig:example2}).



%In their study, Zhang {\em et al.}~\cite{zhang-icse19} have reported
%that
%Such adaptation is still largely manual~\cite{zhang-icse19}.
%Kechagia {\em et al.}~\cite{kechagia-msr14} found that 69\% of the
%methods from Android packages in their stack traces had undocumented
%exceptions in the Android APIs.
The manual adaptation on exception handling by inserting a
\code{try-catch} block is quite popular, yet not automated by any
tools~\cite{zhang-icse19}. Such manual adaptation for a code snippet
could lead to exception-related bugs, which could cause serious issues
including crashes or unstable states~\cite{xrank-fse20}.
%It is not trivial for developers to memorize what API methods could
%cause exceptions and what exceptions need to be
%handled~\cite{xrank-fse20}.
Thus, it is desirable to have an automated tool to recommend proper
exception handling for such adaptation.
%in order to adapt the incomplete code snippets.
%Such a tool could recommend if a \code{try-catch} block is needed for
%the snippet, what lines need to be included in that block, and what
%exception types need to be handled.


\begin{Observation} [{\bf Exception Handling Recommendation}]
\label{ob1}
Automated recommendation to handle exceptions is desirable to
assist developers in adapting incomplete code snippets into their
codebases.
\end{Observation}

%spec, heuristic, mining, IR

\begin{figure}[htbp]
	\centering
	\lstset{
		numbers=left,
		numberstyle= \tiny,
		keywordstyle= \color{blue!70},
		commentstyle= \color{red!50!green!50!blue!50},
		frame=shadowbox,
		rulesepcolor= \color{red!20!green!20!blue!20} ,
		xleftmargin=1.5em,xrightmargin=0em, aboveskip=1em,
		framexleftmargin=1.5em,
                numbersep= 5pt,
		language=C,
    basicstyle=\scriptsize\ttfamily,
    numberstyle=\scriptsize\ttfamily,
    emphstyle=\bfseries,
                moredelim=**[is][\color{red}]{@}{@},
		escapeinside= {(*@}{@*)}
	}
\begin{lstlisting}[]
/** ...
 * taken from http://stackoverflow.com/questions/15409223/
 * adding-new-paths-for-native-libraries-at-runtime-in-java
 */
private static void addLibraryPath(String pathToAdd) {
  (*@{\color{orange}{try}@*) {
    final Field usrPathsField = ClassLoader.class.(*@{\color{orange}{getDeclaredField("usr\_paths");}@*)
    usrPathsField.setAccessible(true);

    // get array of paths
    final String[] paths = (String[]) usrPathsField.get(null);

    // check if the path to add is already present
    for (String path : paths) {
      if (path.equals(pathToAdd)) {
	return;
      }
    }

    // add the new path
    final String[] newPaths = Arrays.copyOf(paths, paths.length + 1);
    newPaths[newPaths.length - 1] = pathToAdd;
    usrPathsField.set(null, newPaths);
  } (*@{\color{orange}{catch (NoSuchFieldException | SecurityException | IllegalArgumentException |    IllegalAccessException e)}@*) {
	throw new RuntimeException(e);
  }
}
\end{lstlisting}
        \vspace{-12pt}
        \caption{GitHub project \code{armint} Adapts SO Post in Figure~\ref{fig:example1}}
        \label{fig:example2}
\end{figure}



\begin{figure}[t] %[htbp]
	\centering
	\lstset{
		numbers=left,
		numberstyle= \tiny,
		keywordstyle= \color{blue!70},
		commentstyle= \color{red!50!green!50!blue!50},
		frame=shadowbox,
		rulesepcolor= \color{red!20!green!20!blue!20} ,
		xleftmargin=1.5em,xrightmargin=0em, aboveskip=1em,
		framexleftmargin=1.5em,
                numbersep= 5pt,
		language=C,
    basicstyle=\scriptsize\ttfamily,
    numberstyle=\scriptsize\ttfamily,
    emphstyle=\bfseries,
                moredelim=**[is][\color{red}]{@}{@},
		escapeinside= {(*@}{@*)}
	}
\begin{lstlisting}[]
public Object readField(Class<?> clazz, String name, Object instance) {
  (*@{\color{orange}{try}@*) {
    Field field = clazz.(*@{\color{orange}{getDeclaredField}@*)(name);
    if (!field.isAccessible()) {
       field.setAccessible(true);
    }
    return field.get(instance);
  } (*@{\color{orange}{catch (NoSuchFieldException | SecurityException | IllegalArgumentException | IllegalAccessException e)}@*) {
   throw new RuntimeException("Cannot read field value: " + clazz.getName() + "#" + name, e);
  }
}
\end{lstlisting}
        \vspace{-16pt}
        \caption{Project \code{quarkus} with same exception handling}
        \label{fig:example3}
\end{figure}


As explained in Section~\ref{sec:intro},  four
categories of automated approaches have been proposed to recommend
exception
handling~\cite{xrank-fse20,barbosa-bsse12,chanchal-scam14,barbosa-tse18,barbosa-tse16}.
%--------------- Tien ---
%While early approaches were not effective in all cases due to their
%{\em heuristics}~\cite{barbosa-bsse12}, the {\em exception policies}
%are too strict in enforcing them, yet requires the rules to be encoded
%in the tools~\cite{barbosa-tse16,barbosa-saner18}. The
%state-of-the-art {\em information retrieval}-based approaches (e.g.,
%XRank/Xhand~\cite{xrank-fse20}) have been shown to outperform the
%existing approaches including the {\em mining
%  approaches}~\cite{chanchal-scam14} (which suffers the issue of
%setting a threshold for frequent occurrences).
%-----------------
However, the state-of-the-art, IR-based approaches~\cite{xrank-fse20},
which have been shown to outperform others, still have 
limitations. First, it is not trivial to pre-define a threshold for
feature matching for a retrieval, e.g., the threshold to determine the
association scores between an API~call (e.g., \code{get\-Declared\-Field})
and an exception type (e.g., \code{No\-Such\-Field\-Exception}). Thus,
the pre-defined threshold affects much the effectiveness.
%
Second, relying on the lexical values of API elements' names, they
suffer the issue of ambiguous names of the APIs or exceptions (e.g.,
the API method \code{get} at line 6 of Figure~\ref{fig:example1}
occurs in multiple libraries) in an incomplete code snippet, which
might not be parseable for fully-qualified name resolution. Thus, this
also reduces effectiveness. Finally, they consider only the {\em
  association pair between an API method and an exception type}, and
{\em discard the surrounding context}.
%in a \code{try-catch} block.
For example, they examine the associations between the names of the API
call (e.g., \code{get\-Declared\-Field}) and the exceptions to be
handled (e.g., \code{No\-Such\-Field\-Exception},
\code{Security\-Exception}, etc.). Without the context, it is
challenging to decide the identities of the APIs and their exceptions via
only simple names.

%and the exceptions to be handled.










Let us consider the complete code example in Figure~\ref{fig:example3}
from the GitHub project named \code{quarkus}. While there are
differences between the complete code in Figure~\ref{fig:example3} and
the adapted code in Figure~\ref{fig:example2}, the lists of the
handled exceptions are the same (line 8 in Figure~\ref{fig:example3}
and line 24 in Figure~\ref{fig:example2}) due to the presence of the
API call to \code{get\-Declared\-Field} in both code. This is expected
because the designers of the JDK library have the intent for
developers to use the API method \code{get\-Declared\-Field} within a
\code{try} block and to handle the list of exceptions as in line 8 of
Figure~\ref{fig:example3}. Thus, to adapt the incomplete code snippet
in Figure~\ref{fig:example1}, a model could learn from the
complete~code in public repositories to suggest proper exception
handling.


\begin{figure}[htbp]
	\centering
	\lstset{
		numbers=left,
		numberstyle= \tiny,
		keywordstyle= \color{blue!70},
		commentstyle= \color{red!50!green!50!blue!50},
		frame=shadowbox,
		rulesepcolor= \color{red!20!green!20!blue!20} ,
		xleftmargin=1.5em,xrightmargin=0em, aboveskip=1em,
		framexleftmargin=1.5em,
                numbersep= 5pt,
		language=C,
    basicstyle=\scriptsize\ttfamily,
    numberstyle=\scriptsize\ttfamily,
    emphstyle=\bfseries,
                moredelim=**[is][\color{red}]{@}{@},
		escapeinside= {(*@}{@*)}
	}
\begin{lstlisting}[]
Charset charset = Charset.forName("US-ASCII");
try (BufferedReader reader = Files.newBufferedReader(file, charset)) {
    String line = null;
    while ((line = reader.readLine()) != null) {
        System.out.println(line);
    }
} catch (IOException x) {
    System.err.format("IOException: %s%n", x);
}
\end{lstlisting}
        \vspace{-12pt}
        \caption{Using \code{newBufferedReader} to read from a file}
        \label{fig:example4}
\end{figure}



\begin{Observation} [{\bf Regularity of Exception Handling}]
\label{ob2}
Finding the patterns from complete code in existing code corpora could
be a good strategy for a model to learn to properly handle the
  exceptions in adapting an (incomplete) code snippet into a
codebase.
\end{Observation}



In addition, the relations between the API
\code{java.\-lang.\-Class.\-get\-Declared\-Field} and the exceptions
\code{No\-Such\-Field\-Exception}, \code{Secur\-ity\-Excep\-tion},
\code{Illegal\-Argument\-Excep\-tion}, and
\code{Illegal\-Access\-Exception} exhibit in the code corpora. Thus, a
model can learn to recommend those exceptions for any complete or
incomplete code snippet involving the API \code{get\-Declared\-Field}.



\begin{Observation} [{\bf Relations between API Elements
      and Exceptions}]
\label{ob3}
The presence of the relations between API elements and exceptions
helps a model learn to suggest exception handling.
\end{Observation}
%======================================================

%\begin{Observation} [{\bf Dependencies and Surrounding Context help resolve name ambiguity}]
%\label{ob4}
%The program dependencies among the API elements and the surrounding
%code context can help resolve the ambiguity of the names of those
%elements in incomplete code snippets.
%\end{Observation}




%For an incomplete code snippet, as explained earlier, the simple names
%of the API elements (methods, fields, classes) could be ambiguous.
%However, if a model can learn from the complete code the
%fully-qualified names of the API elements, the surrounding context
%consisting of those API elements and their program dependencies can
%help a model decide the correct identities of the API elements.

%Overview figure


For an incomplete code snippet, the simple names of the API elements
(methods, fields, classes) can be ambiguous. However, from the
training data, if a model could learn the {\em dependencies among the
  API elements} and the {\em relations between the APIs and the
  exceptions} in the code context, it can match the context of the
given code snippet to the learned relations to suggest exception
handling.

%
In Figure~\ref{fig:example3}, the dependencies among \code{Field}
(line 3), \code{get\-Declared\-Field} (line 3), \code{setAccessible}
(line 5), \code{get} (line 7), etc. are as follows. The return type of
the API call to \code{get\-Declared\-Field} is \code{Field} (thanks to
line 3), which has an API method named \code{set\-Accessible} (thanks
to line 5) and another API method named \code{get} (thanks to line
7). If a model can be trained to learn from such dependencies among
the API elements during training, it can match the given incomplete code in
Figure~\ref{fig:example1} with the similar dependencies among the API
elements \code{Field} (line 2), \code{get\-Decl\-ared\-Field} (line
2), \code{set\-Accessible} (line 3), and \code{get} (line 6). Thus,
the model can learn to suggest the exception handling similarly to the
complete code in training (Figure~\ref{fig:example3}).

%In Figure~\ref{fig:example1}, to derive the identities of
%\code{Field} (line 2), \code{get\-Decl\-ared\-Field} (line 2),
%\code{set\-Accessible} (line 3), \code{get} (line 6), etc., a model
%could rely on the dependencies among them in the surrounding context.
%For example, the return type of \code{get\-Declared\-Field} is
%\code{Field} (thanks to line 2), which has an API method named
%\code{set\-Accessible} (thanks to line 3) and another API method named
%\code{get} (thanks to line 6). Considering all those dependencies among
%the API elements in the context and with the knowledge learned from
%the complete code, a model could decide that in the code snippet, the identity of
%\code{Field} is \code{java.\-lang.\-Class.\-Field}, that of
%\code{set\-Accessible} is
%\code{java.\-lang.\-Class.\-Field.\-set\-Accessible}, and that of
%\code{get} at line 6 is
%\code{java.\-lang.\-Class.\-Field.\-get}. The rationale is that a
%model could see such dependencies among those API elements before in
%a complete code in training.


\begin{Observation} [{\bf Surrounding Code Context with Dependencies among API Elements}]
\label{ob4}
%The program dependencies among the API elements and
The code context with dependencies among the API elements can help resolve
the ambiguity
%of the names of those elements
in incomplete code snippets, leading to better prediction of the
handled exceptions.
\end{Observation}


%======================================================
For the list of statements in an incomplete code, not all of them
needs to be wrapped around in a \code{try} block. For example,
let us consider the example of using \code{new\-Buffered\-Reader} in
Figure~\ref{fig:example4}. While the API call
\code{java.\-nio.\-file.\-new\-Buffered\-Reader} needs to be within a
\code{try} block, the statement at line 1 to retrieve the
character set does not. Moreover, the statement at line 5 with the API
call to \code{readLine} needs to be wrapped in a \code{try}
block as well.






\begin{Observation} [{\bf Learn to decide what statements to be placed within a \code{try} block}]
\label{ob5}
  A model can learn from the code corpora what statements need to be placed
  within a \code{try} block or not.
\end{Observation}
