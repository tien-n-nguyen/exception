\subsection{RQ2. Effectiveness on Try-Catch Statement Detection}
\label{sec:rq2}

\begin{table}[t]
	\caption{RQ2. Effectiveness on Try-Catch Statement Detection}
	\begin{center}
		\small
		\renewcommand{\arraystretch}{1} 
		\begin{tabular}{p{0.8cm}<{\centering}|p{0.4cm}<{\centering}|p{0.4cm}<{\centering}|p{0.4cm}<{\centering}|p{0.4cm}<{\centering}|p{0.4cm}<{\centering}|p{0.4cm}<{\centering}|p{0.4cm}<{\centering}|p{0.4cm}<{\centering}|p{0.4cm}<{\centering}|p{0.4cm}<{\centering}}
			\hline
			 	&  \multicolumn{10}{c}{Accuracy} \\
			\cline{2-11}
			     	&  N1  & N2   &  N3  & N4   &N5    & N6   &N7    & N8   &N9    & N10 \\
			\hline
			\tool       &  &  &  &  &  & 0.76 &  &  &  &   \\
			\hline
		\end{tabular}
		Nx: x is the number of nodes in the sub-graph (try-catch block)
		\label{RQ2_results}
	\end{center}
\end{table}

{\color{red}{N1-N10 are the number of nodes that the subgraph contains which means the size of the try-catch block. Because the average size of the try-catch is 5.7, I currently pick 6 as the size of the try-catch block. The accuracy here is defined as: if a statement is in the try-catch block and our model put it in the subgraph, I regard it is correct $S_c$. All other conditions, I think they are incorrect. If there are $S$ statements in the try-catch block, the total accuracy is calculated as $S_c/S$. Later, I will add an example showing that the statement that our model predicted in the try-catch block contains the method call which lead to the correct exception types prediction.}}