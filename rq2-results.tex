\subsection{Try-Catch Statement Detection Effectiveness (RQ2)}
\label{sec:rq2}

\begin{table}[t]
  \caption{Try-Catch Statement Detection Effectiveness (RQ2)}
  \vspace{-12pt}
	\begin{center}
		\small
		\renewcommand{\arraystretch}{1} 
		\begin{tabular}{p{0.8cm}<{\centering}|p{0.4cm}<{\centering}|p{0.4cm}<{\centering}|p{0.4cm}<{\centering}|p{0.4cm}<{\centering}|p{0.4cm}<{\centering}|p{0.4cm}<{\centering}|p{0.4cm}<{\centering}|p{0.4cm}<{\centering}|p{0.4cm}<{\centering}|p{0.4cm}<{\centering}}
			\hline
			 	&  \multicolumn{10}{c}{Accuracy} \\
			\cline{2-11}
			     	&  N1  & N2   &  N3  & N4   &N5    & N6   &N7    & N8   &N9    & N10 \\
			\hline
			\tool       &  &  &  &  &  & 0.76 &  &  &  &   \\
			\hline
		\end{tabular}
		Nx is number of nodes in the explanation
                sub-graph (\code{try-catch} block)
		\label{tab:rq2}
	\end{center}
\end{table}

Table~\ref{tab:rq2} displays the result on detecting the statements
that need to be placed in a \code{try-catch} block. $N_x$ is a parameter
in GNNExplainer that defines the number of nodes in the explanation
graph $\mathcal{G}_C$, i.e., the number of statements to be placed in the
\code{try-catch} block.
%
As the number of nodes (statements) in $\mathcal{G}_C$ increases, the
number of correct statements covered also increases, thus, accuracy
increases. However, as the number of statements increases higher than
5, accuracy increases more slowly. In our dataset, the average size of
a \code{try-catch} block is 5.7 statements. As seen, the accuracy as
$N$=6 is 76\%. That is, by predicting 6 statements on average, {\tool}
can correctly point out 76\% of the total number of statements in the
entire dataset that need to be placed in a \code{try-catch} block.
%For the statements that do not need to be placed in a \code{try-catch}
%block, {\tool} correctly makes prediction with accuracy of XX\% (not
%shown).
Note that {\tool} via {\xstate} predicts the statements to be placed
in the \code{try-catch} block only after {\xblock} predicted that the
given code needs such a block. Therefore, the incorrect cases from
{\xblock} (i.e., those cases that need to be in a \code{try-catch} block
but were predicted not) are also counted as incorrect in {\xstate}.

As an example, ...

%{\color{red}{N1-N10 are the number of nodes that the subgraph contains which means the size of the try-catch block. Because the average size of the try-catch is 5.7, I currently pick 6 as the size of the try-catch block. The accuracy here is defined as: if a statement is in the try-catch block and our model put it in the subgraph, I regard it is correct $S_c$. All other conditions, I think they are incorrect. If there are $S$ statements in the try-catch block, the total accuracy is calculated as $S_c/S$. Later, I will add an example showing that the statement that our model predicted in the try-catch block contains the method call which lead to the correct exception types prediction.}}
