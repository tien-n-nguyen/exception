\subsection{Impact of Fine-Tuning (RQ6)}
\label{sec:rq6}

Table~\ref{tab:codebert} shows the comparison result between
our model with fine-tuning and CodeBERT without it.
%
As seen, fine-tuning contributes much to {\tool} in much improving
Precision (almost twice) and slightly improving in Recall (1\%), and
much improving in F1-score (relatively 49.5\%). Without fine-tuning,
the model
%In comparison, the CodeBERT baseline model has a much lower Precision,
%around 50\%. However, it achieves a slightly higher Recall. After
%examining the result, we find that the model
overwhelmingly predicts that the input code snippet contains a
\code{try-catch} block: in our balanced test dataset with
30,764 samples, CodeBERT assigned the negative label (i.e., no
\code{try-catch} block) to only 236 samples.

%only 236 samples receives the negative label (i.e., no
%\code{try-catch}) from CodeBERT.

\begin{figure}[t]%[htbp]
	\centering
	\lstset{
		numbers=left,
		numberstyle= \tiny,
		keywordstyle= \color{blue!70},
		commentstyle= \color{red!50!green!50!blue!50},
		frame=shadowbox,
		rulesepcolor= \color{red!20!green!20!blue!20} ,
		xleftmargin=1.5em,xrightmargin=0em, aboveskip=1em,
		framexleftmargin=1.5em,
                numbersep= 5pt,
		language=C,
    basicstyle=\scriptsize\ttfamily,
    numberstyle=\scriptsize\ttfamily,
    emphstyle=\bfseries,
                moredelim=**[is][\color{red}]{@}{@},
		escapeinside= {(*@}{@*)}
	}
\begin{lstlisting}[]
public void onCreate(Bundle state) {
  requestWindowFeature(Window.FEATURE_NO_TITLE);
  (*@{\color{orange}{+ try \{}@*)
    final WindowManager.LayoutParams attrs = getWindow().getAttributes();
    final Class<?> cls = attrs.getClass();
    final Field fld = cls.getField("buttonBrightness");
    if (fld != null && "float".equals(fld.getType().toString())) {
    fld.setFloat(attrs, 0);
  (*@{\color{orange}{+ \}}@*)
  (*@{\color{orange}{+ \} catch (NoSuchFieldException e) \{}@*)
  (*@{\color{orange}{+ \} catch (IllegalAccessException e) \{}@*)
  (*@{\color{orange}{+ \}}@*)...
}
\end{lstlisting}
        \vspace{-16pt}
        \caption{Exception-related Bug \#106 in FuzzyCatch dataset (missing \code{try-catch}) (detected by {\tool})}
        \label{fig:example-bug}
\end{figure}


\begin{table}[htpb]
  \caption{Impact of Fine-Tuning in {\tool} (RQ6)}
  \vspace{-12pt}
  \small
	\begin{center}
		\renewcommand{\arraystretch}{1}
		\begin{tabular}{| p{3.15cm}<{\centering} | p{1.2cm}<{\centering} | p{1.2cm}<{\centering}| p{1.2cm}<{\centering}|}
		  \hline
			GitHub dataset (XBlock)  & Precision  &  Recall & F1-score \\
			\hline
			CodeBERT w/o fine-tuning & 0.497  & \textbf{0.972}   & 0.657\\
%			\hline
%			XRank & 0.810 & 0.530 & 0.630\\
			\hline
			\tool   &  \textbf{0.981} &  {\bf 0.984} & \textbf{0.982}\\
			\hline
		\end{tabular}
		\label{tab:codebert}
	\end{center}
\end{table}
