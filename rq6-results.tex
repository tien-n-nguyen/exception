\subsection{Impact of Fine-Tuning (RQ6)}
\label{sec:rq6}

\begin{table}[h]%[htpb]
  \caption{Impact of Fine-Tuning in {\tool} (RQ6)}
  \vspace{-12pt}
  \small
	\begin{center}
		\renewcommand{\arraystretch}{1}
		\begin{tabular}{| p{3.15cm}<{\centering} | p{1.2cm}<{\centering} | p{1.2cm}<{\centering}| p{1.2cm}<{\centering}|}
		  \hline
			Github dataset (XBlock)  & Precision  &  Recall & F1-score \\
			\hline
			CodeBERT w/o fine-tuning & 0.497  & \textbf{0.972}   & 0.657\\
%			\hline
%			XRank & 0.810 & 0.530 & 0.630\\
			\hline
			\tool   &  \textbf{0.981} &  {\bf 0.984} & \textbf{0.982}\\
			\hline
		\end{tabular}
		\label{tab:codebert}
	\end{center}
\end{table}

As seen in Table~\ref{tab:codebert}, fine-tuning contributes much to
{\tool} in much improving Precision (almost twice) and slightly
improving in Recall (1\%), and much improving in F1-score (relatively
49.5\%). Without fine-tuning, the model 
%In comparison, the CodeBERT baseline model has a much lower Precision,
%around 50\%. However, it achieves a slightly higher Recall. After
%examining the result, we find that the model
overwhelmingly predicts that the input code snippet contains a
\code{try-catch} block: in our balanced test dataset that contains
30,764 samples, only 236 samples receives the negative label (i.e., no
\code{try-catch}) from CodeBert.
