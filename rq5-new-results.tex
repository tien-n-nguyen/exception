\subsection{Exception-Related Bug Detection (RQ5)}
\label{sec:rq1}

\begin{table}[htpb]
  \caption {Exception-Related Bug Detection (RQ5)}
  \vspace{-12pt}
  \small
	\begin{center}
		\renewcommand{\arraystretch}{1}
		\begin{tabular}{|p{1.75cm}<{\centering}|p{1.75cm}<{\centering}|p{1.75cm}<{\centering}|}
		  \hline
			\multirow{2}{*}{} & \multicolumn{2}{c|}{FuzzyCatch Dataset} \\
			\cline{2-3}
			  & \tool  & FuzzyCatch~\cite{xrank-fse20} \\
			\hline
			Recall    & \textbf{0.95}& 0.76\\
			Precision & \textbf{1.0} & 0.54\\
			F1-score   & \textbf{0.97} & 0.62\\
			\hline
		\end{tabular}
		\label{tab:bug}
	\end{center}
\end{table}

%{\color{red}{This section waiting for the XRank Results. But from the current estimate, our approach should have higher F-score. But the recall and precision I'm not sure. Once I have the results, I will update this section.}}

Table~\ref{tab:bug} displays the comparison result on
exception-related bug detection. As seen, {\tool} can be used to
detect well real-world exception-related bugs in which a code snippet
needs but did not have a \code{try-catch} block or missed some
exceptions. In comparison, {\tool} improves relatively over
FuzzyCatch~\cite{xrank-fse20} 85.2\% in Precision, 25\% in Recall, and
56.5\% in F1-score.
%The reason for a higher recall from FuzzyCatch is the same as in
%RQ1. That is,
%While the recall values between two models are almost the same,
%FuzzyCatch has lower precision.
%It tends to predict ``Yes'' (buggy) for all code snippets. That is
%
If the association score between a single API method call in the code
snippet and one exception type surpasses the pre-defined threshold, Fuzzy-Catch
will determine that the snippet contains a bug.
%Because if there is an association score between {\em only} one API
%method call in the code snippet and one exception type higher than the
%threshold, FuzzyCatch will decide that the snippet is buggy.
Its result is more on the ``Yes'' (buggy) side. Thus, its precision is
54\%, a slightly better than the probability of a coin
toss.

Figure~\ref{fig:example-bug} shows a bug detected by {\tool} and the
correctly suggested fix. The code snippet uses the API method
\code{getField} of the JDK library at line 6. {\tool} is able to
detect the missing of a \code{try-catch} block and to recommend the
fix by adding such a block around the statements from line 4 to line
8, and adding the \code{catch} clauses with two exception types
\code{NoSuchFieldException} and
\code{IllegalAccessException}. {\xstate} was also able to exclude the
lines 1--2 from the \code{try} block. Note that {\tool} performed
correctly even without the knowledge of the JDK documentation.  This
highlights the potential of {\tool} to serve as a valuable complement
to API documentation, aiding developers in ensuring accurate API
usage.

%All buggy code and fixes are available in FuzzyCatch's repository:
%ebrand.ly/ExDataset.



%\begin{figure}[t]%[htbp]
	\centering
	\lstset{
		numbers=left,
		numberstyle= \tiny,
		keywordstyle= \color{blue!70},
		commentstyle= \color{red!50!green!50!blue!50},
		frame=shadowbox,
		rulesepcolor= \color{red!20!green!20!blue!20} ,
		xleftmargin=1.5em,xrightmargin=0em, aboveskip=1em,
		framexleftmargin=1.5em,
                numbersep= 5pt,
		language=C,
    basicstyle=\scriptsize\ttfamily,
    numberstyle=\scriptsize\ttfamily,
    emphstyle=\bfseries,
                moredelim=**[is][\color{red}]{@}{@},
		escapeinside= {(*@}{@*)}
	}
\begin{lstlisting}[]
public void onCreate(Bundle state) {
  requestWindowFeature(Window.FEATURE_NO_TITLE);
  + try {
    final WindowManager.LayoutParams attrs = getWindow().getAttributes();
    final Class<?> cls = attrs.getClass();
    final Field fld = cls.getField("buttonBrightness");
    if (fld != null && "float".equals(fld.getType().toString())) {
    fld.setFloat(attrs, 0);
  + }
  + } catch (NoSuchFieldException e) {
  + } catch (IllegalAccessException e) {
  + }...
}
\end{lstlisting}
        \vspace{-16pt}
        \caption{Exception-Related Bug \#106 in FuzzyCatch dataset (missing \code{try-catch}) (detected by {\tool})}
        \label{fig:example-bug}
\end{figure}



