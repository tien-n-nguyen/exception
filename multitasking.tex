\section{Multi-task Learning}
\label{sec:multitasking}

Learning on the three tasks, {\xblock}, {\xstate}, and {\xtype}
can benefit to one another. If a model decides the need
of a \code{try-catch} block, there must be some statements in the code
snippet that will be placed in such a block. If a model learns the
statements to be placed in a \code{try-catch} block, it can decide
that the code snippet needs such a block and make the connections to
what exception types to be caught. The knowledge on the exception
types can help a model decide better what important statements need to
be in a \code{try-catch} block. Thus, we put the three tasks
in a multi-task learning fashion.

%The two tasks dealt with by {\xstate} and {\xtype} can benefit each other. With the learning about which statements need to be put into try blocks, the model will learn the important code tokens in the code snippet, which in turn will help to determine which exceptions need to be handled. Also, knowing the exceptions will give strong hint to the way that statements should be grouped into try blocks.

We calculate the training loss by combining the losses from the three
tasks. $Loss_{{\xblock}}$ is the Binary Cross Entropy loss for the
decision as to if \code{try-catch} block(s) is needed.
%
To calculate $Loss_{{\xstate}}$, we add the classification losses for
all statements in the input, where a statement loss ($loss_{Stmt}$) is
the Cross Entropy loss calculated from the distribution of the three
tags (\code{O}, \code{B-Try}, \code{I-Try}) and the ground-truth tag.
%
%To calculate $Loss_{{\xstate}}$, we first get the classification loss
%for each statement ($loss_{Stmt}$) in the input, and add them
%together. $loss_{Stmt}$ is the Cross Entropy loss calculated from the
%distribution for the three tags -- \code{O}, \code{B-Try},
%\code{I-Try} -- and the ground-truth tag.
%
Finally, in \xtype, since several \code{try-catch} blocks might be
present, $Loss_{{\xtype}}$ comes from the summation of the exception
prediction losses from all the \code{try-catch} blocks. For each
\code{try-catch} block, the $loss_{try-block}$ is calculated by adding
the Binary Cross Entropy loss for the prediction of each exception of
interest.

The overall training loss is calculated as follows. If the input does
not contain any \code{try-catch} block, the overall loss will be the 
$Loss_{{\xblock}}$. If the input contains a \code{try-catch} block,
the overall loss will be the summation of losses from all
three tasks:
\begin{equation}
\label{eqn:loss}
Loss_{overall} =
\begin{cases}
Loss_{{\xblock}},  & \text{no try-catch}\\
Loss_{{\xblock}} + Loss_{{\xstate}} + Loss_{{\xtype}}, &\text{otherwise.}
\end{cases}
\end{equation}
