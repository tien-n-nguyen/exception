\vspace{-6pt}
\subsection{Empirical Methodology}

\subsubsection{Datasets}

%jodatime, jdk, Android, xstream, gwt, hibernate

We conducted experiments on two datasets: 1) {\em GitHub dataset}
for intrinsic evaluation on exception handling recommendation tasks
({\xblock}, {\xstate}, {\xtype}), and 2) {\em FuzzyCatch}
dataset~\cite{xrank-fse20} for extrinsic evaluation on
exception-related bug detection.
%
%FuzzyCatch dataset was provided by the authors of
%XRank/XHand~\cite{xrank-fse20}. We used FuzzyCatch dataset for our
%extrinsic evaluation.
%
We collected the GitHub dataset as follows. We first chose in GitHub
5,726 Java projects with the highest ratings that use
JDK and Android libraries.
%: jodatime, JDK, Android, xtream, GWT, and Hibernate.
These are the well-established libraries that have been used in
several prior research on the topics related to
APIs~\cite{icse18,liveapi14}.
%
We then selected the methods with at least one \code{try-catch} block
as positive samples, and we also randomly selected from the same
GitHub projects the same amount of code snippets that do not have any
\code{try-catch} block as the negative samples. In total, we have 246,118
code snippets for training, 30,764 for validation, and 30,764 for
testing. In 30,764 testing samples, there are an equal number of
positive and negative ones (15,382).

%There are the equal number of positive/negative samples in 30,764
%testing ones.

%We then selected the methods with at least one \code{try-catch} block,
%which was not part of any fix in a later version. In total, we have
%246,118 code snippets for training, and 30,764 for validation.  We
%also randomly selected from the same GitHub projects the same amount
%of code snippets that do not have any \code{try-catch} block as the
%negative samples. In total, we have 30,764 instances for testing.


For extrinsic evaluation, we used FuzzyCatch dataset, provided by the
authors of XRank/XHand~\cite{xrank-fse20}, which contains 609 Android
incomplete code snippets with exception-related bugs (missing
\code{try- catch} blocks or exceptions). Finally, we used 553 snippets
because the others are not valid for the experiment.
%Tien
%We also randomly selected from the projects in FuzzyCatch dataset the
%same amount of snippets with no exception-related bugs as the negative
%samples.

%We have conducted our experiments to evaluate {\tool} on the dataset DeepEx that we collected, and the existing dataset FuzzyCatch provided in a prior work~\cite{nguyen2020code}.

%To build the dataset DeepEx, we first collect XXX methods from XXX Java projects. And then, we picked XXX methods that contain at least one try-catch block from all methods. To avoid the influence of multiple try-catch blocks, we split the methods into code snippets by checking the above and below statements, one by one, starting from the closest one for each try-catch block, to verify if the statement is included in the other try-catch block. If not, we put it in the current code snippet. If yes, we stop here and finish the code snippet.

%Following the steps mentioned above, we have XXX code snippets containing one try-catch block as positive data. Then, we randomly create the same amount of code snippets that do not contain a try-catch block as negative data from the Java projects we collected. So, in total, the DeepEx dataset includes XXX code snippets. As for the FuzzyCatch dataset, it includes 1,000 data, and all of them are positive data.


\subsubsection{RQ1. Effectiveness on Try-Catch Necessity Checking}~\\
\indent {\em Baselines.} We compared {\xblock} with
GPT-3.5~\cite{ChatGPT}. Due to cost of using GPT-3.5 on OpenAI, we
performed sampling on the GitHub dataset of 30,764 snippets. To obtain
the confidence level of 95\%, we randomly selected 380 code snippets
in which 190 are negative samples (no \code{try-catch} block), and 190
are positive samples (at least one \code{try-catch} block). We trained
on GitHub dataset and compared with GPT-3.5 on this sampled test set.

%the pre-trained CodeBERT without any fine-tuning steps. In the
%pre-trained CodeBERT, We add a randomly-initialized linear layer on
%top of the output vector of the \texttt{[CLS]} token, and use a
%softmax function to learn the decision.
We also compared {\xblock} with XRank~\cite{xrank-fse20} (XRank is
part of FuzzyCatch) on the GitHub dataset. XRank computed
the exception risk score for each API call. If a score of a call in
the snippet is higher than a threshold, we consider it as needing a
\code{try-catch} block.

{\em Procedure.} We randomly split both the positive and negative sets
in a dataset into 80\%, 10\%, and 10\% of the code
snippets for training, tuning, and testing. Meanwhile, we make sure
that each partition contains the equal amount of positive samples and
negative samples; and the training and tuning partitions do not
contain any duplicates.

To request responses from ChatGPT, we construct prompt with the format
"question + code", where the question we pose is "Does the code below
need to catch any exceptions?\textbackslash n\textbackslash n".
Considering that the answers from ChatGPT for the same prompt may
vary, for each code snippet, we send the prompt three times through
the Chat Completions API.
%(https://platform.openai.com/docs/guides/gpt/chat-completions-api).
Labeling the responses has three steps. First, we check whether the
first word in each response is Yes or No. If it is a Yes, we assign a
positive label; If it is a No, we assign a negative label. Second, for
the responses that do not start with Yes or No, we read each response
and manually assign labels to them.  However, there are some cases in
which it is hard to make the decision on whether or not the code needs
any try-catch blocks. The common scenarios are (1) the response is not
informative enough, as it only relays a general advice about exception
handling, (2) the response states that it is uncertain whether a
try-catch block is needed, or (3) the decision making depends on the
background knowledge on either the project structure or certain method
specifications. Thus, in evaluating GPT-3.5's performance, for
these responses, given the benefit of doubts, we assign correct
prediction labels to them, assuming the best performance of GPT-3.5.
%for each metric, we calculate the lower and upper bound. In the upper
%bound, all such uncertain cases are considered as correct,
%in the lower bound, all such cases are considered as incorrect.
We assign the final label for each instance from the majority
vote of the three attempts.

%We took all the code snippets from DeepEX and FuzzyCatch datasets. On
%the DeepEx dataset, we randomly split both the positive and negative
%data points into 80\%, 10\%, 10\%, in which 80\% of the code snippets
%as the training dataset, 10\% of the code snippets as the tuning
%dataset, and 10\% of the code snippets as the testing dataset for the
%baseline and {\tool}.

%THIS PART FOR RQ5
%----------------
%And on the FuzzyCatch dataset, we directly use
%the trained model from the DeepEx dataset and test it on the
%FuzzyCatch dataset for both the baseline and {\tool}.

{\em Tuning.} We trained {\tool} for $15$ epochs with the following key hyper-parameters: (1) Batch size is set to $32$; (2) Learning rate is set to $0.000006$; (3) Weight decay is set to $0.01$. We select the model with the lowest overall validation loss.

%We tuned {\tool} with autoML~\cite{NNI} for the
%following key hyper-parameters to have the best performance: (1) Epoch
%size (50, 100, 150); (2) Batch size (32, 64, 128); (3) Learning rate
%(0.001, 0.003, 0.005); (4) Vector length of feature embeddings and its
%output (64, 128, 256); (5) Number of R-GCN layers (4, 6, 8).

{\em Metrics.} We use \textbf{Precision, Recall}, and {\bf F1-score} to
evaluate the performance of the approaches. They are calculated as follows.
$Precision = \frac{TP}{TP+FP}$, $Recall = \frac{TP}{TP+FN}$, $F1$-score
$=$ $\frac{2*Recall*Precision}{Recall+Precision}$. $TP$: true
positive, $FN$: false negative, and $FP$: false positive.


\subsubsection{RQ2. Effectiveness on Try-Catch Statement Detection\\}

\indent {\em Baselines.} We compared {\xstate} against the pre-trained
CodeBERT model without any fine-tuning step as in RQ1.

{\em Procedure.} We used the same procedure as in RQ1.

{\em Tuning.} We used the same tuning as in RQ1.

{\em Metrics.} We used the same metrics as in RQ1. However, we
computed Recall, Precision, and F-score in two ways. First, we
evaluated {\xstate} in connection with {\xblock}. That is, we consider
a correct case if {\xblock} gives a correct prediction on whether a
\code{try-catch} block is needed, {\em and} {\xstate} predicts
correctly on whether a statement needs to be in such a block. Second,
we evaluated {\xstate} as individual. That is, we assumed that
{\xblock} predicts correctly whether a code snippet needs a
\code{try-catch} block or not. Thus, we used the code snippets that
need such a block in the oracle and used {\xstate} to predict if the
statements in those snippets need to be inside a \code{try-catch}
block or not.

%For a snippet that needs \code{try-catch}, but was predicted as
%not, we consider it as incorrect because the resulting statements
%predicted from {\xstate} do not make sense for incorrect {\xblock}'s
%detection. A snippet that does not need \code{try-catch}, but were
%predicted as yes, we also consider them as incorrect for the same
%reason. For a snippet that does not need a \code{try-catch} block,
%and was predicted as not needing, we consider it as correct.
%For a snippet needing such a block, and was predicted as yes,
%we consider if the current statement was predicted correctly or not.

%Thus, for the evaluation of {\xstate}, we used the set of code
%snippets that {\xblock} predicted correctly as needing a
%\code{try-catch} block. We used the same data splitting scheme with
%80\%, 10\%, and 10\% for training, tuning, and testing as in RQ1. Each
%code snippet in the testing set and the trained R-GCN model in
%\xblock} are the input of the GNNExplainer model in {\xstate} in this
%experiment.

%{\em Tuning.} We used the same parameters as in
%GNNExplainer~\cite{GNNExplainer}. It also has a parameter on the limit
%of the number $N$ of the nodes in the explanation sub-graph. We
%varied $N$ from 1 to 10. The average number of statements in a
%\code{try-catch} block in Github dataset is 5.9.

%{\em Metrics.} If {\xstate} predicts for a statement correctly if it
%is in a \code{try-catch} block or not, we count it as a correct case.
%Otherwise, it is an incorrect one. \textbf{Accuracy} is
%defined as the ratio between the number of correct statements over the
%total number of statements.


\subsubsection{RQ3. Effectiveness on Exception Type Recommendation\\}

{\em Baselines.} We compared {\xstate} against the pre-trained
CodeBERT model without any fine-tuning step as in RQ1.

%We compared {\tool} against the state-of-the-art exception type recommendation approach XRank~\cite{nguyen2020code}.

{\em Procedure.} We used the same procedure as in RQ1.

%In this RQ, \tool is evaluated on the DeepEx dataset. Because only the data points that \tool can successfully predict all statements in the try-catch block successfully, the \tool will run the exception type prediction on them. Similarly, as RQ2, to fully use the dataset and evaluate \tool on more data, here we estimate the Accuracy of RQ2 prediction is 100\% which means that we use all positive data in DeepEx dataset as the data we use in this RQ. We do the same data split as RQ1 and RQ2 to train, validate and test the model performance.

{\em Tuning.} We used the same tuning as in RQ1.

{\em Metrics.} We used the same metrics as in RQ1. However, we
computed Recall, Precision, and F-score in two ways. First, we evaluate {\xtype}
in connection with {\xblock} and {\xstate}. That is, we consider a correct case if {xblock} 
gives a correct prediction on whether \code{try-catch} blocks are needed, {\xstate} produces 
correct tags for all the statements, and {\xtype} 
predicted correctly on the exceptions that need to be caught for all \code{try-catch} blocks. 
Second, we evaluate {\xtype} as individual. That is, we assumed that both {\xblock} and {\xstate}
have given 100\% correct prediction results for their own tasks. Thus, we put the code snippets that
need \code{try-catch} blocks and also have statements all labelled correctly in the oracle and used
{\xtype} to predict exceptions to be caught.

%Let us use $E$ and $P$ to denote the set of the
%exception types in the oracle and the set of predicted
%ones for one code snippet in the testing set. We aim to measure 1) how
%precise {\tool} predicts the exception types ({\bf Precision}), i.e.,
%how accurate the predicted types in $P$, and 2) how much {\tool} can
%cover in its prediction with respect to the oracle ({\bf Recall}),
%i.e., how much of $E$ that the predicted set $P$ can cover. Toward
%measuring Precision and Recall, we define {\bf Hit-$n$} as the number
%of the cases in which the predicted set $P$ contains
%{\em at least} $n$ correct exception types, i.e., $P$ and $E$ overlaps
%{\em at least} $n$ exception types regardless of the sizes of both
%sets.
%
%In Github dataset, more than 98.1\% of the code snippets have 1--3
%exception types in a \code{catch} clause. Thus, we compute Hit-$n$,
%Precision, and Recall for the size of $E$ (|$E$|) from 1--3 and 3+,
%and for the size of $P$ (|$P$|) from 1--3 and 3+.
%%
%When computing Recall, we use the set $E$ as the basis. Recall at a
%size $N_E$ of $E$ is computed as the ratio between Hit-$n$ at that
%size and the total number of cases with that size $N_E$. We compute
%Recall for all $N_E$ = 1--3, and 3+, and $n$=1--$N_E$. We also define
%{\bf \code{Hit-All}$_{Rec}$} as \code{Hit}-$n$ when the number $n$ of
%overlapping exception types is equal to |$E$|, i.e., all exception
%types in the oracle set for a code snippet are covered.
%%
%When computing Precision, we use the set $P$ as the basis. Precision
%at a size $N_P$ of $P$ is computed as the ratio between Hit-$n$ at
%that size and the total number of cases with that size $N_P$. We
%compute Precision for all $N_P$ = 1--3, and $n$=1--$N_P$. We also
%define {\bf \code{Hit-All}$_{Prec}$} as \code{Hit}-$n$ when the number $n$
%of overlapping exception types is equal to |$P$|, i.e., all predicted
%types in the predicted set $P$ for a snippet are
%correct.


%In this RQ, we use \textbf{Hit-n} as the evaluation metrics. Hit-n
%here means within the true labeled statement set, there are at least
%\textbf{n} statements predicted correctly by \tool.


%\subsubsection{RQ2. Effectiveness on Try-Catch Statement Detection\\}

%\indent {\em Baselines.} None. FuzzyCatch (XRank/XHand) does not have this.

%{\em Procedure.} We processed the GitHub dataset for this experiment
%as follows. For the snippets that need \code{try-catch}, but were
%predicted as not, we consider them as incorrect because the resulting
%statements predicted from {\xstate} do not make sense for incorrect
%{\xblock}'s detection. The snippets that do not need \code{try-catch},
%but were predicted as yes, we also consider them as incorrect for the
%same reason. Thus, for the evaluation of {\xstate}, we used the set of
%code snippets that {\xblock} predicted correctly as needing a
%\code{try-catch} block. We used the same data splitting scheme with
%80\%, 10\%, and 10\% for training, tuning, and testing as in RQ1. Each code
%snippet in the testing set and the trained R-GCN model in {\xblock}
%are the input of the GNNExplainer model in {\xstate} in this
%experiment.

%{\em Tuning.} We used the same parameters as in
%GNNExplainer~\cite{GNNExplainer}. It also has a parameter on the limit
%of the number $N$ of the nodes in the explanation sub-graph. We
%varied $N$ from 1 to 10. The average number of statements in a
%\code{try-catch} block in GitHub dataset is 5.9.

%{\em Metrics.} If {\xstate} predicts for a statement correctly if it
%is in a \code{try-catch} block or not, we count it as a correct case.
%Otherwise, it is an incorrect one. \textbf{Accuracy} is
%defined as the ratio between the number of correct statements over the
%total number of statements.



\subsubsection{RQ4. Statement Dependency Probing}~\\
%In this question,
We aim to evaluate if {\tool} could learn the
connections/dependencies among the statements
%to be placed
inside the same \code{try-catch} block. We selected the
instances that {\tool} predicted correctly in all three tasks. For
a \code{try-catch} block in an instance, we randomly selected a
statement $S_1$ and another statement $S_2$ inside the block. We then
randomly selected another statement $T$ outside of the block. We
measured the cosine distance $d_1(S_1,S_2)$ and $d_2(S_1,T)$ for the
statement embeddings. We repeated the process for all triples of
$(S_1,S_2,T)$ in the Github dataset, and computed the cosine distances
for the group of inside statement pairs and the group of
inside-to-outside statement pairs. For each group, we constructed
confidence intervals at 95\% confidence via bootstrapping for the mean
of the distances (the number of resamples is set to be 1000)

%Based on the population in our dataset, that size gives the confidence level of
%95\% and the confidence interval of 5\%. 
%We used statistical $p$-value evaluate our hypothesis $H_1$: $d_1 \geq d_2$. The null-hypothesis $H_0$ is $d_1 > d_2$.

%In this research question, we examine whether {\tool} puts more attention weights for statements inside try blocks. The evaluation is done on the correctly predicted portion of the Github dataset. For each input code snippet, we feed it through the Transformer body of {\tool}, and extract the attention weight matrix from the last layer. We sum up the attention weights between statements inside try blocks, and sum up the attention weights from statements inside to statements outside try blocks (recall that statements are represented by [SEP] tokens). Therefore, for all instances, we have a list of attention scores for inside statements (see Formula~\ref{E:ins}), and a list of attention scores for inside to outside statements (see Formula~\ref{E:out}). Through two sample t\text{-}test, we analyze whether means of the two lists ($\overline{X}_{in}$ and $\overline{Y}_{in\rightarrow out}$) are different and report the p-value. If $p\text{-}value < 0.05$, we reject the Null Hypothesis---there is no difference between the means.

%\begin{equation}\label{E:ins}
%X_{in} = [ S^{1}_{attn, in}, S^{2}_{attn, in}, ... , S^{n}_{attn, in} ]
%\end{equation}

%\begin{equation}\label{E:out}
%Y_{in\rightarrow out} = [ S^{1}_{attn, in\rightarrow out}, S^{2}_{attn, in\righ%tarrow out}, ... , S^{n}_{attn, in\rightarrow out} ]
%\end{equation}
 

%
%In this experiment, we aim to evaluate the impact on the performance
%of the key features: sequences of code tokens and Abstract Syntax Tree.
%We removed one key feature at a time and compared the performance with
%the original model to evaluate its impact. We used the same evaluation
%metrics as in the previous experiments. Because our model is designed
%with R-GCN, we cannot remove it to evaluate the impact of dependencies.
%{\em Metrics.} We use the same evaluation metrics as RQ1 and RQ3 to evaluate the impact of the different features on the experiment results.
\vspace{2pt}
\subsubsection{Extrinsic Evaluation on Exception Bug Detection (RQ5)}~\\
\noindent {\em Baselines.} We compared with
FuzzyCatch~\cite{xrank-fse20} in exception-related
bug detection, i.e., missing \code{try-catch} blocks and/or exceptions.

%leverages XRank to detect the exception-related bugs, which are the
%StackOverflow code snippets that were supposed to handle exceptions,
%but missed \code{try-catch} blocks and/or exceptions.

%We used {\tool} to detect such bugs and compare with FuzzyCatch.

{\em Procedure.} We trained {\tool} on the GitHub dataset and detected
the exception-related bugs~\cite{fuzzycatchbugs} in FuzzyCatch dataset of
incomplete code snippets. If exception handling in a snippet
matches exactly with {\tool}'s suggestion, we consider it as correct.

%To make the FuzzyCatch dataset a balanced dataset for a fair comparison, we randomly select the same number of non-buggy code snippets comparing the bugs as the negative samples from the methods in the FuzzyCatch dataset.

%{\em Metrics.}
%We use {\bf Recall}, {\bf Precision}, and {\bf F1-score} as in RQ1.

\vspace{2pt}
\subsubsection{Ablation Study (RQ6)}~\\
We aim to evaluate the contribution of fine-tuning in {\tool}. We
compared {\tool} against CodeBERT without fine-tuning.
