\subsection{Key Ideas}
\label{key:sec}

\noindent We design {\tool} for exception handling recommendation for
Java code: 1) given a code snippet, it will point out which statements
in the snippet need to be placed within a \code{try-catch} block, and
what exceptions need to be handled in the catch clause. Following
Observations 1--4, we design {\tool} with the following key ideas:

%\vspace{3pt}
\subsubsection{{\bf [Key Idea 1] Neural Network-Based Approach to Exception Handling Recommendation}}
%\vspace{2pt}
%\subsubsection*{{\bf [Key Idea 1] Neural Network-Based Approach to Partial Program Dependence Analysis}}
Instead of deterministically deriving the exceptions to be handled in
a heuristic or mining manner for a given (incomplete) code snippet, following
Observation 2, we design a deep learning model (DL) to learn to
analyze that snippet to recommend the statements in the
\code{try-catch} block and the exceptions in the \code{catch} clause.
By leveraging the \code{try-catch} blocks of the complete code in the
open-source projects (e.g., GitHub) in the training process, our DL
model can decide what statements in the given code snippet to be
placed in a \code{try-catch} block and what exceptions to be handled.


%\vspace{2pt}
\subsubsection{{\bf [Key Idea 2] Span-based ...}}
We seek inspiration from the neural network-based dependency parsing
approaches~\cite{?} in Natural Language Processing (NLP). They
successfully learn ... Following suit, we design \tool to learn the
representations for the statements in source code so as to learn the
span ...

\vspace{2pt}
\subsubsection{{\bf [Key Idea 3] Context ...}}
...
