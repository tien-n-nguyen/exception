\subsection{Key Ideas}
\label{key:sec}

\noindent We design {\tool} for exception handling recommendation for
Java code: given a code snippet, it will point out which statements in
the snippet need to be placed within a \code{try-catch} block, and
what exceptions need to be handled in the catch clause. We refer to
those two tasks as {\xstate} and {\xtype}, respectively. Following
Observations 1-5, we design it with the following key ideas:

%\vspace{3pt}
\subsubsection{{\bf [Key Idea 1] Neural Network-Based Approach to Exception Handling Recommendation}}
%\vspace{2pt}
%\subsubsection*{{\bf [Key Idea 1] Neural Network-Based Approach to Partial Program Dependence Analysis}}
Instead of deterministically deriving the exceptions to be handled for
a given (incomplete) code snippet, following Observation~\ref{ob2}, we
design a deep learning model (DL) to learn to analyze that snippet to
recommend the statements to be placed within the \code{try-catch}
block and to recommend the exceptions in the \code{catch} clause.  By
leveraging the \code{try-catch} blocks of the complete code in the
open-source projects (e.g., GitHub) in the training process, our DL
model can help the adaptation in those above two tasks: {\xstate} and
{\xtype}.

%: 1) deciding what
%statements in the given code snippet to be placed in a
%\code{try-catch} block, and 2) deciding what exceptions to be handled.

\vspace{2pt}
\subsubsection{{\bf [Key Idea 2] Leveraging Context to Avoid
Ambiguity and Learning the Connections between API elements and
Exception Types}} Instead of learning only the associations between an
API element and exception types as in XRank~\cite{xrank-fse20}, we
leverage as the context the complete methods in the code corpus for
training, which are parseable and provide us the identities (i.e.,
FQNs) of the API elements. Our DL model will also leverage the context
and dependencies among the API elements to learn their identities as
explained in Observation~\ref{ob4}. Importantly, that leads to the
learning of the connections between the key API elements in the
context and the exception types that need to be handled
(Observation~\ref{ob3}).

\subsubsection{{\bf [Key Idea 3] Dual-Task Learning between {\xstate} and {\xtype}}} The learning of deciding what statements to be in a \code{try-catch}
block could benefit much on the learning of deciding what exception
types to be handled. In Figure~\ref{fig:example4}, if a model learns
that line 3 (with \code{new\-Buffered\-Reader}) and line 5 (with
\code{read\-Line}) need to be in \code{try-catch}, it could learn from
the code corpus and decide \code{IOException} as the exception type to
be handled. In contrast, if a model learns correctly that the
\code{IOException} needs to be handled, from the history, it can learn
that \code{new\-Buffered\-Reader} and \code{read\-Line} can throw that
exception and determine that the line 3 and 5 need to be in
\code{try-catch}.

\subsubsection{{\bf [Key Idea 4] Span-based ...}}
We seek inspiration from the neural network-based span-based ...
approaches~\cite{?} in Natural Language Processing (NLP). They
successfully learn ... Following suit, we design \tool to learn the
representations for the statements in source code so as to learn the
span ...


