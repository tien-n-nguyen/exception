\subsection{Ablation Study (RQ4)}
\label{sec:rq4}



\begin{table}[t]
  \caption{Impact of Different Features on {\xblock} (RQ4)}
    %Try-Catch Block Necessity Checking}
  \vspace{-12pt}
  \begin{center}
    \small
		\renewcommand{\arraystretch}{1}
		\begin{tabular}{p{1.75cm}<{\centering}|p{1.75cm}<{\centering}|p{1.75cm}<{\centering}|p{1.75cm}<{\centering}}
			\hline
			  & \tool w/o SOT & \tool w/o AST & \tool \\
			\hline
			Recall    & 0.69 & 0.71 & \textbf{0.79} \\
			Precision & 0.64 & 0.57 &\textbf{0.68} \\
			F-score   & 0.66 & 0.63 &\textbf{0.73} \\
			\hline
		\end{tabular}
		SOT: Sequence of tokens; AST: Abstract Syntax Tree
		\label{tab:sensi-xblock}
	\end{center}
\end{table}

\subsubsection{Impact of Different Features on Try-Catch Block Necessity Checking ({\xblock})}

Table~\ref{tab:sensi-xblock} displays the results when we removed the
two key features in {\tool} and measured {\xblock}'s performance. As
seen, without the sequences of code (lexical tokens) for each
statement, Recall, Precision, and F-score decrease 12.7\%, 5.9\%, and
9.6\%, respectively. Without considering the AST structure, Recall,
Precision, and F-score also decrease even further with 10.1\%, 16.2\%, and
13.7\%, respectively. In other words, {\em the code structure has a higher
contribution than the lexical values of code tokens}.


%{\color{red}{Without sequence of tokens or AST in try-catch block necessity checking will lead to the reduce of results. Once I finish this sensitivity experiments, I will update this table. }}

\begin{table}[t]
  \caption{Impact of Different Features on {\xstate} (RQ4)}
  \vspace{-12pt}
    %\code{Try-catch} Statement Prediction}
  \small
	\begin{center}
		\renewcommand{\arraystretch}{1}
		\begin{tabular}{p{1.75cm}<{\centering}|p{1.75cm}<{\centering}|p{1.75cm}<{\centering}|p{1.75cm}<{\centering}}
			\hline
			  & \tool w/o SOT & \tool w/o AST & \tool \\
			\hline
			Accuracy    & 0.75 & 0.73 & \textbf{0.74} \\
			\hline
		\end{tabular}
		SOT: Sequence of tokens; AST: Abstract Syntax Tree
		\label{tab:sensi-xstate}
	\end{center}
\end{table}

\subsubsection{Impact of Different Features on Try-Catch Statement Detection ({\xstate})}

%Table~\ref{tab:sensi-xstate} displays the results when we removed the
%two key features in {\tool} and measured the performance of {\xstate}.
We used six as the limit on the number of nodes in the explanation
sub-graph in which {\tool} achieves 0.74 of Accuracy. Note: the
average number of the statements in a \code{try-catch} block in our
dataset is 5.9. As seen, the result in Table~\ref{tab:sensi-xstate} is
consistent with Table~\ref{tab:sensi-xblock}, as AST contributes
slightly more than code sequences.

%without the sequences of code (lexical tokens) for each statement,
%Accuracy decreases XX\%. Without considering the AST structure,
%Accuracy also decreases even further with XX\%. In other words, the
%code structure has a higher contribution than the lexical values of
%code tokens.

\subsubsection{Impact of Different Features on Exception Type Recommendation ({\xtype})}

Tables~\ref{tab:sensi-xtype-recall}
and~\ref{tab:sensi-xtype-precision} display the results. As seen, the
impact result is consistent with those in
Tables~\ref{tab:sensi-xblock} and~\ref{tab:sensi-xstate}.
Thus, code structure has slightly higher impact than code sequence.

%when we removed the two key features in {\tool} and measured the
%performance of {\xtype}. As seen, without the sequences of code
%(lexical tokens) for each statement, all the evaluation metrics
%\code{Hit}-$n$ decrease across the board in both recall and
%precision. Without considering the AST structure, all the evaluation
%metrics decrease slightly further.  Thus, this result is consistent
%with the above results as the code structure has a higher contribution
%than the lexical values of code tokens.

\begin{table}[t]
  \caption{Impact of Different Features on {\xtype} (Recall)}
  \vspace{-12pt}
	\tabcolsep 2pt
	{\small
		\begin{center}
			\renewcommand{\arraystretch}{1}
			\begin{tabular}{p{2cm}<{\centering}|p{1cm}<{\centering}|p{1.5cm}<{\centering}|p{1.5cm}<{\centering}|p{1.5cm}<{\centering}}
				\hline
				\# ET in \code{Try-Catch} Block in Oracle & Metrics &{\textsc{\tool w/o SOT}\xspace}&{\textsc{\tool w/o AST}\xspace}& {\textsc{\tool}\xspace} \\
				\hline
				\multirow{1}{*}{1 (1,385 instances)} & Hit-1  &833 (60\%)& 791 (57\%)& 918 (66\%) \\
				\hline
				\multirow{2}{*}{2 (98 instances)}    & Hit-1  &69 (70\%)& 66 (67\%)& 75 (77\%) \\
				                                     & Hit-2  &39 (40\%)& 37 (38\%)&  41 (42\%) \\
				\hline
				\multirow{3}{*}{3 (18 instances)}    & Hit-1  &10 (56\%)&11 (61\%)& 12 (67\%) \\
				                                     & Hit-2  &4 (22\%)& 4 (22\%)& 5 (28\%)\\
				                                     & Hit-3  &2 (11\%)& 3 (17\%)& 3 (17\%) \\
				\hline
				\multirow{4}{*}{3+ (30 instances)}   & Hit-1  &16 (53\%)&15 (50\%)& 18 (60\%) \\
				                                     & Hit-2  &9 (30\%)& 8 (27\%)& 10 (33\%) \\
				                                     & Hit-3  &3 (10\%)& 2 (7\%)& 4 (13\%)\\
				\hline
			\end{tabular}
		ET:Exception types; SOT: Sequence of tokens; AST: Abstract syntax tree		
			\label{tab:sensi-xtype-recall}
		\end{center}
	}
\end{table}

\begin{table}[t]
	\caption{Impact of Different Features on {\xtype} (Precision)}
	\vspace{-10pt}
	\tabcolsep 2pt
	{\small
		\begin{center}
			\renewcommand{\arraystretch}{1}
			\begin{tabular}{p{2cm}<{\centering}|p{1cm}<{\centering}|p{1.5cm}<{\centering}|p{1.5cm}<{\centering}|p{1.5cm}<{\centering}}
				\hline
				\# ET in Predicted \code{Try-Catch} Block & Metrics &{\textsc{\tool w/o SOT}\xspace}&{\textsc{\tool w/o AST}\xspace}& {\textsc{\tool}\xspace} \\
				\hline
				\multirow{1}{*}{1 (1,154 instances)}   & Hit-1  &357 (31\%)& 309 (27\%)& 442 (38\%) \\
				\hline
				\multirow{2}{*}{2 (294 instances)}  & Hit-1   &83 (28\%)& 78 (27\%)& 90 (31\%) \\
				& Hit-2       						&57 (19\%)& 51 (17\%)& 61 (21\%) \\
				\hline
				\multirow{3}{*}{3 (83 instances)}  & Hit-1    & 24 (29\%)& 21 (25\%)& 27 (33\%) \\
				& Hit-2         					&19 (23\%)& 18 (22\%)& 23 (28\%)\\
				& Hit-3         				  	&6 (7\%)& 4 (5\%)& 6 (7\%) \\
				\hline
			\end{tabular}
		ET: Exception types; SOT: Sequence of tokens; AST: Abstract syntax tree
			\label{tab:sensi-xtype-precision}
		\end{center}
	}
\end{table}

%{\color{red}{Without sequence of tokens or AST in exception type prediction will lead to the reduce of results of all hit-1, hit-2, and hit-3. Once I finish this sensitivity experiments, I will update these two table. }}


%%%\begin{table}[t]
%%%	\caption{RQ4. Impact of the \# Limit of Exception Type on Exception Type Prediction (Recall)}
%%%	\tabcolsep 2pt
%%%	{\small
%%%		\begin{center}
%%%			\renewcommand{\arraystretch}{1}
%%%			\begin{tabular}{p{3.5cm}<{\centering}|p{2cm}<{\centering}|p{2.5cm}<{\centering}}
%%%				\hline
%%%				\# ET in Oracle Block& Metrics& {\textsc{\tool}\xspace} \\
%%%				\hline
%%%				\multirow{1}{*}{1 (40376)}   & Hit-1  & 27867 (69\%) \\
%%%				\hline
%%%				\multirow{2}{*}{2 (2634)}  & Hit-1   & 2131 (81\%) \\
%%%				& Hit-2         &  1162 (44\%) \\
%%%				\hline
%%%				\multirow{3}{*}{3 (547)}  & Hit-1    & 353 (64\%) \\
%%%				& Hit-2     & 189 (35\%)\\
%%%				& Hit-3     & 142 (26\%) \\
%%%				\hline
%%%				\multirow{4}{*}{4 (267)}  & Hit-1   & 175 (65\%) \\
%%%				& Hit-2     & 107 (40\%) \\
%%%				& Hit-3     & 55 (21\%))\\
%%%				& Hit-4     & 21 (8\%))\\
%%%				\hline
%%%				\multirow{4}{*}{4+ (124)}  & Hit-1   & 77 (62\%) \\
%%%				& Hit-2     & 46 (38\%) \\
%%%				& Hit-3     & 21 (17\%))\\
%%%				& Hit-4     & 5 (4\%))\\
%%%				\hline
%%%			\end{tabular}		
%%%			ET: Exception types
%%%			\label{RQ4_results_4}
%%%		\end{center}
%%%	}
%%%\end{table}

%%%\begin{table}[t]
%%%	\caption{RQ4. Impact of the \# Limit of Exception Type on Exception Type Prediction (Precision)}
%%%	\vspace{-10pt}
%%%	\tabcolsep 2pt
%%%	{\small
%%%		\begin{center}
%%%			\renewcommand{\arraystretch}{1}
%%%			\begin{tabular}{p{3.5cm}<{\centering}|p{2cm}<{\centering}|p{2.5cm}<{\centering}}
%%%				\hline
%%%				\# ET in Predicted Block & Metrics & {\textsc{\tool}\xspace} \\
%%%				\hline
%%%				\multirow{1}{*}{1 (34707)}   & Hit-1  & 11113 (32\%) \\
%%%				\hline
%%%				\multirow{2}{*}{2 (5892)}  & Hit-1   & 1829 (31\%) \\
%%%				& Hit-2       						& 764 (13\%) \\
%%%				\hline
%%%				\multirow{3}{*}{3 (2321)}  & Hit-1    & 654 (28\%) \\
%%%				& Hit-2         					& 437 (19\%)\\
%%%				& Hit-3         				  	& 159 (7\%) \\
%%%				\hline
%%%				\multirow{4}{*}{4 (1028)}  & Hit-1    & 268 (26\%) \\
%%%				& Hit-2         					& 211 (21\%)\\
%%%				& Hit-3         				  	& 94 (9\%) \\
%%%				& Hit-4         				  	& 33 (3\%) \\
%%%				\hline
%%%			\end{tabular}
%%%			ET: Exception types
%%%			\label{RQ4_results_5}
%%%		\end{center}
%%%	}
%%%\end{table}
