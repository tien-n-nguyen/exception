\subsection{RQ4. Ablation Study}
\label{sec:rq4}


\begin{table}[h]
	\caption{RQ4. Impact of Different Features on Try-Catch Block Necessity Checking}
	\begin{center}
		\renewcommand{\arraystretch}{1}
		\begin{tabular}{p{1.75cm}<{\centering}p{1.75cm}<{\centering}p{1.75cm}<{\centering}p{1.75cm}<{\centering}}
			\hline
			Category  & \tool w/o SOT & \tool w/o AST & \tool \\
			\hline
			Recall    & & & \textbf{0.81} \\
			Precision & & &\textbf{0.66} \\
			F-score   & & &\textbf{0.73} \\
			\hline
		\end{tabular}
		SOT: Sequence of tokens; AST: Abstract syntax tree
		\label{RQ4_results_1}
	\end{center}
\end{table}

{\color{red}{Without sequence of tokens or AST in try-catch block necessity checking will lead to the reduce of results. Once I finish this sensitivity experiments, I will update this table. }}

\begin{table}[t]
	\caption{RQ4. Impact of Different Features on Exception Type Prediction (Recall)}
	\tabcolsep 2pt
	{\small
		\begin{center}
			\renewcommand{\arraystretch}{1}
			\begin{tabular}{p{2cm}<{\centering}|p{1cm}<{\centering}|p{1.5cm}<{\centering}|p{1.5cm}<{\centering}|p{1.5cm}<{\centering}}
				\hline
				\# ET in Oracle Block& Metrics &{\textsc{\tool w/o SOT}\xspace}&{\textsc{\tool w/o AST}\xspace}& {\textsc{\tool}\xspace} \\
				\hline
				\multirow{1}{*}{1 (40376)}   & Hit-1  &&& 25841 (64\%) \\
				\hline
				\multirow{2}{*}{2 (2634)}  & Hit-1   &&& 2028 (77\%) \\
				& Hit-2         &&&  1081 (41\%) \\
				\hline
				\multirow{3}{*}{3 (547)}  & Hit-1    &&& 334 (61\%) \\
				& Hit-2     &&& 181 (33\%)\\
				& Hit-3     &&& 115 (21\%) \\
				\hline
				\multirow{4}{*}{3+ (391)}  & Hit-1   &&& 233 (59\%) \\
				& Hit-2     &&& 134 (35\%) \\
				& Hit-3     &&& 65 (17\%))\\
				\hline
			\end{tabular}
		ET:Exception types; SOT: Sequence of tokens; AST: Abstract syntax tree		
			\label{RQ4_results_2}
		\end{center}
	}
\end{table}

\begin{table}[t]
	\caption{RQ4. Impact of Different Features on Exception Type Prediction (Precision)}
	\vspace{-10pt}
	\tabcolsep 2pt
	{\small
		\begin{center}
			\renewcommand{\arraystretch}{1}
			\begin{tabular}{p{2cm}<{\centering}|p{1cm}<{\centering}|p{1.5cm}<{\centering}|p{1.5cm}<{\centering}|p{1.5cm}<{\centering}}
				\hline
				\# ET in Predicted Block & Metrics &{\textsc{\tool w/o SOT}\xspace}&{\textsc{\tool w/o AST}\xspace}& {\textsc{\tool}\xspace} \\
				\hline
				\multirow{1}{*}{1 (36995)}   & Hit-1  &&& 13589 (37\%) \\
				\hline
				\multirow{2}{*}{2 (4981)}  & Hit-1   &&& 1594 (32\%) \\
				& Hit-2       						&&& 947 (19\%) \\
				\hline
				\multirow{3}{*}{3 (1972)}  & Hit-1   && & 611 (31\%) \\
				& Hit-2         					&&& 458 (23\%)\\
				& Hit-3         				  	&&& 185 (9\%) \\
				\hline
			\end{tabular}
		ET: Exception types; SOT: Sequence of tokens; AST: Abstract syntax tree
			\label{RQ4_results_3}
		\end{center}
	}
\end{table}

{\color{red}{Without sequence of tokens or AST in exception type prediction will lead to the reduce of results of all hit-1, hit-2, and hit-3. Once I finish this sensitivity experiments, I will update these two table. }}


\begin{table}[t]
	\caption{RQ4. Impact of the \# Limit of Exception Type on Exception Type Prediction (Recall)}
	\tabcolsep 2pt
	{\small
		\begin{center}
			\renewcommand{\arraystretch}{1}
			\begin{tabular}{p{3cm}<{\centering}|p{2cm}<{\centering}|p{2cm}<{\centering}}
				\hline
				\# ET in Oracle Block& Metrics& {\textsc{\tool}\xspace} \\
				\hline
				\multirow{1}{*}{1 (40376)}   & Hit-1  & 27867 (69\%) \\
				\hline
				\multirow{2}{*}{2 (2634)}  & Hit-1   & 2131 (81\%) \\
				& Hit-2         &  1162 (44\%) \\
				\hline
				\multirow{3}{*}{3 (547)}  & Hit-1    & 353 (64\%) \\
				& Hit-2     & 189 (35\%)\\
				& Hit-3     & 142 (26\%) \\
				\hline
				\multirow{4}{*}{4 (267)}  & Hit-1   & 175 (65\%) \\
				& Hit-2     & 107 (40\%) \\
				& Hit-3     & 55 (21\%))\\
				& Hit-4     & 21 (8\%))\\
				\hline
				\multirow{4}{*}{4+ (124)}  & Hit-1   & 77 (62\%) \\
				& Hit-2     & 46 (38\%) \\
				& Hit-3     & 21 (17\%))\\
				& Hit-4     & 5 (4\%))\\
				\hline
			\end{tabular}		
			ET: Exception types
			\label{RQ4_results_4}
		\end{center}
	}
\end{table}

\begin{table}[t]
	\caption{RQ4. Impact of the \# Limit of Exception Type on Exception Type Prediction (Precision)}
	\vspace{-10pt}
	\tabcolsep 2pt
	{\small
		\begin{center}
			\renewcommand{\arraystretch}{1}
			\begin{tabular}{p{3cm}<{\centering}|p{2cm}<{\centering}|p{2cm}<{\centering}}
				\hline
				\# ET in Predicted Block & Metrics & {\textsc{\tool}\xspace} \\
				\hline
				\multirow{1}{*}{1 (34707)}   & Hit-1  & 11113 (32\%) \\
				\hline
				\multirow{2}{*}{2 (5892)}  & Hit-1   & 1829 (31\%) \\
				& Hit-2       						& 764 (13\%) \\
				\hline
				\multirow{3}{*}{3 (2321)}  & Hit-1    & 654 (28\%) \\
				& Hit-2         					& 437 (19\%)\\
				& Hit-3         				  	& 159 (7\%) \\
				\hline
				\multirow{4}{*}{4 (1028)}  & Hit-1    & 268 (26\%) \\
				& Hit-2         					& 211 (21\%)\\
				& Hit-3         				  	& 94 (9\%) \\
				& Hit-4         				  	& 33 (3\%) \\
				\hline
			\end{tabular}
			ET: Exception types
			\label{RQ4_results_5}
		\end{center}
	}
\end{table}