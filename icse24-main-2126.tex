%%
%% This is file `sample-sigconf.tex',
%% generated with the docstrip utility.
%%
%% The original source files were:
%%
%% samples.dtx  (with options: `sigconf')
%% 
%% IMPORTANT NOTICE:
%% 
%% For the copyright see the source file.
%% 
%% Any modified versions of this file must be renamed
%% with new filenames distinct from sample-sigconf.tex.
%% 
%% For distribution of the original source see the terms
%% for copying and modification in the file samples.dtx.
%% 
%% This generated file may be distributed as long as the
%% original source files, as listed above, are part of the
%% same distribution. (The sources need not necessarily be
%% in the same archive or directory.)
%%
%%
%% Commands for TeXCount
%TC:macro \cite [option:text,text]
%TC:macro \citep [option:text,text]
%TC:macro \citet [option:text,text]
%TC:envir table 0 1
%TC:envir table* 0 1
%TC:envir tabular [ignore] word
%TC:envir displaymath 0 word
%TC:envir math 0 word
%TC:envir comment 0 0
%%
%%
%% The first command in your LaTeX source must be the \documentclass
%% command.
%%
%% For submission and review of your manuscript please change the
%% command to \documentclass[manuscript, screen, review]{acmart}.
%%
%% When submitting camera ready or to TAPS, please change the command
%% to \documentclass[sigconf]{acmart} or whichever template is required
%% for your publication.
%%
%%
\documentclass[sigconf]{acmart}

\acmConference[ICSE 2024]{46th International Conference on Software Engineering}{April 2024}{Lisbon, Portugal}

%%
%% \BibTeX command to typeset BibTeX logo in the docs
\AtBeginDocument{%
  \providecommand\BibTeX{{%
    Bib\TeX}}}

\usepackage{booktabs}   %% For formal tables:
                        %% http://ctan.org/pkg/booktabs
\usepackage{subcaption} %% For complex figures with subfigures/subcaptions
                        %% http://ctan.org/pkg/subcaption
\usepackage{array}
\usepackage{amsmath,amsfonts}
\usepackage{algorithm}
\usepackage[noend]{algpseudocode}
%\usepackage{algorithmic}
\usepackage{graphicx}
\usepackage{textcomp}
\usepackage{float}
\usepackage{listings}
\usepackage{xspace}
\usepackage{multirow}
\usepackage{amsthm}
\newtheorem{definition}{Definition}
\usepackage{balance}

\DeclareRobustCommand{\bbone}{\text{\usefont{U}{bbold}{m}{n}1}}

\DeclareMathOperator{\EX}{\mathbb{E}}% expected value

\usepackage[skins]{tcolorbox}

\usepackage{xcolor,pifont}
\newcommand*\colourcheck[1]{%
	\expandafter\newcommand\csname #1check\endcsname{\textcolor{#1}{\ding{52}}}%
}
\colourcheck{blue}
\colourcheck{green}
\colourcheck{red}

\newtcolorbox{myframe}[2][]{%
  enhanced,colback=white,colframe=black,coltitle=black,
  sharp corners,
  toprule=1.0pt,
  rightrule=0.3pt,
  leftrule=0pt,
  bottomrule=0pt,
  fonttitle=\itshape\scshape\large,
  left=0pt,right=5pt,top=5pt,bottom=3pt,
  attach boxed title to top right={yshift=-0.3\baselineskip-0.4pt,xshift=-5mm},
  boxed title style={tile,size=minimal,left=0.2mm,right=0.5mm,
    colback=white,before upper=\strut},
  title=#2,#1
}

\newcommand{\tool}{\textsc{Neurex}\xspace}

\newcommand{\xblock}{\textsc{XBlock}\xspace}

\newcommand{\xstate}{\textsc{XState}\xspace}

\newcommand{\xtype}{\textsc{XType}\xspace}

\newtheorem{Definition}{Definition}
\newtheorem{Claim}{Claim}
\newtheorem{Lemma}{Lemma}
\newtheorem{Theorem}{Theorem}

\newcolumntype{L}[1]{>{\raggedright\arraybackslash}p{#1}}
\newtheorem{Observation}{Observation}
\newtheorem{property}{Property}
\newcommand{\code}[1]{{\footnotesize\texttt{#1}}}
\usepackage{amsthm}
 \definecolor{dkgreen}{rgb}{0,0.6,0}
\definecolor{gray}{rgb}{0.5,0.5,0.5}
\definecolor{mauve}{rgb}{0.58,0,0.82}
\lstset{frame=tb,
  language=Java,
  aboveskip=3mm,
  belowskip=3mm,
  showstringspaces=false,
  columns=flexible,
  basicstyle={\small\ttfamily},
  numbers=left,
  numberstyle=\tiny\color{gray},
  keywordstyle=\color{blue},
  commentstyle=\color{dkgreen},
  stringstyle=\color{mauve},
  breaklines=true,
  breakatwhitespace=true,
  tabsize=4
}

\newcommand{\cf}{\hbox{\emph{cf.}}\xspace}
\newcommand{\deletia}{\ldots [deletia] \ldots}
\newcommand{\etal}{\hbox{\emph{et al.}}\xspace}
\newcommand{\eg}{\hbox{\emph{e.g.,}}\xspace}
\newcommand{\ie}{\hbox{\emph{i.e.,}}\xspace}
\newcommand{\st}{\hbox{\emph{s.t.}}\xspace}
\newcommand{\wrt}{\hbox{\emph{w.r.t.}}\xspace}
\newcommand{\viz}{\hbox{\emph{viz.}}\xspace}

\definecolor{lightgray}{rgb}{.9,.9,.9}
\definecolor{darkgray}{rgb}{.4,.4,.4}
\definecolor{purple}{rgb}{0.65, 0.12, 0.82}

%FIXME
\copyrightyear{2024}
\acmYear{2024}
\setcopyright{rightsretained}
\acmConference[ICSE '24]{2024 IEEE/ACM 46th International Conference on Software Engineering}{April 14--20, 2024}{Lisbon, Portugal}
\acmBooktitle{2024 IEEE/ACM 46th International Conference on Software Engineering (ICSE '24), April 14--20, 2024, Lisbon, Portugal}\acmDOI{10.1145/3597503.3639188}
\acmISBN{979-8-4007-0217-4/24/04}

%\setcopyright{acmlicensed}
%\acmPrice{}
%\acmDOI{10.1145/3611643.3616368}
%\acmYear{2023}
%\copyrightyear{2023}
%\acmSubmissionID{icse24-main-2126}
%\acmISBN{979-8-4007-0327-0/23/12}
%\acmConference[ICSE '24]{Proceedings of the 46th International Conference on Software Engineering}{April 14--20, 2024}{Lisbon, Portugal}
%\acmBooktitle{Proceedings of the 46th International Conference on Software Engineering (ICSE '24), April 14--20, 2024, Lisbon, Portugal}
%\received{2023-02-02}
%\received[accepted]{2023-07-27}


%% Rights management information.  This information is sent to you
%% when you complete the rights form.  These commands have SAMPLE
%% values in them; it is your responsibility as an author to replace
%% the commands and values with those provided to you when you
%% complete the rights form.

%Here
%\setcopyright{acmlicensed}
%\copyrightyear{2018}
%\acmYear{2018}
%\acmDOI{XXXXXXX.XXXXXXX}

%% These commands are for a PROCEEDINGS abstract or paper.

%Here
%\acmConference[Conference acronym 'XX]{Make sure to enter the correct
%  conference title from your rights confirmation emai}{June 03--05,
%  2018}{Woodstock, NY}

%%
%%  Uncomment \acmBooktitle if the title of the proceedings is different
%%  from ``Proceedings of ...''!
%%
%%\acmBooktitle{Woodstock '18: ACM Symposium on Neural Gaze Detection,
%%  June 03--05, 2018, Woodstock, NY}

%Here
%\acmISBN{978-1-4503-XXXX-X/18/06}




%%
%% Submission ID.
%% Use this when submitting an article to a sponsored event. You'll
%% receive a unique submission ID from the organizers
%% of the event, and this ID should be used as the parameter to this command.
%%\acmSubmissionID{123-A56-BU3}

%%
%% For managing citations, it is recommended to use bibliography
%% files in BibTeX format.
%%
%% You can then either use BibTeX with the ACM-Reference-Format style,
%% or BibLaTeX with the acmnumeric or acmauthoryear sytles, that include
%% support for advanced citation of software artefact from the
%% biblatex-software package, also separately available on CTAN.
%%
%% Look at the sample-*-biblatex.tex files for templates showcasing
%% the biblatex styles.
%%

%%
%% The majority of ACM publications use numbered citations and
%% references.  The command \citestyle{authoryear} switches to the
%% "author year" style.
%%
%% If you are preparing content for an event
%% sponsored by ACM SIGGRAPH, you must use the "author year" style of
%% citations and references.
%% Uncommenting
%% the next command will enable that style.
%%\citestyle{acmauthoryear}


%%
%% end of the preamble, start of the body of the document source.
\begin{document}

%%
%% The "title" command has an optional parameter,
%% allowing the author to define a "short title" to be used in page headers.
%\title{The Name of the Title Is Hope}

\title[Programming Assistant for Exception Handling with CodeBERT]{Programming Assistant for Exception Handling with CodeBERT}

%%
%% The "author" command and its associated commands are used to define
%% the authors and their affiliations.
%% Of note is the shared affiliation of the first two authors, and the
%% "authornote" and "authornotemark" commands
%% used to denote shared contribution to the research.

\author{Yuchen Cai}
\affiliation{
  \institution{University of Texas at Dallas}
  \state{Texas}
  \country{USA}
}
\email{yuchen.cai@utdallas.edu}
\author{Aashish Yadavally}
\affiliation{
  \institution{University of Texas at Dallas}
  \state{Texas}
  \country{USA}
}
\email{aashish.yadavally@utdallas.edu}
\author{Abhishek Mishra}
\affiliation{
  \institution{University of Texas at Dallas}
  \state{Texas}
  \country{USA}
}
\email{abhishek.mishra@utdallas.edu}
\author{Genesis Montejo}
\affiliation{
  \institution{University of Texas at Dallas}
  \state{Texas}
  \country{USA}
}
\email{genesis.montejo@utdallas.edu}
\author{Tien N. Nguyen}
\affiliation{
  \institution{University of Texas at Dallas}
  \state{Texas}
  \country{USA}
}
\email{tien.n.nguyen@utdallas.edu}

%\author{Ben Trovato}
%\authornote{Both authors contributed equally to this research.}
%\email{trovato@corporation.com}
%\orcid{1234-5678-9012}
%\author{G.K.M. Tobin}
%\authornotemark[1]
%\email{webmaster@marysville-ohio.com}
%\affiliation{%
%  \institution{Institute for Clarity in Documentation}
%  \streetaddress{P.O. Box 1212}
%  \city{Dublin}
%  \state{Ohio}
%  \country{USA}
%  \postcode{43017-6221}
%}

%\author{Lars Th{\o}rv{\"a}ld}
%\affiliation{%
%  \institution{The Th{\o}rv{\"a}ld Group}
%  \streetaddress{1 Th{\o}rv{\"a}ld Circle}
%  \city{Hekla}
%  \country{Iceland}}
%\email{larst@affiliation.org}

%\author{Valerie B\'eranger}
%\affiliation{%
%  \institution{Inria Paris-Rocquencourt}
%  \city{Rocquencourt}
%  \country{France}
%}

%\author{Aparna Patel}
%\affiliation{%
% \institution{Rajiv Gandhi University}
% \streetaddress{Rono-Hills}
% \city{Doimukh}
% \state{Arunachal Pradesh}
% \country{India}}

%\author{Huifen Chan}
%\affiliation{%
%  \institution{Tsinghua University}
%  \streetaddress{30 Shuangqing Rd}
%  \city{Haidian Qu}
%  \state{Beijing Shi}
%  \country{China}}

%\author{Charles Palmer}
%\affiliation{%
%  \institution{Palmer Research Laboratories}
%  \streetaddress{8600 Datapoint Drive}
%  \city{San Antonio}
%  \state{Texas}
%  \country{USA}
%  \postcode{78229}}
%\email{cpalmer@prl.com}

%\author{John Smith}
%\affiliation{%
%  \institution{The Th{\o}rv{\"a}ld Group}
%  \streetaddress{1 Th{\o}rv{\"a}ld Circle}
%  \city{Hekla}
%  \country{Iceland}}
%\email{jsmith@affiliation.org}

%\author{Julius P. Kumquat}
%\affiliation{%
%  \institution{The Kumquat Consortium}
%  \city{New York}
%  \country{USA}}
%\email{jpkumquat@consortium.net}

%%
%% By default, the full list of authors will be used in the page
%% headers. Often, this list is too long, and will overlap
%% other information printed in the page headers. This command allows
%% the author to define a more concise list
%% of authors' names for this purpose.
\renewcommand{\shortauthors}{Yuchen Cai, Aashish Yadavally, Abhishek Mishra, Genesis Montejo, and Tien N. Nguyen}

%%
%% The abstract is a short summary of the work to be presented in the
%% article.
\begin{abstract}
With practical code reuse, the code fragments from developers' forums
often migrate to applications. Owing to the incomplete nature of such
fragments, they often lack the details on exception handling. The
adaptation for exception handling to the codebase is not trivial as
developers must learn and memorize what API methods could cause
exceptions and what exceptions need to be handled. We propose {\tool},
an exception handling recommender that learns from complete code, and
accepts a given Java code snippet and recommends 1) if a
\code{try-catch} block is needed, 2) what statements need to be placed
in a \code{try} block, and 3) what exception types need to be
caught in the \code{catch} clause.
%We overcome the limitations of the state-of-the-art information
%retrieval approaches by our learning exception handling from complete
%code.
Inspired by the sequence chunking techniques in natural language
processing, we design {\tool} via a multi-tasking model with the
fine-tuning of the large language model CodeBERT for the three above
exception handling recommending tasks. Via the large language model,
{\tool} can learn the surrounding context, leading to better
learning the dependencies among the API elements, and the relations
between the statements and the corresponding exception types needed to
be~handled.

Our empirical evaluation shows that {\tool} correctly performs all
three exception handling recommendation tasks in 71.5\% of the cases
with a F1-score of 70.2\%, which is a relative improvement of 166\% over the
baseline.
%with an F1-score of {\color{red}{\bf XX.X\%}} (an relative
%improvement of up to {\color{red}{\bf XX.X\%}} over the baselines).
It achieves high F1-score from 98.2\%--99.7\% in \code{try-catch}
block necessity checking (a relative improvement of up to 55.9\% over
the baselines). It also correctly decides both the need for
\code{try-catch} block(s) and the statements to be placed in \code{try}
blocks with the F1-scores of 74.7\% and 87.1\% at the instance and
statement levels, an improvement of 129.1\% and 44.9\% over the
baseline, respectively. Our extrinsic evaluation shows that {\tool}
relatively improves over the baseline by 56.5\% in F1-score for
detecting exception-related bugs in incomplete Android code
snippets.
\end{abstract}

%%
%% The code below is generated by the tool at http://dl.acm.org/ccs.cfm.
%% Please copy and paste the code instead of the example below.
%%
\begin{CCSXML}
<ccs2012>
<concept>
<concept_id>10010147.10010257.10010293.10010294</concept_id>
<concept_desc>Computing methodologies~Neural networks</concept_desc>
<concept_significance>500</concept_significance>
</concept>
<concept_id>10011007.10011006.10011008.10011024</concept_id>
<concept_desc>Software and its engineering~Language features</concept_desc>
<concept_significance>500</concept_significance>
</concept>
</ccs2012>
\end{CCSXML}

\ccsdesc[500]{Computing methodologies~Neural networks}
\ccsdesc[500]{Software and its engineering~Language features}
%\ccsdesc[500]{Software and its engineering~Language types}

%%
%% Keywords. The author(s) should pick words that accurately describe
%% the work being presented. Separate the keywords with commas.
\keywords{AI4SE, Large Language Models, Automated Exception Handling}


%% A "teaser" image appears between the author and affiliation
%% information and the body of the document, and typically spans the
%% page.
%\begin{teaserfigure}
%  \includegraphics[width=\textwidth]{sampleteaser}
%  \caption{Seattle Mariners at Spring Training, 2010.}
%  \Description{Enjoying the baseball game from the third-base
%  seats. Ichiro Suzuki preparing to bat.}
%  \label{fig:teaser}
%\end{teaserfigure}

%\received{20 February 2007}
%\received[revised]{12 March 2009}
%\received[accepted]{5 June 2009}

%%
%% This command processes the author and affiliation and title
%% information and builds the first part of the formatted document.
\maketitle

\section{Introduction}
\label{sec:intro}

The online question and answering (Q\&A) forums, e.g., StackOverflow
(S/O) provide important resources for developers to learn how to use
software libraries and frameworks. While the code snippets in an S/O
answer are good starting points, they are often incomplete with
several missing details, even with ambiguous references, etc.  Zhang
{\em et al.}~\cite{zhang-icse19} have conducted a large-scale
empirical study on the nature and extent of manual adaptations of the
S/O code snippets by developers into their GitHub repositories.  They
reported that the adaptations from S/O code examples to their GitHub
counterpart projects are prevalent. They qualitatively inspected all
the adaptation cases and classified them into 24 different adaptation
types. They highlighted several adaptation types including {\em type
  conversion}, {\em handling potential exceptions}, and {\em adding if
  checks}~\cite{zhang-icse19}. Among them, adding a \code{try-catch}
block to wrap the code snippet and listing the handled exceptions in
the \code{catch} clause are frequently performed,
%(132 cases out of 629 S/O examples),
yet not automated by existing tools.
%code integration techniques.

The adaptation process for exception handling is not trivial as Nguyen
{\em et al.}~\cite{xrank-fse20} have reported that it is challenging
for developers to learn and memorize what API methods could cause
exceptions and what exceptions need to be handled. Kechagia {\em et
  al.}~\cite{kechagia-msr14} found that 19\% of the crashes in Android
applications could have been caused by insufficient documented
exceptions in Android APIs. Thus, it is desirable to have an automated
tool to recommend proper exception handling for the adaptation of
online code snippets.

There exist several approaches to automatic recommendation of
exception
handling~\cite{barbosa-bsse12,chanchal-scam14,barbosa-tse18,barbosa-tse16,xrank-fse20,throw-ase22}. They
can be classified into four categories. The first category of
approaches relies on a few {\em program analysis heuristics} on
exception types, API calls, and variable types to recommend exception
handling code~\cite{barbosa-bsse12}. These heuristic-based approaches
do not always work in all cases due to incomplete code. The second
category utilized {\em exception handling policies},
which are enforced in all
cases~\cite{barbosa-tse16,barbosa-saner18}. However, the policies need
to be pre-defined and encoded within the recommending tools.  This is
not an ideal solution considering the fast evolution of software
libraries. To enable more flexibility than policy enforcement, the
third category leverages {\em mining algorithms} that derive similar
exception handling for two similar code
fragments~\cite{chanchal-scam14}. While avoiding hard-coding of the
rules, these mining approaches suffer the issue of how much similar
for two fragments to be considered as having similar exception
handling. For the mining approaches, deterministically setting a
threshold for frequent occurrences is also challenging.
%In fact, two API method calls in two different contexts might require
%to handle the same exception. For example, two fragments of code
%using JDK containing an opening of the files for different purposes
%in two contexts might need to catch and handle the \code{IOException}
%due to the call to \code{java.io.nio.new\-Buffered\-Reader}.

To provide more flexibility in code matching, the fourth category
follows {\em information retrieval}
(IR)~\cite{xrank-fse20}. XRank~\cite{xrank-fse20} takes as input
source code and recommends a ranked list of API method calls in the
code that are potentially involved in the exceptions in a \code{catch-try}
block. XHand~\cite{xrank-fse20} recommends the exception handling
code in a \code{catch} block for a given code. Both use a fuzzy set
technique to compute the associations between the API calls (e.g.,
\code{new\-Buffered\-Reader}) and the exceptions (e.g.,
\code{IOException}).

While the IR-based approach achieves higher accuracy than the
others~\cite{xrank-fse20}, it has key limitations. First, it is not
trivial to {\bf pre-define a threshold} for feature matching for a
retrieval of an exception type or an API element. The effectiveness of
those IR techniques depends much on the correct value of such
pre-defined threshold. Second, the IR-based techniques rely on the
lexical values of the code tokens and API elements, whose names can be
{\em ambiguous} in an incomplete code snippet. For example, the
\code{Document} class in \code{org.\-w3c.\-dom} of the W3C library has
the same simple name as the \code{Document} class in
\code{com.\-google.\-gwt.\-dom.\-client.\-Document} of Google Web
Toolkit library (GWT). An API method to open/write/read a
\code{Document} in the W3C library might need to catch a different set
of exceptions than the one in GWT. Those IR-based techniques are not
sufficiently flexible to handle such {\bf ambiguous names}. Third, the
IR techniques {\em do not consider the {\bf context} of surrounding
  code}, thus, cannot leverage the {\em dependencies} among API
elements to resolve the ambiguity of the names of the APIs and
  exceptions in an incomplete snippet.

In this paper, we propose {\tool}, a learning-based exception handling
recommender, which accepts a given Java code snippet and recommends 1)
{\em whether a try-catch block is needed for the snippet} ({\xblock}),
2) {\em what statements need to be placed in a \code{try-catch} block}
({\xstate}) and 3) {\em what exception types need to be caught in the
  \code{catch} clause} ({\xtype}).  We find a motivation for such a
data-driven, learning-based approach from the previous studies
reporting that exception handling for the API elements is frequently
repeated across different
projects~\cite{chanchal-scam14,zhong-jss18}. The reason for such
repetitions is that the designers of a software library have the
intents for users to use certain API elements with corresponding
exception types. Thus, we design {\tool} to learn from the statements
in \code{try-catch} blocks and exception types retrieved from {\em
  complete source code} in a large code corpus, and derive suggestions
for given {\em (in)complete~code}.
%the above exception handling suggestions for the {\em (partial) code
%  snippet} under study.

We leverage and fine-tune the large language model
CodeBERT~\cite{codebert-emnlp20} to capture the surrounding code {\bf context}
with the dependencies among the API elements. Capturing such
contextual information and the dependencies enable {\tool} to realize
the idea {\em ``Tell Me Your Friends, I'll Tell You Who You Are''} to
learn the {\em {\bf dependencies} among statements with APIs} in a given
(in)complete code, leading to better learning in {\xblock}, {\xstate}
and {\xtype}. Inspired by sequence chunking in natural language
processing (NLP), we formulate our problem as detecting one or
multiple chunks of consecutive statements that need \code{try-catch}
blocks.
%Tien
{\tool} also has the three tasks in a {\bf multi-tasking} mechanism
to enable the mutual impact among the learning, leading to
better performance in all three tasks.

%The context also enables {\tool} learn to connect those statements
%with the corresponding exception types.
%that need to be caught in the \code{catch} clauses.

Our above idea gives {\tool} advantages over the state-of-the-art IR
approach. First,
%with the learning-based approach,
via learning, {\tool} does not rely on a pre-defined threshold for
explicit feature matching to retrieve API elements or
exceptions. Second, instead of using {\em the associations between
  pairs of an API and an exception type as in XRank},
%instead of learning only the associations between an API element and
%an exception type,
{\tool} relies on {\bf dependencies} and {\bf contexts}
%has advantages
in both predicting and training. During training, the complete code
provides the context for {\tool} to learn the dependencies among the
API elements and the exception types. During predicting, for a given
(in)complete code, {\tool} can implicitly match the current context
with such knowledge to avoid the name ambiguity and to derive the
exception handling suggestions.
%During predicting, for a given incomplete code, the surrounding code
%enables {\tool} to {\em learn the identities of the API elements via
%  the dependencies/relations} among them in the context, thus,
%avoiding the name ambiguity. Let us call it {\em dependency
%  context}. During training, the complete code enables the
%identifications (i.e., the fully-qualified names) of the API
%elements.
%
For example, encountering the program dependencies among the API
elements and exceptions in JDK, e.g., \code{get\-Declared\-Field} in
\code{java.\-lang.\-Class}, \code{get} in
\code{java.\-lang.\-Class.\-Field}, and
\code{No\-Such\-Field\-Exception} in the training data will help
     {\tool} learn to match the context in a given code to suggest
     \code{NoSuchFieldException}.

%Third, the context of surrounding code also helps the model implicitly
%learn the important features to connect between the API elements and
%the corresponding exception types.

%{\tool} is designed to capture the basic insights to overcome key
%limitations of the state-of-the-art approaches. First, with the
%learning-based approach, {\tool} does not rely on a pre-defined
%threshold for explicit feature matching for the retrieval of the API
%elements or exception types. Second, instead of learning only the
%associations between the API elements and corresponding exception
%types, we also consider the code context surrounding them. This gives
%{\tool} two advantages. During predicting, for a given incomplete
%code, the context enables {\tool} to {\em learn the identities of the
%  API elements via the dependencies/relations} among them in the
%context, thus, avoiding the name ambiguity. Let us call it {\em
%  dependency context}. During training, the complete code enables the
%identifications (i.e., the fully-qualified names) of the API
%elements. Moreover, the context of surrounding code also helps the
%model implicitly learn the important features to connect between the
%API elements and the corresponding exception types.

%We leverage Relational Graph Convolutional Network (R-GCN)~\cite{rgcn}
%to represent the program dependence graph (PDG) to capture the control
%and data dependencies among the API elements in the surrounding
%context. Learning the dependencies enables us to realize the key idea
%{\em ``Tell Me Your Friends, I'll Tell You Who You Are''} to learn the
%identities of the API elements in a given (in)complete code, leading
%to better learning in {\xblock}, {\xstate} and {\xtype}.
%
%Moreover, we treat the problem of predicting a \code{try-catch} block
%as a classification (Y/N) as well as the problem of deriving the
%handled exception types as a classification among exceptions. To
%predict which statements in a given code snippet that need to be
%wrapped in a \code{try-catch} block, we leverage the explainable ML
%model, GNNExplainer~\cite{GNNExplainer}, that {\em ``explains'' on
%why the GCN model has arrived at its decision}. Specifically, {\tool}
%takes two inputs: 1) the input PDG $G_C$ of the given code $C$, 2)
%the trained GCN model along with its decision on a \code{try-catch}
%block being needed for $C$. GNNExplainer will learn the explanation
%subgraph, which is defined as a minimal sub-graph $\mathcal{G}$ in
%the PDG of $C$ that {\em minimizes the prediction~scores between
%using the entire $G_C$ and using $\mathcal{G}$}. The idea is that if
%an~edge is {\em removed} from $G_C$, and {\em the decision of the
%model is affected, the edge is crucial and must be included in the
%explanation for the detection result}. Thus, the minimal sub-graph
%$\mathcal{G}$ in PDG contains the nodes and edges, {\em i.e.},
%including the {\em crucial statements} that are most
%decisive/relevant to the classification on exception blocks. We
%consider them as the statements to be placed in a
%\code{try-catch}~block.

%=================================================
We conducted several experiments to evaluate {\tool} with
a large dataset of 5,726 projects from GitHub
having 246,118 code snippets for training, 30,764 for validation,
and 30,764 for testing.
%30,764 code snippets (half of them have \code{try-catch} blocks).
Our empirical evaluation shows that {\tool} correctly performs all
three exception handling recommendation tasks in 71.5\% of the cases
with an F1-score of 70.2\%. It has an relative improvement of 166\%
over the baseline. It achieves high F1-score from 98.2\%--99.7\% in
\code{try-catch} block necessity checking (an relative improvement of
up to 55.9\% over the baselines). It also correctly decides both the
need of \code{try-catch} block(s) and the statements to be placed in
such blocks with the F1-scores of 74.7\% and 87.1\% at the instance
and statement levels, an improvement of 127.3\% and 44.9\% over the
baseline, respectively. Our extrinsic evaluation shows that {\tool}
relatively improves over the baseline by 56.5\% in F1-score in
exception-related bug detection in {\em incomplete} Android code
snippets.

%The result shows that {\tool} achieves a high accuracy of {\bf 98.3\%}
%and improves relatively by {\bf 56\%} over the state-of-the-art
%approach in \code{try-catch} block necessity checking. Moreover, it
%can correctly decide both the need of \code{try-catch} block(s) and
%the statements to be placed in such blocks in {\bf 79.4\%} of the
%cases, an improvement of {\bf 62X} over the baseline. Importantly,
%with an accuracy of {\bf 71.5\%} (an improvement of {\bf 146X} over
%the baseline), {\tool} correctly recommend exception handling in all
%three tasks. Our extrinsic evaluation also shows that with its
%recommendations, {\tool} improves by {\bf YY.Y\%} in F-score over the
%existing approach in detecting exception-related bugs.
%====================================================

%Our empirical evaluation shows that {\tool} relatively
%improves 12.3\% in F-score over the state-of-the-art approach,
%XRank~\cite{xrank-fse20} in \code{try-catch} block necessity checking.
%{\tool} also achieves a high accuracy of 74\% in recommending the
%statements to be placed in a \code{try-catch} block. It can cover all
%the needed exception types in 63\% of the cases and predict correctly
%all exception types in 33\% of the cases. Our extrinsic evaluation
%also shows that {\tool} improves 9.8\% in F-score over
%FuzzyCatch~\cite{xrank-fse20} in detecting exception-related~bugs.

%Our empirical evaluation shows that {\tool} achieves a high accuracy
%of XX\% in predicting if a code snippet needs a \code{try-catch}
%block. It also achieves a high accuracy of XX\% in recommending which
%statements in the snippet to be placed in a try-catch block. Our
%result shows that {\xtype} relatively improves over the
%state-of-the-art IR approach in exception type recommendation from
%XX\%–YY\% in accuracy.

In brief, this paper makes the following major contributions:

\noindent {\bf 1. [Neural Network-based Automated Exception Handling
    Recommendation]}. {\tool} is the first neural network approach to
automated exception handling recommendation in three above tasks.
{\tool} works for either complete or {\em incomplete} code.

\noindent {\bf 2. [Multi-tasking among three Exception Handling
    Recommendations]} We formulate the problem as sequence chunking
with a multi-tasking mechanism to learn for three above tasks.

%  \noindent {\bf 2. [Graph Neural Network and Explainable AI (XAI)]} We
%  leverage R-GCN to capture the program
%  dependencies among API elements and exceptions and leverage XAI
%  to derive the statements to be placed in a \code{try-catch} block.

  \noindent {\bf 3. [Empirical Evaluation]}. Our evaluation
  shows {\tool}'s high accuracy in exception handling recommendation
  as well as in exception-related bug detection. Data and code is
  available at~\cite{neurex-website}.


%for exception blocks, statements, and types.




\section{Motivation}
\label{motiv:sec}

\subsection{Motivating Examples}
\label{examples:sec}

\begin{figure}[t]%[htbp]
	\centering
	\lstset{
		numbers=left,
		numberstyle= \tiny,
		keywordstyle= \color{blue!70},
		commentstyle= \color{red!50!green!50!blue!50},
		frame=shadowbox,
		rulesepcolor= \color{red!20!green!20!blue!20} ,
		xleftmargin=1.5em,xrightmargin=0em, aboveskip=1em,
		framexleftmargin=1.5em,
                numbersep= 5pt,
		language=C,
    basicstyle=\scriptsize\ttfamily,
    numberstyle=\scriptsize\ttfamily,
    emphstyle=\bfseries,
                moredelim=**[is][\color{red}]{@}{@},
		escapeinside= {(*@}{@*)}
	}
\begin{lstlisting}[]
public static void addLibraryPath(String pathToAdd) throws Exception {
  final Field usrPathsField = ClassLoader.class.(*@{\color{orange}{getDeclaredField("usr\_paths");}@*)
  usrPathsField.setAccessible(true);

  //get array of paths
  final String[] paths = (String[])usrPathsField.get(null);

  //check if the path to add is already present
  for(String path : paths) {
      if(path.equals(pathToAdd)) {
          return;
      }
  }

  //add the new path
  final String[] newPaths = Arrays.copyOf(paths, paths.length + 1);
  newPaths[newPaths.length-1] = pathToAdd;
  usrPathsField.set(null, newPaths);
}
\end{lstlisting}
        \vspace{-16pt}
        \caption{StackOverflow post \#15409223 on adding new paths for
          native libraries at runtime in Java}
        \label{fig:example1}
\end{figure}


Let us use a few real-world examples to explain the problem and
motivate our approach. Figure~\ref{fig:example1} displays a code
snippet in an answer to the StackOverflow (SO) question 15409223 on
how to ``add new paths for native libraries at runtime in Java''.  The
code snippet serves the illustrating purpose, thus, does not contain
all the details regarding what exceptions that need to be handled. It
just contains a throw of a generic \code{Exception} in the method
header (\code{addLibraryPath}). From the empirical study from Zhang
{\em et al.}~\cite{zhang-icse19}, this code snippet was adopted by the
developers into their Github project named \code{armint} as seen in
Figure~\ref{fig:example2}. In \code{armint} code, the developers
handle in a \code{try-catch} block the all possible exceptions caused
by \code{java\-.lang\-.Class\-.get\-Declared\-Field(...)} (line 7)
according to JDK's documentation including
\code{No\-Such\-Field\-Exception}, \code{Security\-Exception},
\code{Illegal\-Argument\-Exception}, and
\code{Illegal\-Access\-Exception} (line 24 of
Figure~\ref{fig:example2}).

In their study, Zhang {\em et al.}~\cite{zhang-icse19} have reported
that such adaptation is still largely manual due to the lack of
documentation for exception handling. Kechagia {\em et
  al.}~\cite{kechagia-msr14} found that 69\% of the methods from
Android packages in their stack traces had undocumented exceptions in
the Android API. The manual adaptation on exception handling by
inserting a \code{try-catch} block is quite popular (in 31 out 629
cases). Such manual adaptation could lead to exception-related bugs,
which could cause serious issues including crashes or unstable
states~\cite{xrank-fse20}. Thus, it is desirable to have an automated
tool to recommend proper exception handling in order to adapt the
incomplete code snippets. For example, what lines need to be included
in a \code{try-catch} block, and what exceptions need to be handled.

%spec, heuristic, mining, IR

To address the issue, several automated approaches have been proposed
to recommend proper exception
handling~\cite{xrank-fse20,barbosa-bsse12,chanchal-scam14,barbosa-tse18,barbosa-tse16}. Earlier
approaches utilized {\em heuristic strategies} on exception types,
method calls, and variables' types to recommend exception handling
code~\cite{barbosa-bsse12}. While the heuristics might not work in all
cases, {\em exception handling policies} could be
enforced~\cite{barbosa-tse16,barbosa-saner18}. However, the policies
must be pre-defined and encoded in the tool. To overcome that, the
{\em mining approaches} are based on the idea that similar code
requires similar exception handling~\cite{chanchal-scam14}. Some
approaches also leverage mined patterns to repair exception-handling
bugs~\cite{zhong-jss18}. A key issue with the mining approaches is
that the two code fragments might not be exactly matched, but they
share a few key elements that require the same exception handling.
Thus, Nguyen {\em et al.}~\cite{xrank-fse20} propose a {\em
  information retrieval} (IR) approach using a fuzzy set technique to
learn the associations between the method calls (e.g.,
\code{get\-Declared\-Field}) and the exceptions (e.g.,
\code{No\-Such\-Field\-Exception}, \code{Security\-Exception}, etc.)
that need to be handled. A key drawback of the IR approach is that it
relies on the threshold for a match in an retrieval, which might often
not be easily pre-defined.



\begin{figure}[t]%[htbp]
	\centering
	\lstset{
		numbers=left,
		numberstyle= \tiny,
		keywordstyle= \color{blue!70},
		commentstyle= \color{red!50!green!50!blue!50},
		frame=shadowbox,
		rulesepcolor= \color{red!20!green!20!blue!20} ,
		xleftmargin=1.5em,xrightmargin=0em, aboveskip=1em,
		framexleftmargin=1.5em,
                numbersep= 5pt,
		language=C,
    basicstyle=\scriptsize\ttfamily,
    numberstyle=\scriptsize\ttfamily,
    emphstyle=\bfseries,
                moredelim=**[is][\color{red}]{@}{@},
		escapeinside= {(*@}{@*)}
	}
\begin{lstlisting}[]
/** ...
 * taken from http://stackoverflow.com/questions/15409223/
 * adding-new-paths-for-native-libraries-at-runtime-in-java
 */
private static void addLibraryPath(String pathToAdd) {
  (*@{\color{orange}{try}@*) {
    final Field usrPathsField = ClassLoader.class.(*@{\color{orange}{getDeclaredField("usr\_paths");}@*)
    usrPathsField.setAccessible(true);

    // get array of paths
    final String[] paths = (String[]) usrPathsField.get(null);

    // check if the path to add is already present
    for (String path : paths) {
      if (path.equals(pathToAdd)) {
	return;
      }
    }

    // add the new path
    final String[] newPaths = Arrays.copyOf(paths, paths.length + 1);
    newPaths[newPaths.length - 1] = pathToAdd;
    usrPathsField.set(null, newPaths);
  } (*@{\color{orange}{catch (NoSuchFieldException | SecurityException | IllegalArgumentException |    IllegalAccessException e)}@*) {
	throw new RuntimeException(e);
  }
}
\end{lstlisting}
        \vspace{-16pt}
        \caption{GitHub project \code{armint} adapts SO post in Figure~\ref{fig:example1}}
        \label{fig:example2}
\end{figure}




\subsection{Observations}
\label{sec:obs}

To facilitate the reuse of code snippets in a forum, an automated tool
is needed to derive the fully-qualified names of the API elements in
the snippets so that the proper import statements are added in the
code. To build such a tool, we draw the motivation from the following
observations.

\vspace{2pt}
\noindent {\bf Observation 1} [{\em Regularity of API Usages}]. The
designers of software libraries have the intents for developers to use
certain API elements together (including API classes, method calls,
field accesses) in certain combinations and orders to achieve a
programming task. Those API elements do not occur randomly. For
example, in Figure~\ref{fig:example2} at line 2, in GWT, a variable of
the type \code{Button} (FQN:
\code{com.google.gwt.user.client.ui.Button}) is instantiated. Then, at
line 3, to set the handler of that GWT button, one needs to have a
method call to \code{addClickHandler} (FQN:
\code{com.google.gwt.user.client.ui.Button.add\-Click\-Handler}) on
the \code{Button} object with an argument of the type
\code{ClickHandler} (FQN:
\code{com\-.google\-.gwt\-.event\-.dom\-.client\-.ClickHandler}).  The
API elements are provided and intented to be used in such a code,
called {\em an API usage}. Thus, those API elements of API usages
frequently appear together in the client code using the
library. Figure~\ref{fig:example3} shows a complete example published
in the GWT tutorial website \code{gwtproject.org}. Providing all the
proper \code{import} statements, the author shows how to use the GUI
elements in GWT including \code{Button}. Specifically, at line 23,
despite using a different variable name \code{addStockButton}, the
method \code{addClickHandler} is called on a Button object (declared
at line 12) with the argument of the same type \code{ClickHandler}. In
brief, the source code in the public repositories could be a good
source for a model to implicitly learn the API usages to derive the
FQNs of the API elements in an incomplete snippet.

\begin{figure}[htbp]
	\centering
	\lstset{
		numbers=left,
		numberstyle= \tiny,
		keywordstyle= \color{blue!70},
		commentstyle= \color{red!50!green!50!blue!50},
		frame=shadowbox,
		rulesepcolor= \color{red!20!green!20!blue!20} ,
		xleftmargin=1.5em,xrightmargin=0em, aboveskip=1em,
		framexleftmargin=1.5em,
                numbersep= 5pt,
		language=C,
    basicstyle=\scriptsize\ttfamily,
    numberstyle=\scriptsize\ttfamily,
    emphstyle=\bfseries,
                moredelim=**[is][\color{red}]{@}{@},
		escapeinside= {(*@}{@*)}
	}
\begin{lstlisting}[]
public Object readField(Class<?> clazz, String name, Object instance) {
  (*@{\color{orange}{try}@*) {
    Field field = clazz.(*@{\color{orange}{getDeclaredField}@*)(name);
    if (!field.isAccessible()) {
       field.setAccessible(true);
    }
    return field.get(instance);
  } (*@{\color{orange}{catch (NoSuchFieldException | SecurityException | IllegalArgumentException | IllegalAccessException e)}@*) {
   throw new RuntimeException("Cannot read field value: " + clazz.getName() + "#" + name, e);
  }
}
\end{lstlisting}
        \vspace{-16pt}
        \caption{Project \code{quarkus} with same exception handling}
        \label{fig:example3}
\end{figure}


\vspace{2pt}
\noindent {\bf Observation 2} [{\em Dependencies/Relations among API
    Elements in a Usage}]. The dependencies/relations among the API
elements in a API usage can help a model better identify the FQNs of
the elements.  In Figure~\ref{fig:example3}, the API elements
\code{Button}, \code{addClickHandler}, and \code{ClickHandler} in GWT
have the program dependencies/relations among them. For example, in
GWT, to set a handler for a button, the object of the type
\code{Button} needs to be the {\em receiving object of the method
  call} to \code{addClickHandler}, which in turn needs to accept an
object of the type \code{ClickHandler} as an argument. These relations
exhibit in the client code using the GWT library, and are useful in
deciding the FQNs of its API elements. For example, in
Figure~\ref{fig:example2},~at line 3, if \code{addClickHandler} is
determined to be the API element
\code{com.\-google.\-gwt.\-user.\-client.\-ui.\-Button.\-add\-Click\-Handler}, the
FQN of the element at line 4 must be
\code{com\-.google\-.gwt\-.event\-.dom\-.client\-.ClickHandler}.  The
other direction of reasoning is applicable as well. In general, if a
model can learn the dependencies/relations among API elements, it
could leverage such knowledge to decide the FQNs of all those APIs at once.


%the lines 2 and 3 in Figure 2
As another example, the data dependency from the \code{def-use}
relation via the variable \code{myButton} between line 2 and line 3 in
Figure~\ref{fig:example2} is useful in deriving the FQNs of the
above API elements. If a model decides the FQN for \code{Button} at
line 2 as \code{com\-.google\-.gwt\-.user\-.client\-.ui\-.Button}, it
could derive the FQN of \code{add\-Click\-Handler} at line~3 as
\code{com.\-google.\-gwt.\-user.\-client.\-ui.\-Button.\-add\-Click\-Handler},
and vice versa.

\vspace{2pt}
\noindent {\bf State-of-the-Art Approaches.} Several approaches have
been proposed to automatically recover the fully-qualified names
(FQNs) for the API elements in a code snippet. The {\em
  program-analysis-based} approaches (e.g.,
PPA~\cite{dagenais-oopsla08}, RecoDoc~\cite{dagenais-icse12}), {\em
  information-retrieval-based} approaches (e.g.,
Baker~\cite{liveapi14}, COSTER~\cite{coster-ase19}), and {\em
  constraint-based} approaches (e.g., SnR~\cite{snr-icse22}) suffer
the out-of-vocabulary issue (i.e., could not derive the FQNs that were
not seen in the training corpus).

The advances in {\em Artificial Intelligence (AI)} and {\em Machine
  Learning (ML)} have enabled the generation of the new FQNs for the
APIs. However, those ML-based approaches (e.g., StatType~\cite{icse18}
and Huang {\em et al.}~\cite{prompt-ase22}) still do not leverage the
regularity of API usages and the dependencies/relations among relevant
API elements for FQN recovery. StatType~\cite{icse18} uses
phrase-based statistical machine translation from the code without
FQNs to the one with them.  With short phrases of the lengths of 3-8
tokens, it cannot capture the relevant API elements yet far apart. For
example, in Figure~\ref{fig:example1}, such short phrases in StatType
cannot make the connections between the API elements \code{mButton}
and \code{setOnClickListener} at line 11 to the relevant API elements
\code{mButton}, \code{findViewById}, etc. at line 8. In other cases,
the two relevant statements that could help the FQN recovery might be
even farther in the code. In contrast, Huang {\em et
  al.}~\cite{prompt-ase22} uses the context of a few lines surrounding
the prediction point (e.g., line 11) for their filling-in approach
with a masked language model. First, a few lines might not capture the
relevant API elements in the same usage. In Figure~\ref{fig:example1},
\code{mButton} at line 1 is far apart from \code{mButton} at line 8
and \code{mButton} at line 11. Second, each API element might be used
in a different context in the client code. For example, the code at
line 9 in Figure~\ref{fig:example1} is specific to the method
\code{onCreate} at line 4. Thus, this type of context might not help a
model learn the FQN of an API.

\begin{figure}[t]%[htbp]
	\centering
	\lstset{
		numbers=left,
		numberstyle= \tiny,
		keywordstyle= \color{blue!70},
		commentstyle= \color{red!50!green!50!blue!50},
		frame=shadowbox,
		rulesepcolor= \color{red!20!green!20!blue!20} ,
		xleftmargin=1.5em,xrightmargin=0em, aboveskip=1em,
		framexleftmargin=1.5em,
                numbersep= 5pt,
		language=C,
    basicstyle=\scriptsize\ttfamily,
    numberstyle=\scriptsize\ttfamily,
    emphstyle=\bfseries,
                moredelim=**[is][\color{red}]{@}{@},
		escapeinside= {(*@}{@*)}
	}
\begin{lstlisting}[]
Charset charset = Charset.forName("US-ASCII");
(*@{\color{orange}{try}@*) {
    BufferedReader reader = Files.(*@{\color{orange}{newBufferedReader}@*)(file, charset);
    String line = null;
    while ((line = reader.(*@{\color{orange}{readLine}@*)()) != null) {
        System.out.println(line);
    }
} (*@{\color{orange}{catch (IOException x)}@*) {
    System.err.format("IOException: %s%n", x);
}
\end{lstlisting}
        \vspace{-16pt}
        \caption{Using \code{newBufferedReader} to read from a file}
        \label{fig:example4}
\end{figure}

%Talk about StatType and fill-in


\subsection{Key Ideas}
\label{key:sec}

\noindent We design {\tool} for exception handling recommendation for
Java code: 1) given a code snippet, it will point out which statements
in the snippet need to be placed within a \code{try-catch} block, and
what exceptions need to be handled in the catch clause. Following
Observations 1--4, we design {\tool} with the following key ideas:

%\vspace{3pt}
\subsubsection{{\bf [Key Idea 1] Neural Network-Based Approach to Exception Handling Recommendation}}
%\vspace{2pt}
%\subsubsection*{{\bf [Key Idea 1] Neural Network-Based Approach to Partial Program Dependence Analysis}}
Instead of deterministically deriving the exceptions to be handled in
a heuristic or mining manner for a given (incomplete) code snippet, following
Observation 2, we design a deep learning model (DL) to learn to
analyze that snippet to recommend the statements in the
\code{try-catch} block and the exceptions in the \code{catch} clause.
By leveraging the \code{try-catch} blocks of the complete code in the
open-source projects (e.g., GitHub) in the training process, our DL
model can decide what statements in the given code snippet to be
placed in a \code{try-catch} block and what exceptions to be handled.


%\vspace{2pt}
\subsubsection{{\bf [Key Idea 2] Span-based ...}}
We seek inspiration from the neural network-based dependency parsing
approaches~\cite{?} in Natural Language Processing (NLP). They
successfully learn ... Following suit, we design \tool to learn the
representations for the statements in source code so as to learn the
span ...

\vspace{2pt}
\subsubsection{{\bf [Key Idea 3] Context ...}}
...


%\input{concepts}

\section{Approach Overview}
\label{sec:overview}


\begin{figure*}[t]
\begin{center}
\includegraphics[width=5.4in]{overview.png}
\vspace{-10pt}
\caption{{\tool}: Architecture Overview}
\label{overview}
\end{center}
\end{figure*}

Let us present the overview of our approach. In general, {\tool}
consists of three main components. The first component called
{\xblock} aims to check if it is neccessary to have a \code{try-catch}
block for the given code snippet $C$. We extract the program dependence
graph (PDG) from the source code using DeepPDA~\cite{icse23} that is
capable of generating the PDG for any (in)complete code
snippet. The PDG is used an input for a Relational Graph Convolutional
Network (R-GCN)~\cite{yi}. During training, we use the complete source
code in the open-source projects in which each positive sample
contains at least a \code{try-catch} block, and each negative one does
not. If there are multiple consecutive blocks, we split them into
individual ones. During prediction, {\xblock} uses the trained R-GCN
to predict whether the given source code needs a \code{try-catch}
block or not.

In the case of the code snippet $C$ needs a \code{try-catch} block,
the second component, {\xstate}, is aimed to detect which statements
in $C$ that need to be placed within the \code{try-catch} block.
{\xstate} takes two inputs: 1) the PDG of the given code snippet, and
2) the trained R-GCN model for the \code{try-catch} necessity checker
({\xblock}). {\xstate} leverages the Graph-based Explainable AI model
named GNNExplainer~\cite{tien} to determine the statements in the PDG
that are most decisive and relevant to the reason why the snippet $C$
is required to have a \code{try-catch} block. The output of this
component is a list of statements in $C$ (e.g., $S_3$, $S_4$, $S_5$).
During training, ...


%\vspace{-2pt}
\section{Graph-based Try-Catch Necessity Checking Model}
\label{detect:sec}

\begin{figure}[t]
	\centering
	\includegraphics[width=3.2in]{features.png}
        \vspace{-0.08in}
	\caption{Code Representation Learning for Statement}
%        \vspace{-0.05in}
	\label{fig:feature}	
\end{figure}

%Tien
This section describes our graph-based {\xblock} model. We first
explain how we build the context-aware representation learning for the
given code, and then how we use such learned vectors for the detection
of the presence of \code{try-catch} block using R-GCN~\cite{yi}.

\subsection{Code Representation Learning}
\label{replearn:sec}

Let us present how we build the vectors for code
features. We aim to capture the lexical and structural features for a
statement, while the PDG captures the dependencies among statements.

\vspace{-1pt}
\subsubsection{Sequence of Sub-tokens of a Statement}

At the lexical level, the lexical content of a statement is
represented via a sequence of the sub-tokens. The sub-token
granularity has been shown to have higher regularity than the
tokens~\cite{icse20-methodname}. Each statement is tokenized using
CamelCase or Hungarian convention. Then, only variables, methods,
fields, and class' names are kept. The sub-tokens with one character
are removed to avoid noises. As an example, in
Figure~\ref{fig:example1} at line 2, we collect the sequence of
sub-tokens as follows: \code{final}, \code{Field}, \code{usr},
\code{Paths}, etc. Then, we use a word embedding~\cite{glove2014} to
build the vectors for the sub-tokens, together with Gate Recurrent
Unit (GRU)~\cite{chung2014empirical} to build the feature vector for
the sequence of sub-tokens for the current statement.

%At the lexical level, we capture the content of a statement in term of
%the sequence of sub-tokens. We choose the sub-token granularity
%because the sub-tokens are more likely to be repeated than the entire
%lexical tokens in source code~\cite{icse20-methodname}.
%We tokenize each statement and keep only the variables, method and
%class names. The names are broken into sub-tokens using CamelCase or
%Hungarian convention. We remove the sub-tokens with one character to
%avoid the influence of noises. For example, in
%Figure~\ref{fig:feature}, the tokens of $S_{27}$ are collected and
%broken down into the sequence: \code{copy}, \code{to}, \code{user},
%\code{arg}, {\em etc}. Then, we use GloVe~\cite{glove2014}, to build the
%vectors for tokens, together with Gate Recurrent Unit
%(GRU)~\cite{chung2014empirical} to build the feature vector for the
%sequence of sub-tokens for $S_{27}$.
%GloVe is known to capture well semantic similarity among tokens. GRU
%is chosen to summarize the sequence of vectors into one feature vector
%for the next step.

%In the case of a vulnerable statement and its fixed ones, we combine
%them via multiplication to get the feature vector.

%is an effective method for measuring the linguistic or semantic
%similarity of the tokens. We need the GRU to summarize the sequence of
%vectors into one feature vector for the next step.

%{\tool} use GloVe here because we would like to use a word
%representation tool to transform the natural language tokens into
%vectors which could be used in GCN model. After apply GloVe, the
%sequence of tokens will be transformed into a sequence of vectors. In
%order to get only one vector for one feature, we use GRU here to
%summarize the sequence of vectors into one feature vector for the next
%step.

%{\tool} braking the statement $v$ into sequence and using a GRU model \cite{} to summarize the sequence as the second feature representation vector $F_{v,2}$ for statement $v$ by only taking variable, method, and class names and using the CamelCase and Hungarian convention to break each name into a sequence of sub-tokens and in order to avoid influence of basis, {\tool} removed one character length sub-tokens.

%\subsubsection{{\bf Code structure of the statement}}
%\label{ast:sec}

%Tien
%\vspace{0.06in}
%\noindent {\bf 2. Code Structure of a Statement.}

\vspace{-1pt}
\subsubsection{Code Structure of a Statement}

At the syntactic level, we aim to capture the code structure via the
AST. In Figure~\ref{fig:feature}, we parse the code and extract the
AST subtree for the given statement, and then feed it to the Tree-LSTM
model~\cite{tai2015improved}, which produces a feature vector to
capture the structure.

%Tree-based LSTM model is known for capturing well tree structures.

%\vspace{-1pt}
%\subsubsection{Variables and Types}

%For each node ({\em i.e.}, a statement), we collect the names of the
%variables and their static types at their locations, break them into
%the sub-tokens. For example, we collect the variable \code{s$\_$cmd}
%and its static type \code{cross$\_$ec$\_$command}.
%We use the same vector building techniques as for the sub-token
%sequences as in the feature~1, including GloVe and GRU, to apply on the
%sequences of the sub-tokens built from the variables' names ({\em e.g.},
%\code{s$\_$cmd}) and those from the variables' types ({\em e.g.},
%\code{cross$\_$ec$\_$command}).

%\vspace{-1pt}
%\subsubsection{Surrounding Contexts}

%During training, for a statement $s$, we also encode the statements
%surrounding $s$, which we refer to as {\em context}.  We have two
%contexts. Data- and control-dependency contexts contain the
%statements having such dependencies with the current statement.
%For example, the data-dependency context for $S_{27}$ includes
%the statements at the lines 31, 22, 13, 10, and 6.  If the control
%dependencies are considered, the statements with control dependencies
%with $S_{27}$ at the lines 29, 25, 23, and 13 are included.
%
%The vectors for the statements in the context are calculated via GloVe
%and GRU as described earlier. Because the number of dependencies could be
%different, the lengths of the GRU model inputs could be
%different. Thus, we apply zero padding with a masking layer, allowing the model to skip the zeros at the end of the sequence of
%sub-tokens. Those zeros will not be included in training.

%\vspace{-1pt}
%\subsubsection{Attention-Based Bidirectional GRU} 

%After having all vectors for the features $F_1$, $F_2$, ..., we use a
%bi-directional GRU and an attention layer to learn the weight
%vector $W_i$ for each feature $F_i$, based on the hidden states from
%that model.  Then, we compute the weighted vector for each feature by
%multiplying the original vector for the feature by the weight, that is, we have $F'_i$ = $W_i$.$F_i$.

%Finally, we need to consider the impacts from the {\em dependent statements to the current statement in the PDG}. The rationale is that those
%neighboring statements in the PDG must have the influence on the
%current statement if one of them is vulnerable. For example, the
%neighboring statements for $S_{27}$ in the PDG include the statements
%at lines 6, 22, 25, and 29. Thus, we combine and summarize them into
%the final feature vector $F_{S27}$ for the statement $S_{27}$
%as follows:
%\begin{equation}\label{eq:9}
%F_{S27} = \sum_i{W_i{Concat(h(F'_i,j))}}
%\end{equation}
%$W_i$ is the trainable weight for combination; $Concat$ is the
%concatenate layer to link all values into one vector; $h$ is the
%hidden layer to summarize vector into a value; $i$ = S6, S22, S25,
%S27, S29; $j$ is feature index. $F_{27}$ is used in the next step
%with GCN model for detection.

\subsection{\code{Try-catch} Block Checker with R-GCN}
\label{model:sec}

\begin{figure}[t]
	\centering
	\includegraphics[width=3.4in]{xblock.png}
  %      \vspace{-0.1in}
	\caption{\code{Try-catch} Block Checker with R-GCN ({\xblock})}
	\label{fig:gcn}	
\end{figure}

Figure~\ref{fig:gcn} presents how we use the R-GCN model~\cite{yi} to
detect if a \code{try-catch} block is needed.
%The rationale is that FA-GCN can deal well with the graphs with sparse
%features (not all the statements share the same properties), and
%potentially noisy features in a PDG.
First, the code is processed by DeepPDA~\cite{icse23} to build
the PDG. The R-GCN, similar to CNN in processing an image, performs
a sliding window along the nodes in the graph. A window for a node
consists of the neighboring nodes in the PDG.
%Similar to CNN using the filter on an image, FA-GCN performs sliding
%a small window along all the nodes (statements) of the PDG. For
%example, in Figure~\ref{fig:gcn}, the window marked with A for the
%node $S27$ consists of itself and the neighboring statements/nodes
%$S6$, $S22$, $S25$, and $S29$. Another window (marked with B) is for
%the node $S23$, including itself and the neighboring nodes: $S22$ and
%$S25$.
To process a window, the model generates the feature representation
matrix for the node at the center using the procedure described in
Figure~\ref{fig:feature}.
%For example, for the window centered at $S27$, it generates the
%feature vector $F_{S27}$ for $S27$, using the process explained in
%Figure~\ref{fig:feature}.
From the representation vectors for all statements (nodes), the R-GCN
model will produce the outputs at the output layer. We then connect
all of its outputs to a fully connected layer to transform the matrix
into a vector $V_C$ to represent the given code $C$. With $V_C$, we
perform classification by using two hidden layers and a softmax
function to produce a classification for $C$ if it needs a
\code{try-catch} block or not.

%We use those scores as {\em vulnerability scores to rank the methods}
%in a project.

%uses a join layer to link all these vectors into the
%Feature Matrix $\mathcal{F}_{m}$ for method $M$. A row in
%$\mathcal{F}_m$ corresponds to a window in~PDG.
%Next, FA-GCN performs the convolution operation by first calculating
%the symmetric normalized Laplacian matrix~$\tilde{A}$~\cite{GCN16},
%and then calculating the convolution to generate the representation
%matrix $M_{m}$ for the method $m$. After that, we use the traditional
%steps as in a CNN model: using a spatial pyramid pooling layer (to
%normalize the method representation matrix into a uniform size, and
%reduce its total size), and connecting its output to a fully connected
%layer to transform the matrix into a vector $V_m$ to represent
%$m$. With $V_m$, we perform classification by using two hidden layers
%(controlling the length of vectors and output) and a softmax function
%to produce a prediction score for $m$. We use those scores as {\em
%  vulnerability scores to rank the methods} in a project. The decision
%for $m$ as $\mathcal{V}$ or $\mathcal{NV}$ is done via a trainable
%threshold on the prediction score~\cite{li2018vuldeepecker,li2019improving}.

%The model also assigns a score for $m$. We consider those scores



%--- old ----
%Next, FA-GCN performs the convolution operation by first
%calculating the symmetric normalized Laplacian matrix~$\tilde{A}$:
%\begin{equation}\label{eq:7}
%\tilde {A} =D^{-\frac{1}{2}}(I+A)D^{-\frac{1}{2}}
%\end{equation}
%Where $D$ is the degree matrix, $I$ is the identity matrix, and $A$ is
%the adjacency matrix. With the symmetric normalized Laplacian matrix,
%FA-GCN performs the convolution calculation to generate the
%representation matrix $M_{m}$ for method $M$:
%\begin{equation}\label{eq:8}
%{M}_{m}  = g(\tilde{A}\mathcal{F}_mW)
%\end{equation}
%Where $W$ is the weight matrix, and $g$ is an activation function.

%Next, we perform the classification of $\mathcal{V}$ or $\mathcal{NV}$
%for the given method $M$. To achieve that, we use a spatial pyramid
%pooling layer (with the same role as in the CNN model) to normalize
%the method representation matrix, into a uniform size, and then reduce
%the total size of that matrix. Then, we send the output from the
%pooling layer into a fully connected layer to transform the matrix
%into a vector to represent the method $m$. Let us call it the
%representation vector for $V_{m}$.
%
%After having the vector $V_{m}$ for $M$, {\tool} uses a classifier for
%prediction. The classifier is built by two hidden layers, which are
%used to control the length of vectors and output, and the Softmax
%activation function. We use this classifier since it is simple and
%common used in many classification problem~\cite{li2018vuldeepecker,
%  li2019improving}.
%----



%These pooling and fully connected layers are widely used in the CNN
%model, and they play the same role in the FA-GCN model.

%%%%%%%%%%%%%%%%%%%%%%%%
%The
%statement representation matrix for $S27$ is also marked with (A),
%corresponding to the window (A) in the PDG. Similarly, the matrix for
%$S23$ is marked with (B) for the window (B) in the PDG.
%%%%%%%%%%%%%%%%%%%%%%%%
%After we talked about the details about the feature representation generation. We would like to introduce the process for {\tool} to use it to do the vulnerability detection. As we mentioned, first of all, we build the program dependency graph (PDG) for the incoming method $M$. Then, to do the convolution calculation, {\tool} swipes on the PDG for each node just like the CNN \cite{} model swipe the filters on the images. For example, in figure \ref{fig:gcn}, you can see that the blue window and green window on the PDG shows two swipes. The green window shows that the {\tool} is swiping on the \textit{S27} and also including the neighborhood statements include \textit{S6, S22, S25, S29}. Also, the same as the green window, the blue window shows that the {\tool} is swiping on node \textit{S23} with neighborhood statements \textit{S22, S25}.
%After having each statement swiping, we use them to generate the statement representation vectors. We use \textit{S27} as an example. After having the green window, {\tool} firstly do the feature representation generation which is shown in figure \ref{fig:feature} to generate the feature matrix $M_{S27}$ for \textit{S27}. Then we calculate the symmetric normalized Laplacian matrix $\tilde {A}$ by:
%\begin{equation}\label{eq:7}
%\tilde {A} =D^{-\frac{1}{2}}(I+A)D^{-\frac{1}{2}}
%\end{equation}
%%%%%%%%%%%%%%%%%%%%%%%%
%After this step, for each statement/node, we have a statement
%representation matrix (e.g., $M_{S27}$).
%%%%%%%%%%%%%%%%%%%%%%%%



%%%%%%%%%%%%%%%%%%%%
%To achieve that, we need to combine the representations for all
%the statements. We use a spatial pyramid pooling layer to
%normalize the statement representation matrix, e.g., $M_{S27}$ into a
%uniform size, and then reduce the total size of that matrix. Then, we
%send the output from the pooling layer into a fully connected layer to
%transform the matrix into a vector to represent the statement $S27$.
%Let us call it the statement representation vector for $V_{S27}$.
%These pooling and fully connected layers are widely used in the CNN
%model, and they play the same role in the FA-GCN model.
%%%%%%%%%%%%%%%%%%%%
%Because {\tool} needs to do the classification on the whole method which requires to combine all statement representation information together, for each statement, such as \textit{S27}, {\tool} firstly uses a spatial pyramid pooling layer to normalize the $M_{S27, cov}$ into a uniform size and then reduce the total size of the matrix. Then, {\tool} send the output from spatial pyramid pooling layer into a fully connected layer to transform the matrix into statement representation vector for \textit{S27}. These pooling and fully connected layers are also been widely used in CNN model and they playing the similar roles in CNN model.




%%%%%%%%%%%%%%%%%%%%
%After having all statement representation vectors, {\tool} use a join
%layer to link these vectors together and then use the resulting vector
%as the method representation vector for the entire method $M$. Next,
%{\tool} uses a classifier for prediction. The classifier in {\tool} is
%built by 2 hidden layers which are used to control the length of
%vectors and output, and apply the Softmax activation function for
%classification. We use this classifier because it is simple and common
%used in many DL-based classification problem \cite{yi}.
%%%%%%%%%%%%%%%%%%%
%After having the feature-attention based representation vector $x_{V,atte}$, {\tool} generates summarized $N \times D$ feature matrix $X_M$ where $N$ is number of nodes and $D$ is the number of node features. With this matrix, {\tool} is able to preserve the first-order neighborhood relations between statement, and the relationship feature matrix $\tilde{X_M}$ could be computed by:

%\begin{equation}\label{eq:7}
%\tilde {X_M}  = g(\tilde{A}X_MW)
%\end{equation}

%Where $\tilde {A} =D^{-\frac{1}{2}}(I+A)D^{-\frac{1}{2}}$ is the normalized symmetric adjacency matrix, $W$ is the weight matrix, and $g$ is an activation function ReLU represented by $g= max(0, x)$.

%After having $\tilde X_M$, {\tool} is able to use a convolutional layer to classify the whole graph into two groups including \textit{vulnerable} and \textit{non-vulnerable} which is the output of this step in {\tool}.


\section{{\xstate}: Try-Catch Statement Detector}
\label{interpretation:sec}

{\xstate} also uses CodeBERT, the same one as in {\xblock}. However, we create a separate figure for each module so that it is easier to follow. 

Each \texttt{[SEP]} token represents the line of code preceding it (Note that semicolon would not be as consistent a line separator as the \texttt{[SEP]} token, since a line of code may not end with a semicolon. And a semicolon often appears inside string literals and inside for-loop conditions. ). We expect that the relationship between lines of code can be learnt through these \texttt{[SEP]} tokens during training, so that at prediction time, the model can determine for each line as to whether it is inside or outside a try-catch block. We devise three tags in the IOB2 format to accomplish this task: \texttt{O} tag means that the statement is outside any try-blocks; \texttt{B-Try} tag means the statement begins a try-block; and \texttt{I-Try} tag means the statement is inside a try-block and is not the first line in that try-block. 

During training, tags of all the lines are known from the code. During prediction, the model will be able to assign tags to all the lines in the code snippet. From these tags, we get to know how many try-blocks should be added, and which lines of code belong to which try-block.



% This section presents {\xstate} and how we leverage an Explainable AI
% model, GNNExplainer~\cite{GNNExplainer}, to detect what statements in
% the given code snippet $C$ need to be placed in a \code{try-catch} block
% provided that {\xblock} had decided the necessity of such a block.

% %Tien
% {\xstate} takes the following as inputs: 1) the PDG ($G_C$) of the
% given code $C$, 2) the trained R-GCN model for {\xblock}, along with
% the positive detection result $\mathcal{V_C}$ and the prediction score.
% Figure~\ref{fig:GNNEX} displays an illustration of the process in
% {\xstate}.

% % Let us explain how we use GNNExplainer~\cite{GNNExplainer} to build
% %our graph-based interpretation. The input includes the trained FA-GCN
% %model, the PDG ($G_M$) of the method $M$, and the detection result
% %$\mathcal{V}$ or $\mathcal{NV}$, and prediction score.

% %Figure~\ref{fig:GNNEX} illustrates our process for the case of
% %$\mathcal{V}$ (Vulnerable) (the case of $\mathcal{NV}$ is done
% %similarly).



% %The interpretations include 1) a crucial sub-graph $\mathcal{G}_M$,
% %corresponding to the PDG sub-graph consisting of statements relevant
% %to the vulnerability, and 2) a subset of crucial features
% %$\mathcal{X}_M$, corresponding to the set of variables involving to
% %the vulnerability.

% To decide the statements in the given code to be placed in a
% \code{try-catch} block, we aim to determine the statements that are
% most decisive and crucial for the {\xblock} model in deciding its
% outcome of ``Yes'' (i.e, {\xblock} decides that the snippet must be
% placed in a \code{try-catch} block). Therefore, we formulate that as
% the following problem: finding a sub-graph $\mathcal{G}_C$ in the PDG
% $G_C$ for $C$ that minimizes the difference in the prediction scores
% between using the entire graph $G_C$ and using the minimal graph
% $\mathcal{G}_C$. To do so, we leverage the Explainable AI model,
% GNNExplainer~\cite{GNNExplainer}. It uses a masking technique in which
% instead of searching for that minimal subgraph, it learns the {\em
%   edge-mask} set $EM$ of the edges and the {\em feature-mask} set $FM$
% of the features. Learning those two sets $EM$ and $FM$ helps derive
% the explanation sub-graph $\mathcal{G}_C$ and the set of crucial
% feature $\mathcal{X}_C$ by masking-out the edges in $EM$ from $G_C$,
% and the feature in $FM$ from $X_C$:
% \begin{equation}\label{eq:11}
% \mathcal{G}_C = G_C \bigodot EM
% \end{equation}
% \begin{equation}\label{eq:12}
% \mathcal{X}_M = X_M \bigodot FM
% \end{equation}
% $\bigodot$ is used to denote the ``masking-out'' operation.

% %To derive the interpretations, the key goal is to find a sub-graph
% %$\mathcal{G}_M$ in the PDG $G_M$ of the method $M$ that minimizes the
% %difference in the prediction scores between using the entire graph
% %$G_M$ and using the minimal graph $\mathcal{G}_M$. To do so, we use
% %GNNExplainer with the {\em masking technique}~\cite{GNNExplainer},
% %which treats the searching for the minimal graph $\mathcal{G}_M$ as a
% %learning problem of the {\em edge-mask} set $EM$ of the edges.  The
% %idea is that learning $EM$ helps {\tool} derive the interpretation
% %sub-graph $\mathcal{G}_M$ by masking-out the edges in $EM$ from $G_M$
% %(``masked-out'' is denoted by
% %$\bigodot$): \begin{equation}\label{eq:11} \mathcal{G}_M = G_M
% %\bigodot EM \end{equation}

% Figure~\ref{fig:GNNEX} illustrates the principle. Note that the
% trained R-GCN model in {\xblock} had decided that a \code{try-catch}
% block is needed for the given code (in which the entire original PDG
% was used as input). When an edge-mask set is applied, some of the
% edges (and the inducing nodes) are masked. With the new graph being
% used as~the new input, if the trained R-GCN model in {\xblock}
% produces ``Yes'' as the result, i.e., {\em the classification does not
% change, then the edges in the edge-mask set are not important}. Thus,
% they will not be included in the explanation graph $\mathcal{G}_C$. If
% the trained R-GCN model in {\xblock} with the new input graph produces
% ``No'' as the result, i.e., {\em the classification does change from
% ``Yes'' to ``No'', then those edges in the edge-mask set are important
% to the model, thus, being included in the explanation graph
% $\mathcal{G}_C$}. Similar logic is applied to include or exclude the
% features in the {\em feature-mask} $FM$.



% %Tien Figure~\ref{fig:GNNEX} illustrates GNNExplainer's principle. As
% %an edge-mask set is applied, GNNEXplainer checks if the FA-GCN model
% %produces the same result~(in this case the result is
% %$\mathcal{V}$). If yes, the edge in the edge-mask is not important
% %and is not included in $\mathcal{G}_M$. Otherwise, the edge is
% %important and included in $\mathcal{G}_M$.  Because the numbers of
% %possible sub-graphs and the edge-mask sets are untractable,
% %GNNExplainer uses a learning approach for the edge-mask $EM$.

% %\begin{equation}\label{eq:12}
% %\mathcal{X}_M = X_M \bigodot FM
% %\end{equation}

% %For example, as you can see in figure \ref{fig:GNNEX}, when the
% %GNNExplainer get the whole PGD $G_m$, it starts to use the
% %\textit{edge-mask} $EM$ and \textit{feature-mask} $FM$ to mask some
% %edges and features.  As you can see in the second row of the figure,
% %after masked some edges from $G_m$ to generate a graph $G'_m$,
% %GNNExplainer use the detection model to check the new graph to see
% %how the detection result changes. If the result has been influenced a
% %lot, it means that the masked edges are important for the detection
% %result. If not, it means that the masked edges are not important for
% %the detection result. It is similar for the \textit{feature-mask}
% %$FM$.  Because in the last step, we could get $F_{v}$ for each
% %statement $v$ and $F_{v}$ have the same dimension, $FM$ masks the
% %features in the same position in $F_{v}$ for each statement $v$ and
% %then use detection model to evaluate the importance of the masked
% %features just as the edges. As you can see in figure \ref{fig:GNNEX},
% %$FM$ masked the second and the 4th feature for each statement in this
% %example.

% \begin{figure}[t]
% 	\centering
% 	\includegraphics[width=3in]{XAI.png}
%         \vspace{-0.06in}
% 	\caption{Masking to derive the explanation sub-graph containing statements in a \code{Try-Catch} block ({\xstate})}
%         \vspace{-0.06in}
% 	\label{fig:GNNEX}	
% \end{figure}

% %\begin{align}\label{maineq}
% %\nonumber
% %\max_{\mathcal{G}_M} MI(Y,(\mathcal{G}_M,\mathcal{X}_M)) = H(Y) - H(Y|G=\mathcal{G}_M,X=\mathcal{X}_M)
% %\end{align}

% %to minimizing conditional entropy
% %$H(Y|G=\mathcal{G}_M,X=\mathcal{X}_M)$

% %\begin{equation}
% %  \label{eq2}
% %-\EX_{Y|\mathcal{G}_M,
% %  \mathcal{X}_M} [log P_{FA-GCN} (Y|G=\mathcal{G}_M,X=\mathcal{X}_M)]
% %  \end{equation}

% %\begin{equation}
% %  \label{eq3}
% %  \min_{\mathcal{G}} \EX_{\mathcal{G}_M \sim \mathcal{G}} H(Y|G=\mathcal{G}_M,X=\mathcal{X}_M)
% %\end{equation}
% %\begin{equation}
% %  \label{eq4}
% %  \min_{\mathcal{G}} H(Y| G=\EX_{\mathcal{G}}[\mathcal{G}_M], X=\mathcal{X}_M)
% %\end{equation}
% %\nonumber

% Because the numbers of possible sub-graphs, edge-mask sets, and
% feature-mask sets are untractable, GNNExplainer uses a learning
% approach for $EM$ and $FM$. It aims to maximize the mutual information
% (MI) between the minimal graph $\mathcal{G}_C$ and the input
% PDG~$G_C$~\cite{GNNExplainer}:
% \begin{equation}\label{maineq}
% \max_{\mathcal{G}_C} MI(Y,\mathcal{G}_C) = H(Y) - H(Y|G=\mathcal{G}_C)
% \end{equation}
% $Y$ is the outcome decision by the trained R-GCN model. Thus, the
% entropy term $H(Y)$ is constant for the trained R-GCN
% model.

% Maximizing the $MI$ value for all $\mathcal{G}_C$s is equivalent
% to minimizing conditional entropy $H(Y|G=\mathcal{G}_C)$, which by
% definition of conditional entropy can be expressed~as
% \begin{equation}
%   \label{eq2}
% -\EX_{Y|\mathcal{G}_C}
%   [log P_{R-GCN} (Y|G=\mathcal{G}_C)]
% \end{equation}
% This conditional entropy is a measure of how
% much uncertainty remains about the outcome $Y$ when we know
% $G=\mathcal{G}_C$.
% %
% %GNNEXplainer also limits the size of $\mathcal{G}_C$ by $K_C$, {\em
% %  i.e.}, taking $K_C$ edges that give the highest mutual information
% %with the prediction outcome $Y$.
% %
% %Direct optimization of the formula(~\ref{eq2}) is not tractable, thus,
% %GNNExplainer treats $\mathcal{G}_C$ as a random graph variable
% %$\mathcal{G}$. The objective in Equation(~\ref{eq2}) becomes:
% %\begin{equation}
% %  \label{eq3}
% %  \min_{\mathcal{G}} \EX_{\mathcal{G}_C \sim \mathcal{G}} H(Y|G=\mathcal{G}_C)
% %\end{equation}
% %\begin{equation}
% %  \label{eq4}
% %  \min_{\mathcal{G}} H(Y| G=\EX_{\mathcal{G}}[\mathcal{G}_C])
% %\end{equation}
% %From Equation~\ref{eq3}, we obtain Equation~\ref{eq4} with Jensen's
% %inequality.  The conditional entropy in Equation~\ref{eq4} can be
% %optimized by replacing $\EX_{\mathcal{G}}[\mathcal{G}_C]$ to be
% %optimized by masking with $EM$ on the input graph $G_C$.  Now, we can
% %reduce the problem to learning the mask $EM$.  Similar logic is
% %applied to $FM$.

% More details can be found in~\cite{GNNExplainer}. The resulting
% sub-graph $\mathcal{G}_C$ is used to derive the statements to be
% placed in a \code{try-catch} block. Note that, we can set the
% statements to be consecutive.


\section{{\xtype}: Exception Type Recommender}
\label{sec:xtype}

\begin{figure}[t]
\begin{center}
\includegraphics[width=3.2in]{xtype-6.png}
\vspace{-8pt}
\caption{Exception Type Recommendation ({\xtype})}
\label{fig:xtype}
\end{center}
\end{figure}

The goal of {\xtype} (Figure~\ref{fig:xtype}) is to predict what
exception types need to be placed in the \code{catch} clause of each
of the predicted \code{try-catch} block(s) for the given input code
snippet. We use one single CodeBERT~\cite{codebert-emnlp20} model as
in {\xblock} and {\xstate} to build the embeddings for code (sub)-tokens.
%
We expect CodeBERT to learn the connection between the statements in a
\code{try} block and the corresponding exception types to be
caught. From Key idea 2, we expect that via context, CodeBERT can
implicitly learn the dependencies among API elements, leading to
better learning of the exception~types.

During training, we know all the exceptions to be caught in a code
snippet. For each \code{try} block, we identify the statement
that begins it (with the \code{B-Try} label) and ends it (with the
last respective \code{I-Try} label). Then, we add the embeddings, from
CodeBERT, for the [SEP] tokens that correspond to the statements
inside the \code{try} block and feed this \code{try} block's representation vector into a linear layer.
%We consider the embeddings from CodeBERT for the {\bf [SEP]} tokens
%corresponding to the statements from the beginning to the end of the
%block. We add them together to get the embedding for the entire
%\code{try-catch} block, and feed it into a linear layer.
We use a sigmoid function to perform binary classifications for the
exception types of interest.

During prediction, we use the predicted tags for the given statements
in the code snippet. From the predicted tags, we obtain the the
statements in a predicted \code{try} block. From there, the
embeddings computed by CodeBERT are used in the same way as in
training. For example, in Figure~\ref{fig:xtype}, the model predicts
the \code{try} block from the statement \code{out.write} to the
statement \code{print(node.getBody(...))}. The embeddings of all the
statements in the block are used and the output of \code{IOException}
is predicted since its score is higher than 0.5.
%with the highest probability.



%For each try block, we first take the vector outputs of
%\texttt{[SEP]} tokens, from CodeBERT, that correspond to statements
%in the try block.  If we included more libraries for the study, the
%classification head will likely need to change to a different one.
%During training, we know all the tags associated with every line in
%the input code snippet, so we know how to connect them, while





%\begin{figure}[t]
%\begin{center}
%\includegraphics[width=3in]{xtype-3.png}
%\vspace{-10pt}
%\caption{Exception Type Recommendation ({\xtype})}
%\label{fig:xtype}
%\end{center}
%\end{figure}
%
%This section describes {\xtype} that recommends the exception types to
%be handled in the \code{catch} clause for the given code, after the code was
%determined to require a \code{try-catch} block and some of its
%statements were determined to be placed in the \code{try-catch} block.
%
%We use another R-GCN model that acts as a classifier for different
%exception types (Figure~\ref{overview}). For training, the source code
%with \code{try-catch} blocks is used as input. {\tool} parses and
%processes them in the same way to build the PDG subgraph and the
%feature vector representations for the statements as in {\xblock}
%(Figure~\ref{fig:feature}). However, in prediction, the input is the
%sub-graph $\mathcal{G}_C$ of the PDG because the GNNExplainer already
%determined that those statements in that sub-graph need to be placed
%in a \code{try-catch} block. In this case, the feature vectors for the
%sub-graph are also built in the same way as in
%Figure~\ref{fig:feature}. The R-GCN model processes the input
%(sub)graph in a similar manner as in Figure~\ref{fig:gcn} except for the
%processing on the model's output. Instead of connecting
%the output of the R-GCN model to a fully connected layer, we
%feed that output to multiple softmax functions to act as the
%classifiers for the exception types in the dictionary (e.g.,
%\code{IOException}). Each classifier is responsible for one exception
%type.  We could set the maximum number of exception types. The
%positive output from a classifier indicates the presence of the
%corresponding exception type in the \code{catch} clause.


\section{Multi-task Learning}
\label{sec:multitasking}

The two tasks dealt with by {\xstate} and {\xtype} can benefit each other. With the learning about which statements need to be put into try blocks, the model will learn the important code tokens in the code snippet, which in turn will help to determine which exceptions need to be handled. Also, knowing the exceptions will give strong hint to the way that statements should be grouped into try blocks.

We calculate the training loss by combining losses from the three modules. $Loss_{{\xblock}}$ is the Binary Cross Entropy loss for the decision as to whether try-catch blocks exist in the input. To calculate $Loss_{{\xstate}}$, we first get the classification loss for each statement ($loss_{Stmt}$) in the input, and sum them together. $loss_{Stmt}$ is the Cross Entropy loss calculated from the distribution for the three tags---\texttt{O}, \texttt{B-Try}, \texttt{I-Try}---and the ground-truth tag. Finally, in \xtype, several try blocks might be present, so $Loss_{{\xtype}}$ comes from the summation of the exception prediction losses from all try blocks. For each try block, the $loss_{try-block}$ is calculated by adding the Binary Cross Entropy loss for the prediction of each exception that we considered. 

The overall training loss is calculated as following: if the input does not contain any try-catch block, the loss will just be the $Loss_{{\xblock}}$. On the other hand, if the input contain any try-catch block, then the overall loss will be the summation of the losses from all three tasks (\ref{eqn:loss}). 

\begin{equation}
\label{eqn:loss}
Loss_{overall} =
\begin{cases}
Loss_{{\xblock}},  & \text{no try-catch}\\
Loss_{{\xblock}} + Loss_{{\xstate}} + Loss_{{\xtype}}, &\text{otherwise.}
\end{cases}
\end{equation}

\section{Empirical Evaluation}
\label{sec:eval}

\subsection{Research Questions}

We conducted several experiments to evaluate {\tool}. We seek to
answer the following questions:

\vspace{2pt}
%\noindent \textbf{RQ\textsubscript{1}. [Effectiveness on Java code]
%  (Intrinsic Evaluation).} {\em How accurate is {\tool} in generating
%  CFG/PDGs for Java code in general and in different dependency
%  types?}

\noindent \textbf{RQ\textsubscript{1}. [Effectiveness on Try-Catch
    Necessity Checking]} {\em How accurate is {\tool} in predicting
  whether a given code snippet needs to have a \code{try-catch}
  block?}
    
%            program-dependency edges generated by \tool for Java code?
%            How accurately does \tool predict different types of
%            control-flow and program-dependency edges for Java code?}


\vspace{2pt}
\noindent \textbf{RQ\textsubscript{2}. [Effectiveness on Try-Catch
    Statement Detection].} {\em How accurate is {\tool} in predicting
  which statements in a given code snippet needs to be placed in a
  \code{try-catch} block?}

%\vspace{2pt}
%\noindent \textbf{RQ\textsubscript{4}. [Ablation Study on C/C++
%    code].}  {\em How do the different components in \tool contribute to
%  model performance?}

\vspace{2pt}
\noindent \textbf{RQ\textsubscript{3}. [Effectiveness on Exception Type Recommendation].}
{\em How accurate is {\tool} in recommending what exception types need to be handled in the \code{catch} clause of a \code{try-catch} block?}

\vspace{2pt}
\noindent \textbf{RQ\textsubscript{4}. [Dependency Probing].}
{\em How well {\tool} learn the dependencies among statements for grouping them into a \code{try-catch} blocks}

\vspace{2pt}
\noindent \textbf{RQ\textsubscript{5}. [Extrinsic Evaluation on
    Exception-related Bug Detection].}  {\em How well does {\tool}
  detect exception-related bugs?}

\subsection{Empirical Methodology}

\subsubsection{Datasets}

We have conducted our experiments to evaluate {\tool} on the dataset DeepEx that we collected, and the existing dataset FuzzyCatch provided in a prior work~\cite{nguyen2020code}. 

To build the dataset DeepEx, we first collect XXX methods from XXX Java projects. And then, we picked XXX methods that contain at least one try-catch block from all methods. To avoid the influence of multiple try-catch blocks, we split the methods into code snippets by checking the above and below statements, one by one, starting from the closest one for each try-catch block, to verify if the statement is included in the other try-catch block. If not, we put it in the current code snippet. If yes, we stop here and finish the code snippet.

Following the steps mentioned above, we have XXX code snippets containing one try-catch block as positive data. Then, we randomly create the same amount of code snippets that do not contain a try-catch block as negative data from the Java projects we collected. So, in total, the DeepEx dataset includes XXX code snippets. As for the FuzzyCatch dataset, it includes 1,000 data, and all of them are positive data.


\subsubsection{RQ1. Effectiveness on Try-Catch Necessity Checking}

{\em Baselines.} We compared {\tool} against the state-of-the-art try-catch necessity checking approach XRank~\cite{nguyen2020code}.

{\em Procedure.} We took all the code snippets from DeepEX and FuzzyCatch datasets. On the DeepEx dataset, we randomly split both the positive and negative data points into 80\%, 10\%, 10\%, in which 80\% of the code snippets as the training dataset, 10\% of the code snippets as the tuning dataset, and 10\% of the code snippets as the testing dataset for the baseline and {\tool}. And on the FuzzyCatch dataset, we directly use the trained model from the DeepEx dataset and test it on the FuzzyCatch dataset for both the baseline and {\tool}.

{\em Tuning.} We tuned {\tool} with autoML~\cite{NNI} for the following key hyper-parameters to have the best performance: (1) Epoch size (50, 100, 150); (2) Batch size (32, 64, 128); (3) Learning rate (0.001, 0.003, 0.005); (4) Vector length of feature embeddings and its output (64, 128, 256); (5) Number of GCN layers (4, 6, 8).

{\em Metrics.} In this RQ, we use \textbf{Recall, Precision, and F-score} to evaluate the performance of \tool and the baseline. They are calculated as $Recall = \frac{TP}{TP+FN}$, $Precision = \frac{TP}{TP+FP}$, $F-score = \frac{2*Recall*Precision}{Recall+Precision}$

\subsubsection{RQ2. Effectiveness on Try-Catch Statement Detection}

{\em Procedure.} In this RQ, \tool is evaluated on the DeepEx dataset. Because only the data points that are predicted as needing a try-catch block, the \tool will run the try-catch statement detection on them. To fully use the dataset and evaluate \tool on more data, here we estimate the accuracy of RQ1 prediction is 100\% which means that we use all positive data in DeepEx dataset as the data we use in this RQ. We do the same data split as RQ1 to train, validate and test the model performance.

{\em Tuning.} We tuned {\tool} in this RQ by testing different node number limits for generated sub-graph. The range of the node number limit we tested is from $1$ to $10$ nodes.

{\em Metrics.} In this RQ, we use \textbf{Accuracy} as the evaluation metrics. For each statement in the try-catch block, if \tool included it in the generated sub-graph, we think the prediction on it is correct. Otherwise, it is incorrect. The Accuracy is calculated as $Accuracy = \frac{Correct}{Correct + Incorrect}$.

\subsubsection{RQ3. Effectiveness on Exception Type Recommendation}

{\em Baselines.} We compared {\tool} against the state-of-the-art exception type recommendation approach XRank~\cite{nguyen2020code}.

{\em Procedure.} In this RQ, \tool is evaluated on the DeepEx dataset. Because only the data points that \tool can successfully predict all statements in the try-catch block successfully, the \tool will run the exception type prediction on them. Similarly, as RQ2, to fully use the dataset and evaluate \tool on more data, here we estimate the Accuracy of RQ2 prediction is 100\% which means that we use all positive data in DeepEx dataset as the data we use in this RQ. We do the same data split as RQ1 and RQ2 to train, validate and test the model performance.

{\em Tuning.} We tuned {\tool} with autoML~\cite{NNI} for the following key hyper-parameters to have the best performance: (1) Epoch size (50, 100, 150); (2) Batch size (32, 64, 128); (3) Learning rate (0.001, 0.003, 0.005); (4) Vector length of feature embeddings and its output (64, 128, 256); (5) Number of GCN layers (4, 6, 8).

{\em Metrics.} In this RQ, we use \textbf{Hit-n} as the evaluation metrics. Hit-n here means within the true labeled statement set, there are at least \textbf{n} statements predicted correctly by \tool.

\subsubsection{RQ4. Ablation Study}

{\em Procedure.} In this RQ, we would like to evaluate the impact of the key features SOT (sequence of tokens) and AST (abstract syntax tree) on the model performance in RQ1 and RQ3. To do that, we remove one key feature at a time and use the differences between the experimental results to evaluate the impact of that feature.

{\em Metrics.} We use the same evaluation metrics as RQ1 and RQ3 to evaluate the impact of the different features on the experiment results.


\section{Empirical Results}
\label{sec:results}

\subsection{Comparison on Try-Catch Necessity Checking Effectiveness (RQ1)}
%\label{sec:rq1}

\begin{table}[t]%[htpb]
  \caption{Try-Catch Block Comparison with XRank (RQ1)}
  \vspace{-12pt}
  \small
	\begin{center}
		\renewcommand{\arraystretch}{1}
		\begin{tabular}{| p{3.05cm}<{\centering} | p{1.2cm}<{\centering} | p{1.2cm}<{\centering}| p{1.2cm}<{\centering}|}
		  \hline
			Github dataset  & Precision  &  Recall & F1-score \\
			\hline
%			CodeBERT w/o fine-tuning & 0.4969  & \textbf{0.9719}   & 0.6576\\
			\hline
			XRank & 0.810 & 0.530 & 0.630\\
			\hline
			\tool   &  \textbf{0.981} &  {\bf 0.984} & \textbf{0.982}\\
			\hline
		\end{tabular}
		\label{tab:xblock-1}
	\end{center}
\end{table}

\begin{table}[t]%[htpb]
  \caption{Try-Catch Block Comparison with GPT-3.5 (RQ1)}
  \vspace{-12pt}
  \small
	\begin{center}
		\renewcommand{\arraystretch}{1}
		\begin{tabular}{| p{1.85cm}<{\centering} | p{1.6cm}<{\centering} | p{1.6cm}<{\centering}| p{1.6cm}<{\centering}|}
		  \hline
		Small dataset	  & Precision  &  Recall & F1-score \\
			\hline
			GPT-3.5  & 0.666--0.804  & 0.726--0.778   & 0.695--0.791\\
			\hline
			\tool   &  \textbf{0.994} &  {\bf 1.0} & \textbf{0.997}\\
			\hline
		\end{tabular}
		\label{tab:xblock-2}
	\end{center}
\end{table}


%{\color{red}{This section waiting for the XRank Results. But from the current estimate, our approach should have higher F-score. But the recall and precision I'm not sure. Once I have the results, I will update this section.}}

%Table~\ref{tab:xblock} displays the comparison result.

As seen in Table~\ref{tab:xblock-1}, {\tool} achieves very high
Precision, Recall and F-score on the Github dataset---all above
98\%.
%In comparison, the CodeBERT baseline model has a much lower
%Precision, around 50\%. However, it achieves a slightly higher
%Recall. After examining the result, we find that the model
%overwhelmingly predicts that the input
%code snippet contains a \code{try-catch} block: In our balanced test
%dataset that contains 30,764 samples, only 236 samples receives the
%negative label (i.e., no \code{try-catch}) from CodeBert.
In comparison, {\tool} relatively improves over XRank {\bf 21\%, 85.7\%,
and 55.9\%} in Precision, Recall, and F1-score, respectively.

Examining the result, we reported the following. First, precision is
just marginally better than a coin toss (0.53) in our balanced
dataset. In XRank, if the association score of {\em only one API
  method} in the snippet and {\em one exception} is higher than a
threshold, it decides that a \code{try-catch} block is needed.
%
Second, the decisions on the necessity of a \code{try-catch} block or
the exception types depend on the pre-defined thresholds in XRank on
those association scores. Thus, those pre-defined thresholds might not
be suitable across the libraries. Third, for the incomplete code
snippets in which the names of the API methods in different packages
or libraries are the same (e.g., \code{toString} or \code{getText} in
various JDK packages), XRank cannot distinguish them and use one entry
in the dictionary for them due to its IR approach, leading to
mistakenly considering them the same. Unlike XRank, which
considers only the API method calls in a \code{try-catch} block,
{\tool} considers the code in the block as the context to learn the
relations among the names of those API elements,
%That is, it leverages the relations among the names of API elements to
%learn their identities,
thus, better deciding the need of \code{try-catch} blocks.
%and the corresponding exception types.

%ChatGPT results ==== ?

As seen in Table~\ref{tab:xblock-2}, {\tool} relatively improves over
GPT-3.5 from {\bf 23.6\%--49.3\%}, {\bf 28.5\%--37.7\%}, and {\bf
  26\%--43.5\%}, in Precision, Recall, and F1-score,
respectively. Examining GPT-3.5's results, we found that it detected
well only the popular APIs and corresponding exception types because
it was not trained specifically for the exception handling
task. Moreover, for the un-popular API names, GPT-3.5 often resorted
to another API with similar name, and predicts that the given code
snippet needs a \code{try-catch} block because that API requires such
a block. For example, in an instance containing a method call to
\code{interval.parseWithOffset}, which is specific to a project,
GPT-3.5 incorrectly considered it to have a \code{try-catch}
block. GPT-3.5 explained that it is similar to \code{parse}
in a compiler project, which needs to handle
\code{InvalidInputException}. Thus, it incorrectly considers
\code{parseVals} needs to handle that exception.

\begin{figure}[t]
 	\centering
 	\includegraphics[width=3.4in]{rq1-case-study.png}
        \vspace{-20pt}
 	\caption{{\xblock} Case Study}
 	\label{fig:rq1-case}	
\end{figure}

\noindent {\bf Attribution Scores.} To illustrate how {\xblock} makes
the prediction, in Figure~\ref{fig:rq1-case}, we shows a code snippet
that catches an \code{IOException} thrown by the \code{readAllBytes}
API call on an \code{InputStream} object. CodeBERT produces as a
by-product an {\em attribution score} for each code sub-token in the
input. The higher the score of a token the higher attention that the
model pays to that token, contributing to the prediction result. In
Figure~\ref{fig:rq1-case}, for each statement, we show the statement
attribution score, which is calculated by averaging the attribution
scores of all the sub-tokens in the statement.
%The number after each statement is the statement attribution score,
%which is calculated by averaging the attribution scores of all the
%sub-tokens in the statement.
%
A positive attribution score means that the statement contributes
positively to the model's predicted class, while a negative score
means the statement contributes negatively to the predicted class.
As seen, the two statements that receive the highest scores are the
statement that defines the \code{InputStream} variable and the
statement that invokes the \code{readAllBytes} method call on the
\code{InputStream} object. This example illustrates that the model
is able to put the attention on the right (sub)token
of the input to decide the need of the \code{try-catch} block.


\begin{table}[t]%[htpb]
  \caption{Try-Catch Necessity Checking Evaluated on Test Partitions by the Number Of Try-Catch Blocks (RQ1) }
  \vspace{-12pt}
  \small
	\begin{center}
		\renewcommand{\arraystretch}{1}
		\begin{tabular}{| p{1.0cm}<{\centering} | p{0.7cm}<{\centering} | p{0.7cm}<{\centering}| p{0.7cm}<{\centering} | p{0.7cm}<{\centering} | p{0.7cm}<{\centering} | p{0.7cm}<{\centering} | }
		  \hline
			\multirow{2}{*}{} & \multicolumn{6}{c|}{Number of Try-Catch Blocks (Github dataset)} \\
			\cline{2-7}
			  & Zero  & One & Two & Three & Four & Five\\
			\hline
			Precision & 0.0 &  1.0 & 1.0 & 1.0 & 1.0 & 1.0\\
			\hline
			Recall   & 0.0 &  0.983 & 0.9988 & 1.0 & 0.9861 & 1.0\\
			\hline
			F1   & 0.0  &  0.9914 & 0.9994 & 1.0 & 0.993 & 1.0\\
			\hline
		\end{tabular}
		\label{tab:rq1-detailed-result}
	\end{center}
\end{table}

In addition, we partitioned the test dataset according to the number
of \code{try-catch} blocks, and evaluated {\xblock} on each
partition. As seen in Table~\ref{tab:rq1-detailed-result}, {\xblock}
gives 100\% correct prediction on the partitions with zero, three and
five \code{try-catch} blocks. Moreover, it achieves 100\% precision
and above 0.99 F1-score across all the partitions, showing that
{\xblock}'s prediction ability remains strong regardless of the number
of \code{try-catch} blocks in a code snippet.

%With a Precision of 68\%, it can decide correctly
%2 out of 3 cases if a code snippet needs a \code{try-catch}
%block or not. With a Recall of 79\%, {\tool} covers 4 out of
%5 cases that needs to be placed in a \code{try-catch} block. Users
%just need to find 1 out of 5 cases. As a result, it achieves a high
%F-score of 0.73.
%In FuzzyCatch dataset, {\tool} also achieves a high level of
%performance with XX\% precision, YY\% recall, and ZZ\% F-score.

%the state-of-the-art approach, XRank, {\bf -7.1\%} in Recall, {\bf
%  28.3\%} in Precision, and {\bf 12.3\%} in F-score.

%In FuzzyCatch dataset, the relative improvements are XX\%, YY\%, and
%ZZ\% in precision, recall, and F-score, respectively.

%We examined closely the cases that {\tool} performed better than
%XRank.

%Examining the result, we reported the following. First, if the
%association score of {\em only one API method} in the snippet and {\em one
%exception} is higher than a threshold, XRank decides that a
%\code{try-catch} block is needed.  Thus, it often tends to output
%  ``Yes''. {\em Its recall is slightly
%  better, but precision is just marginally better than a coin toss
%  (0.53) in our balanced dataset. That leads to lower F-score than {\tool}}.
%%
%Second,
%%XRank relies on the association scores between the presence of
%%API method calls and the presence of a \code{try-catch} block.
%the decisions on the necessity of a \code{try-catch} block or the
%exception types depend on the pre-defined thresholds in XRank on those
%association scores. Thus, those pre-defined thresholds might not be
%suitable across the libraries. Third, for the incomplete code
%snippets in which the names of the API methods in different packages
%or libraries are the same (e.g., \code{toString} or \code{getText} in
%various JDK packages), XRank cannot distinguish them and use one entry
%in the dictionary for them due to its IR approach. In contrast, unlike
%XRank which considers only the API method calls in a \code{try-catch}
%block, {\tool} considers the code in the block as the context to learn
%the program dependencies/relations among the names of those API
%elements. That is, it leverages the relations among the names of API
%elements to learn their identities, thus, deciding better the need of
%\code{try-catch} blocks and the corresponding exception types.

%Tien:RQ2 Table
%\begin{table}[t]
%  \caption{Try-Catch Statement Detection Effectiveness (RQ2)}
%  \vspace{-12pt}
%	\begin{center}
%		\small
%		\renewcommand{\arraystretch}{1} 
%		\begin{tabular}{p{0.8cm}<{\centering}|p{0.4cm}<{\centering}|p{0.4cm}<{\centering}|p{0.4cm}<{\centering}|p{0.4cm}<{\centering}|p{0.4cm}<{\centering}|p{0.4cm}<{\centering}|p{0.4cm}<{\centering}|p{0.4cm}<{\centering}|p{0.4cm}<{\centering}|p{0.4cm}<{\centering}}
%			\hline
%			 	&  \multicolumn{10}{c}{Accuracy} \\
%			\cline{2-11}
%			     	&  N1  & N2   &  N3  & N4   &N5    & N6   &N7    & N8   &N9    & N10 \\
%			\hline
%			\tool     & 0.42 & 0.47 & 0.59 & 0.66 & 0.71 & 0.74 & 0.76 & 0.77 & 0.78  & 0.79  \\
%			\hline
%		\end{tabular}
%		Nx is number of nodes in the explanation
%                sub-graph (\code{try-catch} block)
%		\label{tab:rq2}
%	\end{center}
%\end{table}


%Take as an example a code snippet (not shown) in our dataset with the
%presence of \code{getText}. This name is popular with a very large number
%of API method candidates.
%%For example, in a code snippet, \code{getText} has a very large number
%%of API method candidates.
%However, considering the relation between \code{css} and
%\code{getText} in the code \code{`...css()\-.getText()'},~the number
%of candidates for \code{getText} is only 4. Finally, considering~the
%return value of \code{getText} as an argument of
%\code{setInnerText(...)} in the code
%\code{`setInnerText(...css()\-.getText())'}, only one candidate is
%remained:
%\code{com\-.google\-.gwt\-.resources\-.client\-.CssResource\-.getText()}.
%Thus, those relations actually help identify the API elements,
%leading to better decision in {\tool} on the \code{try-catch} block
%and exception types.
%%
%Because it has not seen any \code{try-catch} block involving
%\code{com\-.....getText()} and those related ones, {\tool} decides
%that the code snippet does not need a \code{try-catch} block. In
%contrast, XRank considers only the {\em pairwise} associate scores
%between an {\em individual API method call} and the exception types
%in a \code{catch} clause. It disregards those above
%relations/dependencies among the API names. Thus, it might
%misunderstand that \code{getText} needs a \code{try-catch} due to the
%co-occurrences of other API elements that need one. That is, without
%the dependencies, XRank might make incorrect identification of the API
%elements via their names, leading to incorrect exception
%recommendation.

%considering Groum but only to get better API ...


%\begin{table}[htpb]
%  \caption{Try-Catch Necessity Checking Comparison (RQ1)}
%  \vspace{-12pt}
%	\begin{center}
%		\renewcommand{\arraystretch}{1}
%		\begin{tabular}{p{1.5cm}<{\centering}|p{1.25cm}<{\centering}p{1.25cm}<{\centering}|p{1.25cm}<{\centering}p{1.25cm}<{\centering}}
%			\hline
%			\multirow{2}{*}{} & \multicolumn{2}{c|}{{\tool} Dataset} & \multicolumn{2}{c}{FuzzyCatch Dataset}\\
%			\cline{2-5}
%			  & \tool  & XRank & \tool  & XRank\\
%			\hline
%			Recall    & \textbf{0.81} & &&\\
%			Precision & \textbf{0.66} & &&\\
%			F-score   & \textbf{0.73} & &&\\
%			\hline
%		\end{tabular}
%		\label{tab:xblock}
%	\end{center}
%\end{table}

\subsection{Try-Catch Statement Detection (RQ2)}
\label{sec:rq2}

\begin{table}[t]%[htpb]
  \caption{Try-Catch Statement Detecting Comparison ({\xblock}+{\xstate}, Instance Level) (RQ2)}
%  \caption{Try-Catch Statement Detecting Comparison (Evaluate {\xstate} in Connection with {\xblock} --- Instance Level) (RQ2)}
  \vspace{-12pt}
  \small
	\begin{center}
		\renewcommand{\arraystretch}{1}
		\begin{tabular}{| p{3.05cm}<{\centering} | p{1.2cm}<{\centering} | p{1.2cm}<{\centering}| p{1.2cm}<{\centering}|}
		  \hline
		Instance Level	  & Precision  & Recall & F1-score \\
			\hline
                        %			CodeBERT w/o fine-tuning &  0.0 & 0.0  & 0.0\\
                        GPT-3.5 & 0.550 & 0.232 & 0.326 \\
			\hline
			\xblock + \xstate   & \textbf{0.969}  &  \textbf{0.607} & \textbf{0.747}\\
			\hline
		\end{tabular}
		\label{tab:xstate-1}
	\end{center}
\end{table}

\begin{table}[t]%[htpb]
\caption{Try-Catch Statement Detecting Result ({\xstate} as Individual, Instance Level) (RQ2)}
  \vspace{-12pt}
  \small
	\begin{center}
		\renewcommand{\arraystretch}{1}
		\begin{tabular}{| p{3.05cm}<{\centering} | p{1.2cm}<{\centering} | p{1.2cm}<{\centering}| p{1.2cm}<{\centering}|}
		  \hline
		Instance Level	  & Precision  & Recall & F1-score \\
			\hline
%			GPT-3.5 &   &   & \\
			\hline
			\xstate  & \textbf{1.0}  &  \textbf{0.6198} & \textbf{0.7652}\\
			\hline
		\end{tabular}
		\label{tab:xstate-2}
	\end{center}
\end{table}

Table~\ref{tab:xstate-1} shows the result when we evaluated {\xstate}
in connection with {\xblock}. That is, both individual results from
{\xblock} and {\xstate} must be correct for the instance to be
considered correct. {\tool} achieves a very high precision (96.88\%)
and predicts correctly all the statements in \code{try-catch} blocks
for about 60\% of the positive code instances. As seen, {\tool}
improves relatively over GPT-3.5 {\bf 76.1\%}, {\bf 161.7\%}, and {\bf
  128.7\%} in Precision, Recall, and F1-score. Examining the results,
we found that GPT-3.5 did not work well for the code snippets that
have more than one \code{try-catch} blocks. It also does not recognize
well multiple statements with dependencies that need to be placed in
the same block. {\tool} recognizes well the dependencies among
statements (see RQ4), thus, better grouping them into a
\code{try-catch} block.

%several statements in which some statements are left outside of the
%blocks.

Table~\ref{tab:xstate-2} displays the result when we evaluated
{\xstate} individually (i.e., assuming {\xblock} correctly predicts
the presence of \code{try-catch} blocks). As seen, the numbers are
slightly higher than those for {\xblock}+{\xstate} because there is no
impact from {\xblock}'s result. In other words, {\xstate} manages to
achieve a 100\% precision, showing that {\xstate} is capable of giving
correct predictions for those false negatives from {\xblock}.

%As Table~\ref{tab:xstate-1} shows, when evaluating {\xstate} in
%connection with {\xblock}, {\xstate} achieves high precision score and
%is able to recover statements in try-catch blocks for about 60\% of
%the positive code samples, while the codebert baseline model fails the
%task completely.  Furthermore, we present results of evaluating
%{\xstate} as individual in Table~\ref{tab:xstate-2}. As can be seen,
%the codebert baseline still cannot give any correct predictions. In
%comparison, {\xstate} manages to achieve a 100\% precision, showing
%that {\xstate} is capable of giving correct predictions for those
%false negatives from {\xblock}.

%\begin{table}[t]%[htpb]
\caption{Try-Catch Statement Detecting Comparison (Evaluate {\xstate} As Individual --- Block Level) (RQ2)}
  \vspace{-12pt}
  \small
	\begin{center}
		\renewcommand{\arraystretch}{1}
		\begin{tabular}{| p{3.05cm}<{\centering} | p{1.2cm}<{\centering} | p{1.2cm}<{\centering}| p{1.2cm}<{\centering}|}
		  \hline
			  & Precision  & Recall & F1-score \\
			\hline
			CodeBERT w/o fine-tuning &  0.0 & 0.0  & 0.0\\
			\hline
			\tool   & \textbf{0.4}  &  \textbf{0.6369} & \textbf{0.4914}\\
			\hline
		\end{tabular}
		\label{tab:xstate-2}
	\end{center}
\end{table}

%------------------------

\begin{table}[t]%[htpb]
  \caption{Try-Catch Statement Detecting Comparison ({\xblock}+{\xstate}, Statement Level) (RQ2)}
  \vspace{-12pt}
  \small
	\begin{center}
		\renewcommand{\arraystretch}{1}
		\begin{tabular}{| p{3.05cm}<{\centering} | p{1.2cm}<{\centering} | p{1.2cm}<{\centering}| p{1.2cm}<{\centering}|}
		  \hline
		Statement Level	  & Precision  & Recall & F1-score \\
			\hline
                        GPT-3.5 & {\bf 0.XX} & {\bf 0.YY} & {\bf 0.ZZ} \\
			\hline
			\xblock + \xstate   & \textbf{0.XX}  &  \textbf{0.YY} & \textbf{0.ZZ}\\
			\hline
		\end{tabular}
		\label{tab:xstate-3}
	\end{center}
\end{table}

\begin{table}[t]%[htpb]
\caption{Try-Catch Statement Detecting Result ({\xstate} as Individual, Statement Level) (RQ2)}
  \vspace{-12pt}
  \small
	\begin{center}
		\renewcommand{\arraystretch}{1}
		\begin{tabular}{| p{3.05cm}<{\centering} | p{1.2cm}<{\centering} | p{1.2cm}<{\centering}| p{1.2cm}<{\centering}|}
		  \hline
		Statement Level	  & Precision  & Recall & F1-score \\
			\hline
%			GPT-3.5 &   &   & \\
			\hline
			\xstate  & \textbf{0.XX}  &  \textbf{0.YY} & \textbf{0.ZZ}\\
			\hline
		\end{tabular}
		\label{tab:xstate-4}
	\end{center}
\end{table}

Table~\ref{tab:xstate-3} displays the result at the statement level
(i.e., whether a statement needs to be inside a \code{try-catch} block
or not). As seen, {\xblock}+{\xstate} improves relatively over GPT-3.5
{\bf XX.X\%}, {\bf YY.Y\%}, and {\bf ZZ.Z\%} in Precision, Recall, and
F1-score, respectively in prediction at the statement level. As seen
in Table~\ref{tab:xstate-4}, {\tool} also achieves high numbers in
prediction at the statement level as individual.

%--------------------------
%Table~\ref{tab:rq2} displays the result on detecting the statements
%that need to be placed in a \code{try-catch} block. $N_x$ is a parameter
%in GNNExplainer that defines the number of nodes in the explanation
%graph $\mathcal{G}_C$, i.e., the number of statements to be placed in the
%\code{try-catch} block.
%%
%As the number of nodes (statements) in $\mathcal{G}_C$ increases, the
%number of correct statements covered also increases, thus, accuracy
%increases. However, as the number of statements increases higher than
%5, accuracy increases more slowly. In our dataset, the average size of
%a \code{try-catch} block is 5.9 statements. As seen, the accuracy as
%$N_x$=6 is 74\%. That is, by pointing out 6 statements on average,
%{\tool} can correctly suggest 74\% of the total number of statements
%in the dataset that need to be placed in \code{try-catch}
%blocks. That is, it points out correctly 4.5 out of 6 statements to be
%in a \code{try-catch} block. For the statements that do not need to be
%placed in a \code{try-catch} block, {\tool} predicts correctly with
%63\% accuracy (not shown).
%
%\begin{figure}[t]
	\centering
	\lstset{
		numbers=left,
		numberstyle= \tiny,
		keywordstyle= \color{blue!70},
		commentstyle= \color{red!50!green!50!blue!50},
		frame=shadowbox,
		rulesepcolor= \color{red!20!green!20!blue!20} ,
		xleftmargin=1.5em,xrightmargin=0em, aboveskip=1em,
		framexleftmargin=1.5em,
                numbersep= 5pt,
		language=C,
    basicstyle=\scriptsize\ttfamily,
    numberstyle=\scriptsize\ttfamily,
    emphstyle=\bfseries,
                moredelim=**[is][\color{red}]{@}{@},
		escapeinside= {(*@}{@*)}
	}
\begin{lstlisting}[]
int ret = -1;
try {
  FileInputStream fin = new FileInputStream(path);
  int length = fin.available();
  byte[] buf = new byte[length];
  fin.read(buf);
  ret = loadFromBuffer(buf);
  fin.close();
} catch (FileNotFoundException e) {
    Log.e(TAG, "error:" + e);
    e.printStackTrace();
} catch (IOException e) {
    Log.e(TAG, "error:" + e);
    e.printStackTrace();
}
return ret;
\end{lstlisting}
        \vspace{-16pt}
        \caption{Correct Exception Handling Suggestion by {\tool}}
        \label{fig:example-experiment}
\end{figure}

%
%\vspace{2pt}
%\noindent {\bf Example.} Figure~\ref{fig:example-experiment} displays
%an example that {\tool} made correct suggestions. {\bf First}, it made
%a correct suggestion on the need of a \code{try-catch} block for the
%code at lines 1--8, 16.  {\bf Second}, GNNExplainer pointed out that
%{\xblock} used all the statements at lines 3--8 for such correct
%prediction. As a consequence, {\tool} correctly suggests to place
%lines 3--8 into a \code{try-catch} block. Note that, it also correctly
%pointed out that lines 1 and 16 do not need to be inside the
%\code{try-catch} block.  {\bf Third}, GNNExplainer gives three
%statements at lines 3, 6, and 8 highest scores. We can see that those
%lines contain three crucial API method calls: 1)
%\code{FileInputStream}, 2) \code{read}, and 3) \code{close}. {\bf
%  Fourth}, those three lines have data and control dependencies, which
%could help the model learn the identities of the API elements via
%their names \code{FileInputStream}, \code{read}, and \code{close},
%despite that the code snippet does not have the fully-qualified names
%for those API elements. This confirms the need of integrating {\em program
%  dependencies} in our solution. {\bf Finally}, {\tool} was also able
%to learn from the training corpus that those names refer to those API
%elements, which often correspond to the following exception types: 1)
%\code{FileInputStream} with \code{FileNotFoundException}, and 2)
%\code{FileInputStream.read} and \code{FileInputStream.close} with
%\code{IOException}.


%Note that {\tool} via {\xstate} predicts the statements to be placed
%in the \code{try-catch} block only after {\xblock} predicted that the
%given code needs such a block. Therefore, the incorrect cases from
%{\xblock} (i.e., those cases that need to be in a \code{try-catch}
%block but were predicted not) are also counted as incorrect in
%{\xstate}.

%{\color{red}{N1-N10 are the number of nodes that the subgraph contains which means the size of the try-catch block. Because the average size of the try-catch is 5.7, I currently pick 6 as the size of the try-catch block. The accuracy here is defined as: if a statement is in the try-catch block and our model put it in the subgraph, I regard it is correct $S_c$. All other conditions, I think they are incorrect. If there are $S$ statements in the try-catch block, the total accuracy is calculated as $S_c/S$. Later, I will add an example showing that the statement that our model predicted in the try-catch block contains the method call which lead to the correct exception types prediction.}}

\subsection{RQ3. Effectiveness on Exception Type Recommendation}
\label{sec:rq3}


\begin{table}[t]
	\caption{RQ3. Detailed Performance Comparison on Different \# of Statements in an Oracle Try-Catch Block (Recall)}
	\tabcolsep 2pt
	{\small
		\begin{center}
			\renewcommand{\arraystretch}{1}
			\begin{tabular}{p{3cm}<{\centering}|p{2cm}<{\centering}|p{2cm}<{\centering}}
				\hline
				\# ET in Oracle Block& Metrics& {\textsc{\tool}\xspace} \\
				\hline
				\multirow{1}{*}{1 (40376)}   & Hit-1  & 25841 (64\%) \\
				\hline
				\multirow{2}{*}{2 (2634)}  & Hit-1   & 2028 (77\%) \\
				& Hit-2         &  1081 (41\%) \\
				\hline
				\multirow{3}{*}{3 (547)}  & Hit-1    & 334 (61\%) \\
				& Hit-2     & 181 (33\%)\\
				& Hit-3     & 115 (21\%) \\
				\hline
				\multirow{4}{*}{3+ (391)}  & Hit-1   & 233 (59\%) \\
				& Hit-2     & 134 (35\%) \\
				& Hit-3     & 65 (17\%))\\
				\hline
			\end{tabular}		
		ET: Exception types
			\label{RQ3_results_1}
		\end{center}
	}
\end{table}

{\color{red}{1. Our model do well on hit-1 condition which means in most cases, our model can at least predict one exception type correctly for a try-catch block. 2. Results show that the more exception types one try-catch block has, our model is harder to predict all of them correctly. }}


\begin{table}[t]
	\caption{RQ3. Detailed Performance Comparison on Different \# of Statements in a Predicted Try-Catch Block (Precision)}
	\vspace{-10pt}
	\tabcolsep 2pt
	{\small
		\begin{center}
			\renewcommand{\arraystretch}{1}
			\begin{tabular}{p{3cm}<{\centering}|p{2cm}<{\centering}|p{2cm}<{\centering}}
				\hline
				\# ET in Predicted Block & Metrics & {\textsc{\tool}\xspace} \\
				\hline
				\multirow{1}{*}{1 (36995)}   & Hit-1  & 13589 (37\%) \\
				\hline
				\multirow{2}{*}{2 (4981)}  & Hit-1   & 1594 (32\%) \\
				& Hit-2       						& 947 (19\%) \\
				\hline
				\multirow{3}{*}{3 (1972)}  & Hit-1    & 611 (31\%) \\
				& Hit-2         					& 458 (23\%)\\
				& Hit-3         				  	& 185 (9\%) \\
				\hline
			\end{tabular}
		ET: Exception types
			\label{RQ3_results_2}
		\end{center}
	}
\end{table}
\subsection{Ablation Study (RQ4)}
\label{sec:rq4}



\begin{table}[t]
  \caption{Impact of Different Features on {\xblock} (RQ4)}
    %Try-Catch Block Necessity Checking}
  \vspace{-12pt}
  \begin{center}
    \small
		\renewcommand{\arraystretch}{1}
		\begin{tabular}{p{1.75cm}<{\centering}|p{1.75cm}<{\centering}|p{1.75cm}<{\centering}|p{1.75cm}<{\centering}}
			\hline
			  & \tool w/o SOT & \tool w/o AST & \tool \\
			\hline
			Recall    & 0.69 & 0.71 & \textbf{0.79} \\
			Precision & 0.64 & 0.57 &\textbf{0.68} \\
			F-score   & 0.66 & 0.63 &\textbf{0.73} \\
			\hline
		\end{tabular}
		SOT: Sequence of tokens; AST: Abstract Syntax Tree
		\label{tab:sensi-xblock}
	\end{center}
\end{table}

\subsubsection{Impact of Different Features on Try-Catch Block Necessity Checking ({\xblock})}

Table~\ref{tab:sensi-xblock} displays the results when we removed the
two key features in {\tool} and measured {\xblock}'s performance. As
seen, without the sequences of code (lexical tokens) for each
statement, Recall, Precision, and F-score decrease 12.7\%, 5.9\%, and
9.6\%, respectively. Without considering the AST structure, Recall,
Precision, and F-score also decrease even further with 10.1\%, 16.2\%, and
13.7\%, respectively. In other words, {\em the code structure has a higher
contribution than the lexical values of code tokens}.


%{\color{red}{Without sequence of tokens or AST in try-catch block necessity checking will lead to the reduce of results. Once I finish this sensitivity experiments, I will update this table. }}

\begin{table}[t]
  \caption{Impact of Different Features on {\xstate} (RQ4)}
  \vspace{-12pt}
    %\code{Try-catch} Statement Prediction}
  \small
	\begin{center}
		\renewcommand{\arraystretch}{1}
		\begin{tabular}{p{1.75cm}<{\centering}|p{1.75cm}<{\centering}|p{1.75cm}<{\centering}|p{1.75cm}<{\centering}}
			\hline
			  & \tool w/o SOT & \tool w/o AST & \tool \\
			\hline
			Accuracy    & 0.75 & 0.73 & \textbf{0.74} \\
			\hline
		\end{tabular}
		SOT: Sequence of tokens; AST: Abstract Syntax Tree
		\label{tab:sensi-xstate}
	\end{center}
\end{table}

\subsubsection{Impact of Different Features on Try-Catch Statement Detection ({\xstate})}

%Table~\ref{tab:sensi-xstate} displays the results when we removed the
%two key features in {\tool} and measured the performance of {\xstate}.
We used six as the limit on the number of nodes in the explanation
sub-graph in which {\tool} achieves 0.74 of Accuracy. Note: the
average number of the statements in a \code{try-catch} block in our
dataset is 5.9. As seen, the result in Table~\ref{tab:sensi-xstate} is
consistent with Table~\ref{tab:sensi-xblock}, as AST contributes
slightly more than code sequences.

%without the sequences of code (lexical tokens) for each statement,
%Accuracy decreases XX\%. Without considering the AST structure,
%Accuracy also decreases even further with XX\%. In other words, the
%code structure has a higher contribution than the lexical values of
%code tokens.

\subsubsection{Impact of Different Features on Exception Type Recommendation ({\xtype})}

Tables~\ref{tab:sensi-xtype-recall}
and~\ref{tab:sensi-xtype-precision} display the results. As seen, the
impact result is consistent with those in
Tables~\ref{tab:sensi-xblock} and~\ref{tab:sensi-xstate}.
Thus, code structure has slightly higher impact than code sequence.

%when we removed the two key features in {\tool} and measured the
%performance of {\xtype}. As seen, without the sequences of code
%(lexical tokens) for each statement, all the evaluation metrics
%\code{Hit}-$n$ decrease across the board in both recall and
%precision. Without considering the AST structure, all the evaluation
%metrics decrease slightly further.  Thus, this result is consistent
%with the above results as the code structure has a higher contribution
%than the lexical values of code tokens.

\begin{table}[t]
  \caption{Impact of Different Features on {\xtype} (Recall)}
  \vspace{-12pt}
	\tabcolsep 2pt
	{\small
		\begin{center}
			\renewcommand{\arraystretch}{1}
			\begin{tabular}{p{2cm}<{\centering}|p{1cm}<{\centering}|p{1.5cm}<{\centering}|p{1.5cm}<{\centering}|p{1.5cm}<{\centering}}
				\hline
				\# ET in \code{Try-Catch} Block in Oracle & Metrics &{\textsc{\tool w/o SOT}\xspace}&{\textsc{\tool w/o AST}\xspace}& {\textsc{\tool}\xspace} \\
				\hline
				\multirow{1}{*}{1 (1,385 instances)} & Hit-1  &833 (60\%)& 791 (57\%)& 918 (66\%) \\
				\hline
				\multirow{2}{*}{2 (98 instances)}    & Hit-1  &69 (70\%)& 66 (67\%)& 75 (77\%) \\
				                                     & Hit-2  &39 (40\%)& 37 (38\%)&  41 (42\%) \\
				\hline
				\multirow{3}{*}{3 (18 instances)}    & Hit-1  &10 (56\%)&11 (61\%)& 12 (67\%) \\
				                                     & Hit-2  &4 (22\%)& 4 (22\%)& 5 (28\%)\\
				                                     & Hit-3  &2 (11\%)& 3 (17\%)& 3 (17\%) \\
				\hline
				\multirow{4}{*}{3+ (30 instances)}   & Hit-1  &16 (53\%)&15 (50\%)& 18 (60\%) \\
				                                     & Hit-2  &9 (30\%)& 8 (27\%)& 10 (33\%) \\
				                                     & Hit-3  &3 (10\%)& 2 (7\%)& 4 (13\%)\\
				\hline
			\end{tabular}
		ET:Exception types; SOT: Sequence of tokens; AST: Abstract syntax tree		
			\label{tab:sensi-xtype-recall}
		\end{center}
	}
\end{table}

\begin{table}[t]
	\caption{Impact of Different Features on {\xtype} (Precision)}
	\vspace{-10pt}
	\tabcolsep 2pt
	{\small
		\begin{center}
			\renewcommand{\arraystretch}{1}
			\begin{tabular}{p{2cm}<{\centering}|p{1cm}<{\centering}|p{1.5cm}<{\centering}|p{1.5cm}<{\centering}|p{1.5cm}<{\centering}}
				\hline
				\# ET in Predicted \code{Try-Catch} Block & Metrics &{\textsc{\tool w/o SOT}\xspace}&{\textsc{\tool w/o AST}\xspace}& {\textsc{\tool}\xspace} \\
				\hline
				\multirow{1}{*}{1 (1,154 instances)}   & Hit-1  &357 (31\%)& 309 (27\%)& 442 (38\%) \\
				\hline
				\multirow{2}{*}{2 (294 instances)}  & Hit-1   &83 (28\%)& 78 (27\%)& 90 (31\%) \\
				& Hit-2       						&57 (19\%)& 51 (17\%)& 61 (21\%) \\
				\hline
				\multirow{3}{*}{3 (83 instances)}  & Hit-1    & 24 (29\%)& 21 (25\%)& 27 (33\%) \\
				& Hit-2         					&19 (23\%)& 18 (22\%)& 23 (28\%)\\
				& Hit-3         				  	&6 (7\%)& 4 (5\%)& 6 (7\%) \\
				\hline
			\end{tabular}
		ET: Exception types; SOT: Sequence of tokens; AST: Abstract syntax tree
			\label{tab:sensi-xtype-precision}
		\end{center}
	}
\end{table}

%{\color{red}{Without sequence of tokens or AST in exception type prediction will lead to the reduce of results of all hit-1, hit-2, and hit-3. Once I finish this sensitivity experiments, I will update these two table. }}


%%%\begin{table}[t]
%%%	\caption{RQ4. Impact of the \# Limit of Exception Type on Exception Type Prediction (Recall)}
%%%	\tabcolsep 2pt
%%%	{\small
%%%		\begin{center}
%%%			\renewcommand{\arraystretch}{1}
%%%			\begin{tabular}{p{3.5cm}<{\centering}|p{2cm}<{\centering}|p{2.5cm}<{\centering}}
%%%				\hline
%%%				\# ET in Oracle Block& Metrics& {\textsc{\tool}\xspace} \\
%%%				\hline
%%%				\multirow{1}{*}{1 (40376)}   & Hit-1  & 27867 (69\%) \\
%%%				\hline
%%%				\multirow{2}{*}{2 (2634)}  & Hit-1   & 2131 (81\%) \\
%%%				& Hit-2         &  1162 (44\%) \\
%%%				\hline
%%%				\multirow{3}{*}{3 (547)}  & Hit-1    & 353 (64\%) \\
%%%				& Hit-2     & 189 (35\%)\\
%%%				& Hit-3     & 142 (26\%) \\
%%%				\hline
%%%				\multirow{4}{*}{4 (267)}  & Hit-1   & 175 (65\%) \\
%%%				& Hit-2     & 107 (40\%) \\
%%%				& Hit-3     & 55 (21\%))\\
%%%				& Hit-4     & 21 (8\%))\\
%%%				\hline
%%%				\multirow{4}{*}{4+ (124)}  & Hit-1   & 77 (62\%) \\
%%%				& Hit-2     & 46 (38\%) \\
%%%				& Hit-3     & 21 (17\%))\\
%%%				& Hit-4     & 5 (4\%))\\
%%%				\hline
%%%			\end{tabular}		
%%%			ET: Exception types
%%%			\label{RQ4_results_4}
%%%		\end{center}
%%%	}
%%%\end{table}

%%%\begin{table}[t]
%%%	\caption{RQ4. Impact of the \# Limit of Exception Type on Exception Type Prediction (Precision)}
%%%	\vspace{-10pt}
%%%	\tabcolsep 2pt
%%%	{\small
%%%		\begin{center}
%%%			\renewcommand{\arraystretch}{1}
%%%			\begin{tabular}{p{3.5cm}<{\centering}|p{2cm}<{\centering}|p{2.5cm}<{\centering}}
%%%				\hline
%%%				\# ET in Predicted Block & Metrics & {\textsc{\tool}\xspace} \\
%%%				\hline
%%%				\multirow{1}{*}{1 (34707)}   & Hit-1  & 11113 (32\%) \\
%%%				\hline
%%%				\multirow{2}{*}{2 (5892)}  & Hit-1   & 1829 (31\%) \\
%%%				& Hit-2       						& 764 (13\%) \\
%%%				\hline
%%%				\multirow{3}{*}{3 (2321)}  & Hit-1    & 654 (28\%) \\
%%%				& Hit-2         					& 437 (19\%)\\
%%%				& Hit-3         				  	& 159 (7\%) \\
%%%				\hline
%%%				\multirow{4}{*}{4 (1028)}  & Hit-1    & 268 (26\%) \\
%%%				& Hit-2         					& 211 (21\%)\\
%%%				& Hit-3         				  	& 94 (9\%) \\
%%%				& Hit-4         				  	& 33 (3\%) \\
%%%				\hline
%%%			\end{tabular}
%%%			ET: Exception types
%%%			\label{RQ4_results_5}
%%%		\end{center}
%%%	}
%%%\end{table}

\subsection{Exception-Related Bug Detection (RQ5)}
\label{sec:rq1}

\begin{table}[t]%[htpb]
  \caption {Exception-Related Bug Detection (RQ5)}
  \vspace{-12pt}
  \small
	\begin{center}
		\renewcommand{\arraystretch}{1}
		\begin{tabular}{|p{1.75cm}<{\centering}|p{1.75cm}<{\centering}|p{1.75cm}<{\centering}|}
		  \hline
			\multirow{2}{*}{} & \multicolumn{2}{c|}{FuzzyCatch Dataset} \\
			\cline{2-3}
			  & \tool  & FuzzyCatch~\cite{xrank-fse20} \\
			\hline
			Recall    & 0.75 & \textbf{0.79}\\
			Precision & \textbf{0.62} & 0.54\\
			F-score   & \textbf{0.68} & 0.64\\
			\hline
		\end{tabular}
		\label{tab:bug}
	\end{center}
\end{table}

%{\color{red}{This section waiting for the XRank Results. But from the current estimate, our approach should have higher F-score. But the recall and precision I'm not sure. Once I have the results, I will update this section.}}

Table~\ref{tab:bug} displays the comparison result. As seen, {\tool}
can be used to detect well real-world exception-related bugs in which
the code snippet needs but did not have the \code{try-catch} blocks or
miss some exceptions. In comparison, {\tool} improves over the
state-of-the-art FuzzyCatch, an IR approach, {\bf 14.8\%} in Precision,
{\bf -5.1\%} in Recall, and {\bf 5.8\%} in F-score. The examples of buggy
code and fixes with the addition of \code{try-catch} blocks are
available at FuzzyCatch's code repository: ebrand.ly/ExDataset.

%\vspace{3pt}

\subsubsection*{\bf Limitations and Threats to Validity}
%First, {\tool} can not handle the code with multiple \code{try-catch}
%blocks.
First, {\tool} cannot generate new exception types that were not in
the training corpus. Second, it does not support the generation of
exception handling code inside the \code{catch} body. Third, {\tool}
needs training data, thus, does not work for a new library without any
API usage~yet. {\tool} is specifically for Java. Our collected data
might not be representative. However, we use well-established projects
and libraries. FuzzyCatch~\cite{xrank-fse20} does not
suggest \code{try}-block statements and exceptions, thus, we compared
only with XRank (part of FuzzyCatch).



\section{Related Work}
\label{sec:related}

The automated approaches to recommend exception handling can be
classified into four categories as presented in Section~\ref{sec:intro}.
%The first category of approaches defines the {\em heuristics} on
%exception types, API calls, and variable types to recommend proper
%exception handling~\cite{barbosa-bsse12}.
%%These heuristic-based approaches do not always work in all the cases
%%and need to be updated for the new/updated API elements or a new
%%library. To address the heuristics,
%The second category of approaches relied on the enforced {\em
%  exception handling policies}~\cite{barbosa-tse16,barbosa-saner18}.
%%Exception policies are defined via domain-specific
%%language~\cite{barbosa-tse16}.
%%Exception Policy Expert (EPE)~\cite{barbosa-saner18} is a tool
%%embedded in Eclipse IDE that warns developers about exception policy
%%violations. The drawback of this category is the hard-code of the
%%policies in the EPE tool.
%%To overcome the pre-defined exception handling policies,
%The third category of approaches aim to {\em mine the frequent
%  exception handling} code from a large corpus. These mining
%approaches~\cite{chanchal-scam14} provide more flexibility than policy
%enforcement.
%%They rely on the idea that similar code has similar exception
%%handling. As with mining, these approaches face the challenge of
%%deterministically setting the threshold for similar code and similar
%%exception handling. In many cases, the two pieces of code are quite
%%different, yet have the same set of exceptions to be handled.
The closest work to {\tool} is the state-of-the-art {\em information
  retrieval} (IR) approaches~\cite{xrank-fse20}, which provides more
flexibility than the others. XRank~\cite{xrank-fse20} recommends a
ranked list of API calls that might need exception handling and
XHand~\cite{xrank-fse20} recommends exception handling code. Both
leverages fuzzy set theory to compute the associations between API
method calls and the exception types. This direction has three key
limitations. First, one needs to pre-define a threshold for feature
matching for the retrieval of API elements or exception types. Second,
the IR techniques are not flexible as the ML approaches because they
use the lexical values of API simple names. Thus, they suffer the
ambiguity in the names of API elements in incomplete code
snippets. Lastly, XRank/XHand considers only pairwise associations
between the API method calls and exceptions. It disregards the
surrounding code context and the dependencies/relations. XRank/XHand
simply uses Groum~\cite{fse09}, a dependency graph among API elements,
to collect the API calls, but did not use dependencies in
computing the association scores.
%among the API elements to disambiguate the simple names of the API
%elements.

In addition to exception handling recommendation research,
ThEx~\cite{throw-ase22} predict which exception(s) shall be thrown
under a given programming context. ThEx learns a classification model
from existing thrown exceptions in different contexts.

%The features include code information surrounding the thrown
%exceptions, such as the thrown locations and related variable names.



\section{Conclusion}

{\tool} is the first neural network approach to automated exception
handling recommendation in three tasks for both complete as well as
partial code. It is designed to capture the basic insights to overcome
key limitations of the state-of-the-art IR approaches. With the
learning-based approach, it does not rely on a pre-defined threshold
for explicit feature matching. The dependencies and context help
{\tool} learn the identities of API elements to avoid name ambiguity
and to learn the connections between them and the corresponding
exception types. Our empirical evaluation shows that {\tool} improves
over the state-of-the-art approaches in both intrinsic task and
extrinsic one in exception-related bug detection.


\newpage

\balance

%%
%% The acknowledgments section is defined using the "acks" environment
%% (and NOT an unnumbered section). This ensures the proper
%% identification of the section in the article metadata, and the
%% consistent spelling of the heading.
%\begin{acks}
%To Robert, for the bagels and explaining CMYK and color spaces.
%\end{acks}

%%
%% The next two lines define the bibliography style to be used, and
%% the bibliography file.


%\bibliographystyle{ACM-Reference-Format}
%\bibliography{sample-base}

\bibliographystyle{ACM-Reference-Format}

\bibliography{references, Reference, icse21IntVD, FL, yuchen}

%%
%% If your work has an appendix, this is the place to put it.


\end{document}
\endinput
%%
%% End of file `sample-sigconf.tex'.
