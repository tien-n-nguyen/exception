\section{{\xtype}: Exception Type Recommender}
\label{sec:type}

{\xtype} uses the same CodeBERT that is shared  among other modules. We expect CodeBERT to learn the relation between statements in the same try-block, 
as well as the relation between the try-block and the corresponding exceptions that need to be caught. 
We take the vector outputs of \texttt{[SEP]} tokens, from CodeBERT, that correspond to statements in the same try block. Then, we add them together 
to get a vector representation for the try block and put it 
into



%\begin{figure}[t]
%\begin{center}
%\includegraphics[width=3in]{xtype-3.png}
%\vspace{-10pt}
%\caption{Exception Type Recommendation ({\xtype})}
%\label{fig:xtype}
%\end{center}
%\end{figure}
%
%This section describes {\xtype} that recommends the exception types to
%be handled in the \code{catch} clause for the given code, after the code was
%determined to require a \code{try-catch} block and some of its
%statements were determined to be placed in the \code{try-catch} block.
%
%We use another R-GCN model that acts as a classifier for different
%exception types (Figure~\ref{overview}). For training, the source code
%with \code{try-catch} blocks is used as input. {\tool} parses and
%processes them in the same way to build the PDG subgraph and the
%feature vector representations for the statements as in {\xblock}
%(Figure~\ref{fig:feature}). However, in prediction, the input is the
%sub-graph $\mathcal{G}_C$ of the PDG because the GNNExplainer already
%determined that those statements in that sub-graph need to be placed
%in a \code{try-catch} block. In this case, the feature vectors for the
%sub-graph are also built in the same way as in
%Figure~\ref{fig:feature}. The R-GCN model processes the input
%(sub)graph in a similar manner as in Figure~\ref{fig:gcn} except for the
%processing on the model's output. Instead of connecting
%the output of the R-GCN model to a fully connected layer, we
%feed that output to multiple softmax functions to act as the
%classifiers for the exception types in the dictionary (e.g.,
%\code{IOException}). Each classifier is responsible for one exception
%type.  We could set the maximum number of exception types. The
%positive output from a classifier indicates the presence of the
%corresponding exception type in the \code{catch} clause.
