\section{Exception Type Recommender}
\label{sec:type}

\begin{figure}[t]
\begin{center}
\includegraphics[width=3in]{xtype}
\vspace{-10pt}
\caption{Exception Type Recommendation ({\xtype})}
\label{fig:xtype}
\end{center}
\end{figure}

This section describes {\xtype} that recommends the exception types to
be handled in the \code{catch} clause for the given code, after it was
determined to require a \code{try-catch} block and some of its
statements was determined to be placed in that block.

We use another R-GCN model that acts as a classifier for different
exception types (Figure~\ref{overview}). For training, the source code
with \code{try-catch} blocks is used as input. {\tool} parses and
processes them in the same way to build the feature vector
representations for the statements as in {\xblock}
(Figure~\ref{fig:feature}). However, in prediction, the input is the
sub-graph $\mathcal{G}_C$ of the PDG because the GNNExplainer already
determined that those statements in that sub-graph need to be placed
in a \code{try-catch} block. In this case, the feature vectors are
also built in the same way as in Figure~\ref{fig:feature}. The R-GCN
model processes the input (sub)graph in a similar manner as in
Figure~\ref{fig:gcn} except the post-process on the output of that
model.  Instead of connecting the output vector of the R-GCN model to
a fully connected layer, we feed that output to multiple softmax
functions to act as the classifiers for multiple exception types in
the dictionary. Each classifier is responsible for one exception type.
We could limit the maximum number of exception types. The positive
output from a classifier indicates the presence of the corresponding
exception type in the \code{catch} clause.
